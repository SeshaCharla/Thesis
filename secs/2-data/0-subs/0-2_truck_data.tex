% =============================================================
\begin{table}[!ht]
\centering
\caption{Truck Data Variables}
\label{tab::truck_data_variables}
\begin{tabular}{l l l l}
\hline \hline
Data Name & Units & Variable & Description \\ \hline \hline
\hline \hline
tod & s & $t$ & Time\\
pSCRBedTemp & $\lx{^o}{C}$ & $T$ & SCR-in Gas Temperature\\
pExhMF & g/s & $F$ & Exhaust Flow Rate\\
pUreaDosing & ml/sec & $u_{inj}$ & DEF (Urea Sol.) Dosing Rate\\
pNOxOutppm & ppm & $\con{NO_x}^{out}_{\chi}$ & Tailpipe $NO_x$ (cross-sensitive)\\
pNOxInppm & ppm & $\con{NO_x}^{in}$ & Engine-Out $NO_x$\\
\hline
\hline
\end{tabular}
\end{table}
% =============================================================

In contrast, the truck data comprise on-board sensor measurements from four long-haul trucks collected during the study
period (Table~\ref{tab::truck_data_mileage}). FTIR sensor measurements for species concentrations and tailpipe ammonia
measurements are not available. The data sampling frequency is 1 Hz. The operating temperatures
(Figure~\ref{fig::truck_data_temp_profiles}) recorded in the truck data range from 200 to 300 $\lx{^o}{C}$, which is similar to
the hFTP drive cycles of the test-cell data. However, the inlet and outlet $NO_x$ concentrations and urea dosing
dynamics more closely resemble those observed in the RMC cycles. The truck data variables are summarized in
Table~\ref{tab::truck_data_variables}.
%===
\begin{figure}[!ht]
    \centering
    \includegraphics[width=0.8\textwidth]{./figs/2-data/truck_temperature_range.png}
    \caption{Temperature Profiles of Truck Data}
    \label{fig::truck_data_temp_profiles}
\end{figure}
%====

The low sampling frequency and absence of ammonia measurements present significant challenges for developing a model of
the reacting flow dynamics and for identifying model parameters. Furthermore, the model's nonlinearity results in the
capture of dynamics at different frequencies due to discrepancies in sampling rates between truck and test-cell data.
This issue can be mitigated by downsampling the test-cell data to 1 Hz, leading to the loss of high-frequency dynamic
information. Conversely, upsampling the truck data is not feasible because the missing high-frequency information cannot
be recovered through interpolation.

\begin{table}[!ht]
        \centering
        \caption{Truck Data Set Years and Mileage Difference}
        \label{tab::truck_data_mileage}
        \begin{tabular}{c c c c}
        \hline \hline
         Truck    & Earlier Data & Latter Data & Mileage \\
         Code     &  Year            & Year    & Difference\\ \hline \hline
        Truck A & $2015$ & $2017$ & $2.8 \times 10^5$\\
        Truck B & $2015$ & $2018$ & $11.9 \times 10^5$\\
        Truck C & $2015$ & $2017$ & $5.9 \times 10^5$\\
        Truck D & $2015$ & $2016$ & $6.6 \times 10^5$ \\
        \hline \hline
        \end{tabular}
\end{table}

% Truck Mapping
% Truck A - AD Transport (adt)
% Truck B - Mesilla Valley (mes)
% Truck C - Werner (wer)
% Truck D - Transwest (trw)
% =====

