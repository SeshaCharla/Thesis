\section{Data Units}
The available test and truck data used for evaluating the model structure must be converted to common units. The
following table lists the units used for each of the physical properties.
\begin{table}[!ht]
   \centering
   \caption{Measurement Units used in the Model}
   \begin{tabular}{l l}
       \hline \hline
        \itbf{Property} & \itbf{Unit}\\
        \hline \hline
        Concentration   & $ 10^{-3} \, mol/m^{3}$ \\
        Temperature     & $ \times 10 + 200\,^{\circ}\text{C}$ \\
        Flow Rate   & $\times 10 \, g/s$ \\
        Urea Injection  & $\times 10^{-1}\, ml/s$ \\
        Time            & $s$ \\
        \hline \hline
   \end{tabular}
\end{table}

\subsection{Density of the exhaust gas}
The density of the exhaust flow is assumed to be the density of air at that temperature and ambient atmospheric
pressure. Using the ideal gas law:
\begin{align}
    \rho &= \frac{PM}{R T} = \frac{\mu}{T}
\end{align}
\begin{align*}
    \text{where, } &\\
    P &= \text{Pressure of the exhaust gas (ambient pressure)} = 101.325 \: kPa\\
    M &= \text{Molecular weight of the exhaust gas} = 28.9652 \: g/mol\\
    T &= \text{Temperature of the exhaust gas in Kelvin}\\
    R &= \text{Universal gas constant} = 8.314 \: J/(mol.K)
\end{align*}
\itbf{Note}: $P$ and $M$ are replaced with $(P_1 M_1 + P_2 M_2)$ when humidity of the exhaust gas is considered.
\subsubsection{$\%$ Change in density for the temperature range of operation}
We have,
\begin{align*}
    \frac{\delta \rho}{\rho} &= -\frac{\delta T}{T}
\end{align*}
In general the operating temperature is very high ($250 \,^0 C \approx 500 K$), and most of the data lies in $\pm 100 \, ^0 C$ region. Thus, the maximum change in density is less than $10\%$, whose effect on flow rate is far smaller compared to the change in mass flow rate itself. Thus, density can be assumed to be a constant for this process.
\begin{align}
    \rho &= \rho_0
\end{align}
%%==================
\subsubsection{Sensitivity of density to change in $NO_x$ concentrations}
Let, $P_{NO_x}$ be the partial pressure of $NO_x$ and $P_0$ be the total pressure. Assuming the rest of the gas constituents are similar to that of air, we have the density as:
\begin{align*}
    \rho &= \frac{1}{RT} \lr{ P_{NO_x} M_{NO_x} + (P_0 - P_{NO_x}) M_{air}} = \frac{1}{RT} \lr{ P_{NO_x} \lr{M_{NO_x} - M_{air}} + P_0 M_{air}}\\
    \implies \frac{\partial \rho}{\partial P_{NO_x}} &= \frac{1}{RT} \lr{\lr{M_{NO_x} - M_{air}}}\\
    \implies \frac{\delta \rho}{\rho} &= \frac{\lr{\lr{M_{NO_x} - M_{air}} \delta P_{NO_x}}}{\lr{ P_{NO_x} \lr{M_{NO_x} - M_{air}} + P_0 M_{air}}} \qquad \text{at constant temperature}
\end{align*}
Also,
\begin{align*}
    P_{NO_x} &= P_0 \times \frac{n_{NO_x}}{n_{NO_x} + n_{air}} = P_0 \times \frac{\con{NO_x}}{\frac{n_{NO_x} + n_{air}}{V}} = P_0 \times \frac{\con{NO_x}}{N_a (=1)} \qquad \lrb{\because 1 \, mole = n_{tot}/V}
\end{align*}
Thus,
\begin{align*}
    \frac{\delta \rho}{\rho} &= \frac{\lr{\lr{M_{NO_x} - M_{air}} \delta \con{NO_x}}}{\lr{ \con{NO_x}
 \lr{M_{NO_x} - M_{air}} + M_{air}}}
    \qquad \text{at constant temperature}
\end{align*}
We have,
\begin{align*}
    &M_{NO_x} \approx 46 \, g/mol, \qquad M_{air} \approx 29 \, g/mol
\end{align*}
\begin{align*}
    \implies \frac{\delta \rho}{\rho} &= \frac{\lr{ 17 \delta \con{NO_x}}}{17 \con{NO_x} + 29}
    \qquad \text{at constant temperature}
\end{align*}
The above equation demonstrates that the relative change in density due to relative change in $NO_x$ concentration is insignificant, as the molarity of $NO_x$ itself is small.

% ====

\subsection{Parts-per-million to mol/m$^3$}
The commercial and FTIR sensors use ppm (parts per million) in terms of the mole-fraction for concentration
measurements.
%===
\begin{align}
    1 \, ppm^{\lr{mol}} &= \frac{1 \, mol \text{ of gas}}{10^6 \, mol \text{ of air }}
\end{align}
%===
Thus, this measurement when converted into $mol/m^3$, the temperature of the gas will be essential to get the right
value as the volume of 1 mole of air changes with temperature. Assuming ideal gas behavior,
%===
\begin{align}
    V_{air} &= n_{air} T_{air} \times \frac{V_0}{T_0}
    \label{eqn::V_air}
\end{align}
%===
where, $V_0, T_0$ are the volume and temperature of one mole of air under STP conditions. From literature,
$V_0 = 22.4 L = 22.4 \times 10^{-3} m^3$ and $T_0 = 273.15 K$.
%===
\begin{align*}
    T_{air} &= 273.15 + T\\
    n_{air} &= 10^6
\end{align*}
%===
Thus, volume of $10^6$ moles of air at temperature T,
%===
\begin{align}
    V_{air} &= 10^6 \times \frac{273.15 + T}{273.15} \times 22.4 \times 10^{-3} m^3
            = 22.4 \times 10^{3} \times \frac{273.15 + T}{273.15}
    \qquad \lrb{\because \ref{eqn::V_air}}
\end{align}
%===
Thus, we have the conversion between ppm and $mol/m^3$:
%===
\begin{align}
    x \text{ in } mol/m^3 &= \frac{x \text{ in } ppm }{\text{Volume of } 10^6 \text{ moles of air}}
                            = \frac{x \text{ in } ppm}{22.4 \times \lr{\frac{273.15 + T}{273.15}}}
                                \times 10^{-3} \, mol/m^3
\end{align}
%===
