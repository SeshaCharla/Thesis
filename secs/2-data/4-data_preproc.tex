\section{Data Preprocessing in Test-Cell and Truck Data}

Test-cell and truck data are preprocessed prior to system identification and validation. The preprocessing procedure consists of the following steps:
\begin{enumerate}
        \item \itbf{Unit Conversion}: Measurement signal units are converted to a consistent set of units selected for
                analysis.

        \item \itbf{NaN Filtering}: Samples containing one or more measurements identified as NaN (not a number) are
                excluded from the data set. This typically occurs in truck data when on-board sensors fail to record
                signals.

        \item \itbf{Clipping Data to Operating Ranges}: The data set is restricted to selected operating limits for
                relevant states and inputs, primarily temperature. Since polynomial fits are applied, the model's
                operating range is bounded. For hFTP and truck data, the temperature bounds are $200\,^{\circ}C$ to
                $300\,^{\circ}C$, while for RMC data, the range is $250\,^{\circ}C$ to $350\,^{\circ}C$.

        \item \itbf{Decimation to 1 Hz (Test Cell Data)}: As previously mentioned, to consistently capture the same
                frequency range of dynamics across test-cell and truck data sets, test-cell data are decimated from 5 Hz
                to 1 Hz to align with the truck data.

        \item \itbf{Smoothing}: Signals are smoothed using a non-causal 7th-order Chebyshev filter with a passband of 0
                to 0.1 Hz. For truck data, if data breaks occur shorter than 1 minute, linear interpolation is applied
                before smoothing.

        \item \itbf{Generating Drive Segments (Truck Data)}: Smoothed truck data are partitioned into drive segments,
                defined as contiguous data intervals without breaks. These segments differ from drive cycles, as they do
                not necessarily correspond to engine-on/off events and do not have a maximum length of 4 hours. Moreover,
                the drive segments are not repeatable unlike drive cycles.

        \item \itbf{Calculating Functions of Measured Signals}: After smoothing, quantities directly used in the model,
                which are linear or nonlinear functions of the measurement signals, are calculated. These quantities
                include:
        \begin{enumerate}
                \item $NO_x$ reduction in a time step:
                        \begin{align}
                                \eta(k) &= u_1(k-1) - x_1(k)
                        \end{align}
                \item Flow-normalized urea dosing:
                        \begin{align}
                                u_{2F}(k) &= \frac{u_{2}(k)}{F(k)}
                        \end{align}
                \item Flow-normalized inlet $NO_x$ concentration:
                        \begin{align}
                                u_{1F}(k) &= \frac{u_{1}(k)}{F(k)}
                        \end{align}
                \item Flow-scaled $NO_x$ reduction efficiency:
                        \begin{align}
                                \eta_F(k) &= \frac{\eta(k)}{u_{1F}(k-1)}
                        \end{align}
        \end{enumerate}
\end{enumerate}

The preprocessing steps outlined above ensure that the data sets are consistent, and suitable for subsequent system
identification and validation tasks.
