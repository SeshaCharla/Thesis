Test-cell data were collected for both aged and degreened catalysts under three established drive cycles:
\begin{enumerate}
        \item Hot Federal Test Procedure (hFTP)
        \item Cold Federal Test Procedure (cFTP)
        \item Ramped Mode Cycle (RMC)
\end{enumerate}
The main distinction among these three cycles is the operating temperature during the drive cycles (Figure~\ref{fig::test_cell_temp_profiles}). The RMC test operates at higher temperatures than the FTP tests. Although hFTP and cFTP have overlapping temperature ranges, cFTP starts at a much lower temperature. Table~\ref{tab::test_cell_data_vars} summarizes the variables included in the test-cell data.
%==
\begin{figure}[!ht]
        \centering
        \includegraphics[width=0.7\textwidth]{./figs/2-data/test_cell_temp_range.png}
        \caption{Drive Cycle Temperature Profiles of Test Cell Data}
        \label{fig::test_cell_temp_profiles}
\end{figure}
%==
\begin{table}[!ht]
\centering
\caption{Test Cell Data Variables}
\label{tab::test_cell_data_vars}
\begin{tabular}{l l l l}
   \hline \hline
   Data Name &
   Units &
   Variable &
   Description \\ \hline \hline
   %==========================================================================
   LOG\_TM &
   sec &
   $t$ &
   Time
   \\
   %==========================================================================
   EXHAUST\_FLOW &
   kg/min &
   $F$ &
   Exhaust Flow Rate
   \\
   %==========================================================================
   V\_AIM\_TRC\_DPF\_OOUT &
   $\lx{^o}{C}$ &
   $T_{in}$ &
   DPF-out (SCR-in) Gas Temperature
   \\
   % =========================================================================
   V\_AIM\_TRC\_SCR\_OUT &
   $\lx{^o}{C}$ &
   $T_{out}$ &
   SCR/ASC out Gas Temperature
   \\
   % ==========================================================================
   V\_UIM\_FLM\_ESTUREAINJRATE &
   ml/s &
   $u_{inj}$ &
   DEF (Urea Sol.) Dosing Rate
   \\
   % ==========================================================================
   ENG\_CW\_NOX\_FTIR\_COR\_U2 &
   ppm &
   $\con{NO_x}^{out}$ &
   Engine-Out $NO_x$
   \\
   % ==========================================================================
   EXH\_CW\_NOX\_COR\_U1 &
   ppm &
   $\con{NO_x}^{in}$ &
   Tailpipe $NO_x$
   \\
   % ==========================================================================
   EXH\_CW\_AMMONIA\_MEA &
   ppm &
   $\con{NH_3}^{out}$ &
   Tailpipe $NH_3$
   \\
   % ==========================================================================
   EONOX\_COMP\_VALUE &
   ppm &
   $\con{NO_x}^{in}$ &
   Engine-out $NO_x$
   \\
   % ==========================================================================
   V\_SCM\_PPM\_SCR\_OUT\_NOX &
   ppm &
   $\con{NO_x}^{out}_{\chi}$ &
   SCR-out $NO_x$ (cross-sensitive)
   \\
   \hline \hline
\end{tabular}
\end{table}
% =====
A key advantage of test-cell data is the inclusion of FTIR (Fourier Transform Infrared Spectroscopy) sensor measurements, which can estimate both NOx and NH3 gas concentrations. In contrast, the commercial NOx sensor is cross-sensitive to ammonia, whereas FTIR measurements do not exhibit this limitation. The sensor sampling frequency for the test-cell data is 5 Hz (ts = 0.2 s).
