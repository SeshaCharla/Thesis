\chapter{Data Characteristics}
The available test and truck data for testing the model structure needs to be put into common units. The following table
lists the units used for each of the physical properties.
\begin{table}[H]
   \centering
   \begin{tabular}{l l}
       \hline \hline
        \itbf{Property} & \itbf{Unit}\\
        \hline \hline
        Concentration   & $ 10^{-3} \, mol/m^{3}$ \\
        Temperature     & $ \times 10 + 200\lx{^o}{C}$ \\
        Flow Rate       & $\times 10 \, g/s$ \\
        Urea Injection  & $\times 10^{-1}\, ml/s$ \\
        Time            & $s$ \\
        \hline \hline
   \end{tabular}
   \caption{Table of Units used in the Model}
\end{table}

The available data and corresponding units are listed in the following table.
\begin{table}[H]
\centering
\begin{tabular}{l l l }
\hline \hline
Model Variable & Road Data Variable &Units\\
\hline \hline
$t$   & tod & $s$
\\
$T$   & pSCRBedTemp & $\lx{^o}{C}$
\\
$F$   & pExhMF & $g/s$
\\
$u_2$ & pUreaDosing & $ml/sec$
\\
$y_1 $ & pNOxOutppm & $ppm$
\\
$u_1$ & pNOxInppm & $ppm$
\\
 $y_1$ sensor status & V\_ATP\_ppm\_SCR\_Out\_NOx\_Status &
\\
 $u_1$ sensor status & V\_ATP\_ppm\_SCR\_In\_NOx\_Status or EONOx\_Sensor\_Status &
\\
\hline
\end{tabular}
\caption{Road data variables}
\end{table}


\begin{table}[H]
\centering
\begin{tabular}{l l l c}
   \hline \hline
   Variable Name              & Units     & Description & Variable from Model \\ \hline \hline
   LOG\_TM	                  & sec & Time
                                          & $t$\\
   EXHAUST\_FLOW	            & kg/min    & Exhaust Flow Rate
                                          & $F$\\
   V\_AIM\_TRC\_DPF\_OUT	   & $\lx{^o}{C}$ & DPF Out Gas Temperature
                                             & $T_{in}$\\
   V\_AIM\_TRC\_SCR\_OUT	   & $\lx{^o}{C}$ & SCR/ASC Out Gas Temperature
                                          & $T_{out}$\\
   V\_UIM\_FLM\_ESTUREAINJRATE& ml/s      & DEF (Urea Sol.) Dosing Rate
                                          & $u_2$\\
   ENG\_CW\_NOX\_FTIR\_COR\_U2& ppm       & Engine-Out NOx
                                          & $u_1$\\
   EXH\_CW\_NOX\_COR\_U1	   & ppm       & Tailpipe NOx
                                          & $x_1$\\
   EXH\_CW\_AMMONIA\_MEA	   & ppm       & Tailpipe NH3
                                          & $x_2$\\
EONOX\_COMP\_VALUE	         & ppm       & Engine-out NOx (corss-sensitive)
                                          & $u_1$\\
V\_SCM\_PPM\_SCR\_OUT\_NOX	   & ppm       & SCR-out NOx (cross-sensitive)
                                          & $y_1$\\
   \hline \hline
\end{tabular}
\caption{Test Cell Data Variables}
\end{table}


% ==============================================================================
\section{Density of the exhaust gas}

The density of the exhaust flow is assumed to be the density of air at that
temperature and ambient atmospheric pressure. Using the ideal gas law:
\begin{align}
    \rho &= \frac{PM}{R T} = \frac{\mu}{T}
\end{align}
\begin{align*}
    \text{where, } &\\
    P &= \text{Pressure of the exhaust gas (ambient pressure)} = 101.325 \: kPa\\
    M &= \text{Molecular weight of the exhaust gas} = 28.9652 \: g/mol\\
    T &= \text{Temperature of the exhaust gas in Kelvin}\\
    R &= \text{Universal gas constant} = 8.314 \: J/(mol.K)\\
\end{align*}

\itbf{Note}: $P$ and $M$ are replaced with $(P_1 M_1 + P_2 M_2)$ when humidity of the exhaust gas is considered.

\subsection{$\%$ Change in density for the temperature range of operation}
We have,
\begin{align*}
    \frac{\delta \rho}{\rho} &= -\frac{\delta T}{T}
\end{align*}
In general the operating temperature is very high ($250 \,^0 C \approx 500 K$), and most of the data lies in $\pm 100 \, ^0 C$ region. Thus, the maximum change in density is less than $10\%$, whose effect on flow rate is far smaller compared to the change in mass flow rate itself. Thus, density can be assumed to be a constant for this process.
\begin{align}
    \rho &= \rho_0
\end{align}

%%==================

\subsection{Sensitivity of density to change in $NO_x$ concentrations}
Let, $P_{NO_x}$ be the partial pressure of $NO_x$ and $P_0$ be the total pressure. Assuming the rest of the gas constitutions are similar to that of air. We have the density as:
\begin{align*}
    \rho &= \frac{1}{RT} \lr{ P_{NO_x} M_{NO_x} + (P_0 - P_{NO_x}) M_{air}} = \frac{1}{RT} \lr{ P_{NO_x} \lr{M_{NO_x} - M_{air}} + P_0 M_{air}}\\
    \implies \frac{\partial \rho}{\partial P_{NO_x}} &= \frac{1}{RT} \lr{\lr{M_{NO_x} - M_{air}}}\\
    \implies \frac{\delta \rho}{\rho} &= \frac{\lr{\lr{M_{NO_x} - M_{air}} \delta P_{NO_x}}}{\lr{ P_{NO_x} \lr{M_{NO_x} - M_{air}} + P_0 M_{air}}} \qquad \text{at constant temperature}
\end{align*}
Also,
\begin{align*}
    P_{NO_x} &= P_0 \times \frac{n_{NO_x}}{n_{NO_x} + n_{air}} = P_0 \times \frac{\con{NO_x}}{\frac{n_{NO_x} + n_{air}}{V}} = P_0 \times \frac{\con{NO_x}}{N_a (=1)} \qquad \lrb{\because 1 \, mole = n_{tot}/V}
\end{align*}
Thus,
\begin{align*}
    \frac{\delta \rho}{\rho} &= \frac{\lr{\lr{M_{NO_x} - M_{air}} \delta \con{NO_x}}}{\lr{ \con{NO_x}
 \lr{M_{NO_x} - M_{air}} + M_{air}}}
    \qquad \text{at constant temperature}
\end{align*}
We have,
\begin{align*}
    &M_{NO_x} \approx 46 \, g/mol, \qquad M_{air} \approx 29 \, g/mol
\end{align*}
\begin{align*}
    \implies \frac{\delta \rho}{\rho} &= \frac{\lr{ 17 \delta \con{NO_x}}}{17 \con{NO_x} + 29}
    \qquad \text{at constant temperature}
\end{align*}

The above equation demonstrates that the relative change in density due to relative change in $NO_x$ concentration is insignificant. As the molarity of $NO_x$ itself is small.

\section{Parts-per-million to mol/m$^3$}

The commercial and FTIR sensors use ppm (parts per million) in terms of the mole-fraction for concentration
measurements.
%===
\begin{align}
    1 \, ppm^{\lr{mol}} &= \frac{1 \, mol \text{ of gas}}{10^6 \, mol \text{ of air }}
\end{align}
%===
Thus, this measurement when converted into $mol/m^3$, the temperature of the gas will be essential to get the right
value as the volume of 1 mole of air changes with temperature. Assuming ideal gas behaviour,
%===
\begin{align}
    V_{air} &= n_{air} T_{air} \times \frac{V_0}{T_0}
    \label{eqn::V_air}
\end{align}
%===
where, $V_0, T_0$ are the volume and temperature of one mole of air under STP conditions. From literature,
$V_0 = 22.4 L = 22.4 \times 10^{-3} m^3$ and $T_0 = 273.15 K$.
%===
\begin{align*}
    T_{air} &= 273.15 + T\\
    n_{air} &= 10^6
\end{align*}
%===
Thus, volume of $10^6$ moles of air at temperature T,
%===
\begin{align}
    V_{air} &= 10^6 \times \frac{273.15 + T}{273.15} \times 22.4 \times 10^{-3} m^3
            = 22.4 \times 10^{3} \times \frac{273.15 + T}{273.15}
    \qquad \lrb{\because \ref{eqn::V_air}}
\end{align}
%===
Thus, we have the conversion between ppm and $mol/m^3$:
%===
\begin{align}
    x \text{ in } mol/m^3 &= \frac{x \text{ in } ppm }{\text{Volume of } 10^6 \text{ moles of air}}
                            = \frac{x \text{ in } ppm}{22.4 \times \lr{\frac{273.15 + T}{273.15}}}
                                \times 10^{-3} \, mol/m^3
\end{align}
%===

\section{Mass flow rate to volumetric flow rate}
The density of the exhaust gas is used to convert the mass flow rate of the exhaust gas to the volumetric flow rate of the exhaust gas.
\begin{align}
    f_v &= \frac{F}{\rho} = \frac{F}{\rho_0}  \label{eqn::fv_approx}
\end{align}
The effect of change of $f_v$ due to temperature change's effect on density is neglected as change in mass flow rate will be more significant.

\subsection{Approximate model for residence time}
We have the residence time of the exhaust gas in the SCR-ASC system within the operating limits of temperature and flow-rate:
\begin{align}
    \tau &= \frac{V}{f_v} = \frac{V \rho_0}{F} \label{eqn::res_time}
\end{align}
% Thus we have the linear approximation of the residence time:
% \begin{align*}
%     \tau &= \tau_0 + \delta \tau
%           = \tau_0 - \frac{V}{\bar{f_v} ^2} \delta f_v\\
%          &= \tau_0 - \frac{V}{\bar{f_v} ^2}  \frac{1}{\mu} \lr{F_0 \delta T + T_0 \delta F}
%           = \tau_0 - \tau_T \delta T - \tau_F \delta F
% \end{align*}
% dropping $\delta$ for notational convenience, we have the linear model for residence time:
% \begin{align}
%     \tau &= \tau_0 - \tau_T T - \tau_F F        \label{eqn::res_time}
% \end{align}
where,
\begin{align*}
    F_{min} &\leq F \leq F_{max}\\
    T_{min} &\leq T \leq T_{max}
\end{align*}

% ==============================================================================

\section{Test-Cell Data Preprocessing}
% \subsection{Test Data Signal Ranges}
The ranges of the measurement signals are tabulated bellow:
\begin{table}[H]
\centering
\begin{tabular}{c c c c c}
\hline \hline
Variable & Units & & & \\
\hline \hline
Degreend Data & & Cold FTP & Hot FTP & RMC\\ \hline
$T$   & $+200 \, ^0C$ & [60.96, -174.0] & [67.38, -64.7] & [148.76, 43.76]
\\
$F$   & $g/s$ & [439.04, 0.0]& [431.14, 0.0]& [403.84, 32.65]
\\
$x_1$ & $mol/m^3$ & [17.24, -2.97] &  [3.55, -0.57]& [2.23, 0.14]
\\
$x_2$ & $mol/m^3$ & [0.06, 0.0] & [0.17, -0.01] & [0.27, 0.06]
\\
$u_1$ & $mol/m^3$ & [20.16, 0.0] & [16.46, 0.0] & [24.26, 0.0]
\\
$u_2$ & $mol/m^3$ & [1.4, 0.0] & [1.0, 0.0] & [0.71, 0.01]
\\
$y_1$ & $mol/m^3$ & [0.07, 0.0]& [0.82, 0.0]& [2.11, 0.0]
\\
\hline
Aged Data & &Cold FTP & Hot FTP & RMC\\ \hline
$T$   & $+200 \, ^0C$ & [67.03, -172.85]& [72.5, -56.35]& [154.38, 48.18]
\\
$F$   & $g/s$ & [416.85, 0.0]& [418.24, 0.0]& [409.59, 32.63]
\\
$x_1$ & $mol/m^3$ & [25.03, 0.01] & [4.9, -0.42] & [3.24, 0.08]
\\
$x_2$ & $mol/m^3$ & [0.04, 0.0] & [0.6, -0.02] & [0.23, 0.08]
\\
$u_1$ & $mol/m^3$ &  [18.58, 0.0]& [16.26, 0.0] & [22.89, 0.0]
\\
$u_2$ & $mol/m^3$ & [1.31, 0.0] & [1.0, 0.0] & [0.67, 0.01]
\\
$y_1$ & $mol/m^3$ & [0.09, 0.0]& [1.06, 0.0]& [2.86, 0.0]
\\
\hline \hline
\end{tabular}
\caption{Test cell data ranges}
\end{table}


%==============================================================================

% \subsection{Signal Processing Steps}
The signal processing steps for getting the viable data used for model validation are summarized bellow.
\begin{figure}[H]
        \centering
        \includegraphics[width = 0.9\textwidth]{./figs/2-data/DataProcessingPipeline.png}
        \caption{Summary of Signal Processing Steps for Test Cell Data}
\end{figure}


% ==============================================================================

\section{Truck Data Preprocessing}
% \subsection{Truck Data Signal Ranges}
The ranges of the measurement signals are tabulated below:
\begin{table}[H]
\centering
\begin{tabular}{c c c c c c}
\hline \hline
Variable & Units & & & & \\
\hline \hline
Degreened Data & & adt\_15 & mes\_15 & wer\_15 & trw\_15 \\ \hline
$T$   & $+200 \, ^0C$ & [135.71, -79.68]& [116.91, -93.05]& [98.7, -142.85]& [125.21, -169.55]
\\
$F$   & $g/s$ & [425.94, 15.82]& [529.08, 0.0]& [386.51, 0.0]& [559.86, 0.0]
\\
$u_1$ & $mol/m^3$ & [27.34, 0.0]& [103.47, -6.87]& [49.99, 0.0]& [100.22, -6.73]
\\
$u_2$ & $mol/m^3$ & [0.89, 0.0]& [1.5, 0.0]& [1.07, 0.0]& [2.18, 0.0]
\\
$y_1$ & $mol/m^3$ & [24.66, 0.0] & [20.86, 0.0] & [22.51, 0.0]& [30.73, 0.0]
\\
\hline
Aged Data & & adt\_17 & mes\_18 & wer\_20 & trw\_16 \\ \hline
$T$   & $+200 \, ^0C$ & [111.04, -150.35]& [173.4, -86.67]& [353.84, -127.68]& [152.48, -155.45]
\\
$F$   & $g/s$ & [440.63, 0.0]& [391.06, 0.0]& [425.77, 0.0]& [569.7, 0.0]
\\
$u_1$ & $mol/m^3$ & [41.76, 0.0]& [36.89, 0.0]& [35.16, 0.0]& [68.93, 0.0]
\\
$u_2$ & $mol/m^3$ & [1.88, 0.0]& [1.07, 0.0]& [1.28, 0.0]& [1.66, 0.0]
\\
$y_1$ & $mol/m^3$ & [9.14, 0.0]& [19.6, 0.0]& [49.99, 0.0]& [38.39, 0.0]
\\
\hline \hline
\end{tabular}
\caption{Truck data ranges}
\end{table}


% =============================================================================

% \subsection{Signal Processing Steps}
The road data was collected from trucks operating in the United States  for two days separated by a few years
(Table~\ref{tab::truck_data_summary}) during which the catalyst has aged based on the existing performance measures. The
data contains measurements of $NO_x$ sensors before and after the catalyst, as well as other required variables
including flow rate, temperature and urea injection rate. The data preporcessing includes,
\begin{enumerate}
        \item removing the rows with missing values,
        \item removing the rows whose operating temperature range is beyond range of $200-300 \,^0C$,
        \item interpolating the missing data if the data breaks are smaller than a minute, and finally,
        \item smooting the data using non-causal chebyshev filter.
\end{enumerate}
The preprocessed data is than partitioned into drive segments with no breaks which correspond to a continuous driving
period from engine start to engine stop.
%===
\begin{table}[H]
        \centering
        \caption{Truck Data Set Years and Mileage Difference}
        \label{tab::truck_data_summary}
        \begin{tabular}{l l l r}
        \hline \hline
               & Earlier Data & Latter Data & Milage \\
         Truck &  Year            & Year    & Difference\\ \hline \hline
        AD Transport (adt) & $2015$ & $2017$ & $2.79 \times 10^5$\\
        Mesilla Valley (mes) & $2015$ & $2018$ & $11.87 \times 10^5$\\
        Werner (wer) & $2015$ & $2017$ & $5.89 \times 10^5$\\
        Transwest (trw) & $2015$ & $2016$ & $6.56 \times 10^5$ \\
        \hline \hline
        \end{tabular}
\end{table}
