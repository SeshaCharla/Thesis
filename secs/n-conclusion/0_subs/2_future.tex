\section{Future Work}

Several directions emerge naturally from the present work. A first is to further validate and refine the assumed
distribution for the error between the saturated catalyst response and the actual system response. In particular,
fitting a more general Weibull distribution to the error data, using an expectation maximization (EM) algorithm for
parameter estimation, could provide a more accurate statistical description and tighter confidence bounds for
saturation-based decisions.

In parallel, the impact of the proposed nonlinear dynamic model on urea-dosing control should be investigated. By
explicitly exploiting the model structure, future studies can assess whether more informed control policies improve
de-$NO_x$ efficiency, for example by deliberately maximizing the duration of safely maintained catalyst saturation while
respecting ammonia slip and hardware constraints.

From a modelling standpoint, the Self-Excited Threshold Autoregressive (SETAR) framework offers a promising avenue for
capturing regime-switching behavior in SCR-ASC dynamics. Finally, alternative formulations for the temperature
dependence of reaction and transport parameters should be explored, with the goal of achieving a wider temperature
validity range and better-posed (i.e., uniquely identifiable) parameter estimates across operating conditions.
