\section{Summary of Contributions}
The main contributions of this thesis are summarized below.
\begin{enumerate}
    \item \textbf{Discrete switched nonlinear model for SCR-ASC dynamics:}
    A new discrete-time, switched nonlinear model of $NO_x$ reduction in the SCR--ASC system was derived directly from
                molar conservation over a control volume, rather than from spatially discretized CSTR chains. The model:
    \begin{enumerate}
          \item explicitly incorporates the interplay between residence time and sensor sampling time, ensuring that the
                  dynamics captured by the model are consistent with the temporal resolution of available measurements;
          \item employs linear-in-parameter approximations for physical properties such as residence time and
                  temperature-dependent rate constants, while using polynomial (Chebyshev) bases to retain a structure
                    that is linear in parameters and numerically well conditioned; and
          \item models catalyst saturation and desaturation by introducing a switched structure, with causal switching
                  condition, that distinguishes saturated and unsaturated regimes and constrains $NO_x$ reduction
                    dynamics accordingly.
    \end{enumerate}

    \item \textbf{Identifiable parametrization and convex parameter estimation:}
    The model was reformulated into an identifiable parametric structure that can be estimated from tailpipe $NO_x$
                measurements, urea dosing, flow rate, and temperature. A key step was the algebraic elimination of the
                unmeasured surface ammonia state by introducing the recursive quantity $\eta(k) = u_1(k-1) - x_1(k)$,
                which represents the change in $NO_x$ concentration due to reduction over one sampling interval. This
                led to:
    \begin{enumerate}
          \item a recursive dynamic model that is linear in parameters, reducing the estimation problem to convex
                  optimization;
          \item a linear programming formulation for saturated-catalyst parameters that minimizes the area under the
                  $NO_x$ reduction per time step under saturation'' curve, subject to lower bounds imposed by the actual
                    observed $NO_x$ reduction;
          \item a data-driven method to identify saturated and unsaturated segments, enabling separate estimation of
                  parameters in each regime; and
          \item a switched nonlinear model whose parameters are uniquely identifiable from available data.
    \end{enumerate}

    \item \textbf{Model validation using test cell data:}
    The switched nonlinear model was validated using experimental test-cell data (RMC, Hot-FTP, and Cold-FTP) for both
                degreened and aged catalysts. Across all cases, the model achieved goodness-of-fit values exceeding
                $60\%$, and consistently outperformed the linearized CSTR model from the literature, whose fits were
                typically below $30\%$ on the same data. The results demonstrate that explicitly enforcing conservation
                over the sampling interval and modeling saturation yields a more accurate and physically interpretable
                description of SCR--ASC dynamics, while retaining a parsimonious parameter set.

    \item \textbf{Statistical hypothesis-testing framework for aging detection:}
    The saturated-model parameters were embedded in a composite hypothesis test for catalyst health, where the null
                hypothesis corresponds to a degreened catalyst and the alternative corresponds to an aged catalyst.
                Using maximum likelihood estimation and asymptotic properties of the Fisher information matrix, a
                Wald-test statistic was derived that compares the estimated saturated parameters to their degreened
                reference values. The framework:
    \begin{enumerate}
          \item explicitly accounts for $NO_x$ sensor ammonia cross-sensitivity by treating it as a non-negative
                  directional error, which preserves the bounding relationships used in the saturated-model estimation;
          \item provides analytical expressions for the distributions of the test statistic under both hypotheses,
                  enabling threshold selection via the Neyman--Pearson criterion; and
          \item offers performance measures in terms of false-alarm, detection, and missed-detection probabilities.
    \end{enumerate}

    \item \textbf{Experimental validation on test-cell and truck data:}
    The complete detection framework was validated using test-cell experiments (RMC and Hot-FTP cycles) and real-world
    data from four long-haul trucks, each evaluated in both degreened and aged states. Despite sensor cross-sensitivity
    and variability in operating conditions, the detector reliably distinguished aged from degreened catalysts whenever
    the data contained a sufficient number of samples near catalyst saturation. This establishes a practically
    implementable, non-intrusive, and statistically principled approach to SCR catalyst aging diagnostics.
\end{enumerate}
