\begin{abstract}%
        A primary function of diesel engine after-treatment systems is to eliminate engine-out (EO) NOx through
        catalytic reduction. NOx reduction occurs in the Selective Catalytic Reduction (SCR) system, which utilizes
        ammonia generated from urea dosing. Excess ammonia is catalytically oxidized in the downstream Ammonia Slip
        Catalysis (ASC) section of the SCR-ASC system. Catalyst degradation, or aging, diminishes NOx reduction
        effectiveness and may result in exhaust gas concentrations that exceed regulatory limits. Therefore, an on-board
        diagnostics (OBD) tool capable of monitoring catalyst aging using only available sensor data would enhance
        reliability in regulatory compliance.

        The present work develops a robust, model-based, and non-intrusive on-board diagnostics algorithm that can
        detect catalyst aging under the real-world limitations of observability, sampling, and NOx sensor
        cross-sensitivity. The algorithm is based on a discrete nonlinear model that switches between catalyst
        saturation and unsaturation, developed from first principles. This model captures the reacting flow dynamics
        observed in the sensor data by explicitly accounting for the correlation between the residence time and sampling
        frequency. A novel parameter estimation algorithm is introduced that first detects regions of the data where the
        catalyst is close to saturation, and then uses the classified data to estimate saturated and unsaturated model
        parameters. The model's validity is demonstrated using the test-cell (experimental) data provided by Cummins.

        Based on the physical interpretation of model parameters and the hypothesis that catalyst aging reduces the
        concentration of adsorption sites, a parameter estimation-based hypothesis-testing framework for aging detection
        is proposed. The detector based on saturated model parameters demonstrated greater effectiveness and robustness
        to NOx sensor cross-sensitivity. The detector's performance is validated using both test-cell data and data from
        four long-haul trucks. Truck data, spanning a day, was collected at two different times: first, when the
        catalyst was de-greened (new), and second, when it was confirmed to be aged in accordance with existing metrics.
\end{abstract}
