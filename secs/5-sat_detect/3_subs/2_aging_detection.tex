\section{Catalyst Aging Detection in Test Cell Data}
The parameters of the saturated system are a function of the rate-constant for SCR reaction and the maximum surface
concentration of viable voids on the catalyst. The adsorption site concentration is hypothesized to change with aging of
the catalyst allowing the parameters to capture the aging of the catalyst. This is demonstrated using the test-cell
experiments on a new (degreened) and hydro-thermally aged catalyst with the same capacity. The predicted $NO_x$
reduction for an aged catalyst is found to be statistically lower than the predicted $NO_x$ reduction in a degreened
catalyst (Figure-\ref{fig::aging_diff}).
\begin{figure}[ht]
        \centering
        \includegraphics[width=\figWidth]{./figs/5-sat_detect/3_parm_ID/AgingDiff.png}
        \caption{Predicted $\eta_{sat}$ for Aged and Degreened Catalyst}
        \label{fig::aging_diff}
\end{figure}
%===
When the $\eta_{sat}$ response is normalized with respect to the inlet $NO_x$ concentration and flow-rate, we end up
with a quantity that is only a function of temperature and represents the product of rate-constant and maximum surface
coverage. From (\ref{eqn::regression}),
\begin{align}
        \alpha_{sat} &= \frac{F(k)}{u_1(k)} \eta_{sat}(k) = \tau_0 k_{scr}\Gamma = \phi_T^T(k)\theta_{sat}\\
        \text{where,} \quad \phi_T^T(k) &= \bm{2\lr{\frac{T-T_0}{T_r}}^2-1 & \frac{T-T_0}{T_r} & 1}
\end{align}
%===
This quantity, $\alpha_{sat}$ can be estimated and changes with aging of the catalyst based on the hypothesis that the
aging changes the adsorption site concentration. The change in $\alpha_{sat}$ with aging is demonstrated using the
test-cell experiments (Figures~\ref{fig::alpha_sat_ssd} and \ref{fig::alpha_sat_iod}).
\begin{figure}[!ht]
        \begin{minipage}{0.49\textwidth}
                \begin{figure}[H]
                        \centering
                        \includegraphics[width=\textwidth]{./figs/5-sat_detect/3_parm_ID/aging_factor_ssd.png}
                        \caption{$\hat \alpha_{sat}$ from RMC FTIR Data}
                        \label{fig::alpha_sat_ssd}
                \end{figure}
        \end{minipage}
        \begin{minipage}{0.49\textwidth}
                \begin{figure}[H]
                        \centering
                        \includegraphics[width=\textwidth]{./figs/5-sat_detect/3_parm_ID/aging_factor_iod.png}
                        \caption{$\hat \alpha_{sat}$ from RMC $NO_x$ Sensor Data}
                        \label{fig::alpha_sat_iod}
                \end{figure}
        \end{minipage}
\end{figure}
%===
\par Thus, $\theta_{sat}$ captures the aging of the catalyst. Hence, the catalyst aging detection problem can be posed
as a hypothesis testing problem with null hypothesis $\lr{\mathcal{H}_0}$ corresponding to a degreened catalyst and
alternative hypothesis $\lr{\mathcal{H}_1}$ corresponding to an aged catalyst. Since the distribution of the error is
assumed to be known (half-normal) for both hypotheses and differs by the unknown parameter $\theta_{sat}$ which can be
estimated, the following Wald test \cite{wald1943tests} is proposed to check if the parameter estimates for the given
data set correspond to an aged catalyst or a known degreened catalyst.
Decide $\mathcal{H}_1$ if,
\begin{align}
        T_w  &= \lr{\hat \theta_{sat} - \theta_{sat}^{dg}}^T I\lr{\hat \theta_{sat}} \lr{\hat \theta_{sat} - \theta_{sat}^{dg}} > \beta
\end{align}
where $\theta_{sat}^{dg}$ is the parameter estimate for the degreened catalyst under the same input conditions,
including temperature, flow-rate and urea-dosing rate ranges. The threshold $\beta$ is decided based on the experimental
results. For RMC and FTP data sets, the value of the test-statistic is tabulated in Table~\ref{tab::Tw_tst}.
% Table of test-statistic
\begin{table}[ht]
        \centering
        \caption{Wald Test Results On the Test Cell Data}
        \label{tab::Tw_tst}
        \begin{tabular}{l c c}
        \hline \hline
        Test & $T_w$ from FTIR & $T_w$ from $NO_x$ sensor \\\hline \hline
        DG RMC 1      & 0.26 & 5.54\\
        DG RMC 2      & 0.00 & 0.00\\
        DG RMC 3      & 6.71 & 5.46\\
        Aged RMC      & 47.16 & 17.68\\
        \hline
        DG Hot FTP 1  & 0.49 & 0.81\\
        DG Hot FTP 2  & 3.26 & 1.98\\
        DG Hot FTP 3  & 0.00 & 0.00\\
        Aged Hot FTP  & 4.84 & 11.44\\
        \hline \hline
        \end{tabular}
\end{table}
The distribution of the test-statistic under aging (non-central $\chi^2$, asymptotically) is different for RMC and FTP
test conditions due to the differences in the ranges of inputs (temperature, flow-rate and urea-dosing rate) experienced
by the catalyst during the tests. Based on the results, a reasonably discerning threshold that maximizes the probability of detection for Hot-FTP and RMC test conditions is found to be:
\begin{align}
        \beta_{hFTP} &= 4.00, \qquad \beta_{RMC} = 10.00
\end{align}
Further investigation with more data is needed to arrive at the probability of detection and false alarm for these
thresholds. The proposed Wald test detects the aged catalyst for both FTIR and $NO_x$ sensor measurements under both
test conditions.
