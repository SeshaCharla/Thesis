\subsection{Catalyst Mode Detection}
The saturated catalyst model's response is not dependent on the previous state but only on the inputs, making the
response independent of the initial conditions. Thus, using the parameter estimates, a given data point can be
classified as saturated catalyst response if the prediction error for the same inputs (inlet $NO_x$, $u_1$, and
mass flow rate, $F$) is less than the $\epsilon_{sat}$.
\begin{align}
        \abs{\hat \eta_{sat}(k) - \eta(k)} \leq \epsilon_{sat} \implies \text{Catalyst Saturation} \label{eqn::sat_cond}
\end{align}
For the current test-cell and truck dataset, $\epsilon_{sat} = 2.5\times 10^{-3} \, mol/m^{-3}$. This value is chosen
based on the variance of distribution of the prediction error for the saturated catalyst model.
\begin{figure}[!ht]
        \begin{minipage}{0.49\textwidth}
                \begin{figure}[H]
                        \centering
                        \includegraphics[width=\textwidth]{./figs/5-sat_detect/3_parm_ID/SatDetect_aged_rmc_FTIR.png}
                \end{figure}
        \end{minipage}
        \begin{minipage}{0.49\textwidth}
                \begin{figure}[H]
                        \centering
                        \includegraphics[width=\textwidth]{./figs/5-sat_detect/3_parm_ID/SatDetect_dg_rmc_1_FTIR.png}
                \end{figure}
        \end{minipage}
        \begin{minipage}{0.49\textwidth}
                \begin{figure}[H]
                        \centering
                        \includegraphics[width=\textwidth]{./figs/5-sat_detect/3_parm_ID/SatDetect_dg_rmc_2_FTIR.png}
                \end{figure}
        \end{minipage}
        \begin{minipage}{0.49\textwidth}
                \begin{figure}[H]
                        \centering
                        \includegraphics[width=\textwidth]{./figs/5-sat_detect/3_parm_ID/SatDetect_dg_rmc_3_FTIR.png}
                \end{figure}
        \end{minipage}
        \caption{Catalyst mode detection results for test-cell data}
        \label{fig::cat_mode_det}
\end{figure}
The catalyst mode detection for DG RMC 1 test-cell data using FTIR measurements is shown in
Figure~\ref{fig::cat_mode_det}. The catalyst is found to be close to saturation (or at its maximum $NO_x$ reduction)
when the mass flow rate and the urea dosing rates are high, consistent with the physical understanding of the catalyst
operation. Further, the flow rate can introduce a virtual ceiling on the maximum achievable $NO_x$ reduction due to
reduced residence time in the catalyst, which would be indistinguishable from saturated catalyst behavior as the $NO_x$
reduction would be independent of urea dosing.

A fraction of the data from the test-cell experiments under RMC and Hot-FTP test conditions for both degreened and aged
catalysts are found to be in saturated mode. The number of data points close to catalyst saturation $\lr{N_{sat}}$ and
the total number of samples ($N$) for different test conditions is tabulated in Tables~\ref{tab::sat_data_perc_ssd} and
\ref{tab::sat_data_perc_iod}.
%==============================================================
\begin{table}[ht]
        \centering
        \caption{Data Points Close to Catalyst Saturation in Test-Cell FTIR Data}
        \label{tab::sat_data_perc_ssd}
        \begin{tabular}{l r r}
                \hline \hline
                Test & $N$ & $N_{sat}$ \\\hline \hline
                DG RMC 1  & 2402 & 490  \\
                DG RMC 2  & 2402 & 493  \\
                DG RMC 3  & 2402 & 491  \\
                Aged RMC  & 2402 & 515  \\ \hline
                DG Hot FTP 1 & 634 & 186  \\
                DG Hot FTP 2 & 634 & 181  \\
                DG Hot FTP 3 & 638 & 180  \\
                Aged Hot FTP & 636 & 198  \\
                \hline\hline
        \end{tabular}
\end{table}
\begin{table}[ht]
        \centering
        \caption{Data Points Close to Catalyst Saturation in Test-Cell $NO_x$ Sensor Data}
        \label{tab::sat_data_perc_iod}
        \begin{tabular}{l r r}
                \hline \hline
                Test & $N$ & $N_{sat}$ \\\hline \hline
                DG RMC 1  & 2402 & 492  \\
                DG RMC 2  & 2402 & 494  \\
                DG RMC 3  & 2402 & 493  \\
                Aged RMC  & 2402 & 515  \\ \hline
                DG Hot FTP 1 & 634 & 194  \\
                DG Hot FTP 2 & 634 & 191  \\
                DG Hot FTP 3 & 638 & 198  \\
                Aged Hot FTP & 636 & 220  \\
                \hline\hline
        \end{tabular}
\end{table}
Note that the higher number of data points classified as close to catalyst saturation in RMC test conditions, compared
to Hot-FTP, results in better parameter estimates and performance of the aging detection test statistic in this case
(Table~\ref{tab::Tw_tst}).
