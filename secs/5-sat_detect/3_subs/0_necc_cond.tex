\subsection{Limitation: Parameter Estimation Assumes Catalyst Saturation \label{sec::necessary_cond}}
The catalyst mode detection algorithm presented fails when the catalyst is not saturated within the duration for a
significant length of time in the data. If the data-set does not have any regions of saturation, the approach fits a
saturated model as if the maximum achieved catalyst ammonia storage, $(\max\lr{\sigma(k)})$, is the storage capacity of
the catalyst $(\Gamma)$. This is demonstrated using data from high-fidelity simulation model (AVL Cruise
\cite{AVLCruise}) tuned to RMC (Ramped Mode Cycle) test on a test-cell. The RMC test is then simulated with $\pm 20\%$
change in the gain of urea-dosing controller to change the length of the data where the catalyst remains saturated.
\begin{figure}[ht]
        \centering
        \includegraphics[width=\figWidth]{./figs/5-sat_detect/3_parm_ID/eta_sim.png}
        \caption{RMC $\eta$ simulation with $\pm 20\%$ Urea Dosing}
        \label{fig::necessary_cond_sim}
\end{figure}
\begin{figure}[ht]
        \centering
        \includegraphics[width=\figWidth]{./figs/5-sat_detect/3_parm_ID/eta_sat.png}
        \caption{RMC $\eta_{sat}$ prediction with $\pm 20\%$ Urea Dosing}
        \label{fig::necessary_cond_sat}
\end{figure}
Figure~\ref{fig::necessary_cond_sim} shows no increase in $NO_x$ reduction when urea dosing is increased by $20\%$,
which indicates that maximum $NO_x$ reduction is already reached at nominal urea dosing. This could be because the
catalyst is saturated or because the inlet $NO_x$ concentration is sufficiently low. Conversely, a $20\%$ decrease in
urea dosing decreases $NO_x$ reduction. When the $-20\%$ urea-dosing data are used to estimate saturated-model
parameters, the model predicts a substantially lower $NO_x$ reduction under saturation than for the nominal and $+20\%$
dosing cases (Figure~\ref{fig::necessary_cond_sat}), the latter two giving essentially identical results. Thus for the nominal and $+20\%$ dosing cases, the catalyst is indeed saturated, and the parameter estimates show the same. However, for the $-20\%$ dosing case, the catalyst is not saturated in the data, instead the parameters fit the maximum surface coverage achieved $\lrf{\theta_{max} = \sfrac{\gamma_{max}}{\Gamma}}$ in the data, i.e.,
\begin{align}
        \eta_{sat}(k+1) &= \frac{u_1(k)}{F(k)} \tau_0 \; k_{scr} \Gamma \theta_{max} = \phi_{sat} \theta_{sat}
\end{align}

It is assumed that under ordinary drive conditions, the catalyst will reach saturation for a significant length of time, making the proposed parameter estimation approach applicable.
