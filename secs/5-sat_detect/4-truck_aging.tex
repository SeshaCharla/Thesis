\section{Aging Detection from Truck Data}
%===
The performance of the proposed aging detector is also demonstrated using the $NO_x$ sensor data from four long-haul
trucks. The road data were collected from trucks operating in the United States for two days separated by a few years
(Table~\ref{tab::truck_data_mileage}) during which the catalyst has aged based on the existing performance measures. The
data contains measurements of $NO_x$ sensors before and after the catalyst, as well as other required variables
including flow rate, temperature and urea injection rate. The preprocessed data are then partitioned into drive segments
with no breaks which correspond to a continuous driving period from engine start to engine stop. Unlike the test-cell data where the engine drive-cycles are repeatable and produce a significant fraction of samples
where the catalyst is close to saturation, the truck data drive segments constitute varied driving conditions with fewer
samples $\lrf{\leq 10\%}$ that are close to catalyst saturation. A necessary condition for the proposed parameter
estimation and aging detection framework is that the data contains a sufficiently large number of samples close to
catalyst saturation such that the asymptotic properties of the MLE and the test statistic hold (\ref{eqn::Ns_min}).
Hence, the drive segment data are further pruned to only include segments with at least $150$ samples ($120$ in the case
of Truck C) close to catalyst saturation.
\begin{align}
        N_{sat} &\geq N_{sat}^{min} \label{eqn::Ns_min} \\
        \beta_{truck} &= 250 \label{eqn::beta_truck}
\end{align}
The threshold of the Wald test statistic (\ref{eqn::beta_truck}) for aging detection in trucks is chosen such that it
maximizes the probability of detection while minimizing the probability of false alarm and missed detection for the
given limited truck data. A larger data set would help arrive at the thresholds using Neyman-Pearson criteria
\cite{kay1998detection}. The value of the test statistic for the drive segments for each truck is tabulated in
Tables~\ref{tab::adt_aging_res}, \ref{tab::mes_aging_res}, \ref{tab::wer_aging_res} and \ref{tab::trw_aging_res}. A
summary of the aging detection results is tabulated in Table~\ref{tab::truck_aging_summary}.
%===
\begin{table}[!ht]
        \centering
        \caption{Summary of Aging Detection Results on Truck Data}
        \label{tab::truck_aging_summary}
        \begin{tabular}{c c c c c}
        \hline \hline
         Truck & Total Drive & Detection & False  & Missed \\
               & Segments    &         & Alarm & Detection  \\\hline \hline
        Truck A   & 8  & 4 & 0 & 4 \\
        Truck B   & 11 & 9 & 1 & 1 \\
        Truck C   & 9  & 7 & 1 & 1 \\
        Truck D   & 8  & 6 & 2 & 0 \\
        \hline \hline
        \end{tabular}
\end{table}
The detector performs reasonably well on three of the four trucks with high probability of detection and low false alarm
and missed detection rates. Truck A has relatively lower mileage difference between the earlier and latter data years
which may explain the lower detection performance. The performance is limited by the amount of data available with
sufficient samples close to catalyst saturation. Further, the lack of ground truth on the actual individual catalyst's
aging levels limits the ability to quantify the performance of the detector. Nevertheless, the results demonstrate the
potential of the proposed aging detection framework for on-board application using $NO_x$ sensor data.

% =================================================================

\subsection{Aging Detector Results on Truck Data Drive Segments}
The value of the Wald test statistic for each drive segment that met the necessary condition of sufficient samples close
to catalyst saturation is calculated and tabulated in Tables~\ref{tab::adt_aging_res}, \ref{tab::mes_aging_res}, \ref{tab::wer_aging_res} and \ref{tab::trw_aging_res}, for each truck. The number and percentage of samples close to
catalyst saturation in each drive segment is also tabulated. There is a strong correlation between detection of aging
and the number of samples close to catalyst saturation in the drive segment.
%===
% Truck Maping
% Truck A - AD Transport (adt)
% Truck B - Mesilla Valley (mes)
% Truck C - Werner (wer)
% Truck D - Transwest (trw)
% =====
\begin{table}[!ht]
        \centering
        \caption{Aging Detection Results for Truck A}
        \label{tab::adt_aging_res}
        \begin{tabular}{l r r r}
        \hline \hline
                Drive Segment& $T_w$ & $N_{sat}$ & $\% \lr{ \sfrac{N_{sat}}{N} }$ \\ \hline \hline
                A$\_15\_0$ & 0.00 & 797 & 6.70\\
                A$\_15\_1$ & 28.95 & 305 & 2.75\\
                A$\_15\_2$ & 58.50 & 276 & 5.93\\
                A$\_15\_3$ & 89.56 & 271 & 12.77\\ \hline
                A$\_17\_0$ & 65.59 & 414 & 2.97\\
                A$\_17\_1$ & 127.79 & 614 & 6.76\\
                A$\_17\_2$ & 12.13 & 330 & 8.44\\
                A$\_17\_3$ & 61.76 & 336 & 9.73\\
                \hline \hline
        \end{tabular}
\end{table}
\begin{table}[!ht]
        \centering
        \caption{Aging Detection Results for Truck B}
        \label{tab::mes_aging_res}
        \begin{tabular}{l r r r}
        \hline \hline
                Drive Segment& $T_w$ & $N_{sat}$ & $\% \lr{ \sfrac{N_{sat}}{N} }$ \\ \hline \hline
                B$\_15\_0$ & 672.43 & 167 & 1.26\\
                B$\_15\_1$ & 106.66 & 189 & 1.81\\
                B$\_15\_3$ & 237.79 & 159 & 2.95\\
                B$\_15\_4$ & 0.00 & 226 & 6.03\\
                B$\_15\_6$ & 84.73 & 179 & 5.75\\ \hline
                B$\_18\_0$ & 543.55 & 231 & 2.10\\
                B$\_18\_1$ & 1050.24 & 308 & 3.00\\
                B$\_18\_4$ & 22.48 & 195 & 6.14\\
                B$\_18\_6$ & 7150.19 & 378 & 12.73\\
                B$\_18\_7$ & 760.89 & 224 & 8.14\\
                B$\_18\_8$ & 342.67 & 246 & 9.04\\
        \hline \hline
        \end{tabular}
\end{table}
\begin{table}[!ht]
        \centering
        \caption{Aging Detection Results for Truck C}
        \label{tab::wer_aging_res}
        \begin{tabular}{l r r r}
        \hline \hline
                Drive Segment& $T_w$ & $N_{sat}$ & $\% \lr{ \sfrac{N_{sat}}{N} }$ \\ \hline \hline
                C$\_15\_1$ & 0.000 & 186 & 1.30\\
                C$\_15\_4$ & 113.86 & 143 & 3.84\\
                C$\_15\_5$ & 4340.08 & 124 & 4.16\\ \hline
                C$\_17\_0$ & 700.88 & 186 & 1.24\\
                C$\_17\_2$ & 3993.85 & 446 & 7.23\\
                C$\_17\_3$ & 1693.49 & 199 & 4.00\\
                C$\_17\_4$ & 186.64 & 120 & 2.81\\
                C$\_17\_7$ & 1878.52 & 129 & 4.73\\
                C$\_17\_9$ & 7500.10 & 218 & 11.63\\
        \hline \hline
        \end{tabular}
\end{table}
\begin{table}[!ht]
        \centering
        \caption{Aging Detection Results for Truck D}
        \label{tab::trw_aging_res}
        \begin{tabular}{l r r r}
        \hline \hline
                Drive Segment& $T_w$ & $N_{sat}$ & $\% \lr{ \sfrac{N_{sat}}{N} }$ \\ \hline \hline
                D$\_15\_0$ & 0.00 & 214 & 1.29 \\
                D$\_15\_2$ & 98.99 & 160 & 2.95 \\
                D$\_15\_4$ & 1546.55 & 213 & 4.90 \\
                D$\_15\_6$ & 935.05 & 152 & 8.86 \\ \hline
                D$\_16\_1$ & 423.20 & 246 & 3.67 \\
                D$\_16\_2$ & 1022.06 & 199 & 5.22 \\
                D$\_16\_3$ & 697.78 & 170 & 5.36 \\
                D$\_16\_5$ & 313.62 & 200 & 9.31 \\
        \hline \hline
        \end{tabular}
\end{table}
