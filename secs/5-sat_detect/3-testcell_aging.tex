\section{Parameter Estimation and Aging Detection in Test Cell Data}
Using the parameter estimation approach outlined in the previous section, we analyze the test cell data to estimate the
parameters of the saturated catalyst model. The parameter estimates using FTIR measurements from the test cell
experiments are tabulated in Table-\ref{tab::parm_FTIR}. The parameters show consistency across repeated RMC (Ramped
Mode Cycle) and Hot-FTP (Federal Test Procedure) tests on the same catalyst validating the estimation approach. However,
the parameters vary significantly between RMC and Hot-FTP tests. This is predominantly due to the differences in the
temperature ranges experienced by the catalyst during the tests, which affects the Arrhenius temperature dependence of
the rate constant and the concentration of viable voids on the catalyst surface. The temperature range of the data
considered during the RMC tests is between 240°C and 360°C, while for the Hot-FTP tests it is between 200°C and 300°C.
\begin{table}[ht]
        \centering
        \caption{Parameter Estimates using FTIR Data}
        \label{tab::parm_FTIR}
        \begin{tabular}{l c c c c}
                \hline \hline
                Test & $\hat \theta_1 $ & $\hat \theta_2 $ & $\hat \theta_3 $ & $\sigma$\\ \hline \hline
                        DG RMC 1  & -0.63 & 0.55 & 38.74 & 1.14\\
                        DG RMC 2  & -0.82 & 0.69 & 38.61 & 1.08\\
                        DG RMC 3  & -0.80 & 1.12 & 38.78 & 1.10\\
                        Aged RMC  & -1.31 & 1.89 & 36.94 & 1.01\\ \hline
                        DG Hot FTP 1 & -2.14 & -5.80 & 39.53 & 1.35\\
                        DG Hot FTP 2 & -3.45 & -7.99 & 38.05 & 1.30\\
                        DG Hot FTP 3 & -3.37 & -6.42 & 38.88 & 1.33\\
                        Aged Hot FTP & -4.54 & -8.01 & 37.00 & 1.31\\
                \hline\hline
        \end{tabular}
\end{table}
% ======================================================================================================================
The parameter estimates using $NO_x$ sensor measurements from the test-cell experiments are tabulated in
Table-\ref{tab::parm_nox}. The parameter estimates are different from those obtained using FTIR measurements
(Table-\ref{tab::parm_FTIR}) due to the cross-sensitivity of the $NO_x$ sensor to ammonia. However, the parameters show
consistency across repeated tests on the same catalyst under similar test conditions. The $NO_x$ sensor cross-
sensitivity error $\varepsilon_{\chi}(T)$ introduces bias in the parameter estimates as part of the error that has the
same temperature dependence as the model structure enters into the parameter estimates. Thus, the linear programming problem estimated the parameters with an added bias term as shown below:
\begin{align}
    \varepsilon_{\chi}(T) &= \phi_{sat}(T(k))^T \theta_{\varepsilon_{\chi}} + \epsilon_{\chi}\\
    J(\theta_{sat})
                    &= \sum_{k=0}^{N-1} \lr{ \phi^T(k) \lr{\theta_{sat} + \theta_{\varepsilon_{\chi}}} - \eta (k) }
\end{align}
Thus, the estimated parameters using $NO_x$ sensor measurements will have a bias dependent on the cross-sensitivity
error characteristics of the sensor used. From, the experimental results, it is observed that $NO_x$ sensor
cross-sensitivity reduces the difference between the estimates of aging-factor
(Figure~\ref{fig::alpha_sat_ssd}, \ref{fig::alpha_sat_iod}) between aged and degreened catalysts. But the overall trend
of change in parameter values remains consistent.
%===
\begin{table}[ht]
        \centering
        \caption{Parameter Estimates using $NO_x$ Sensor Data}
        \label{tab::parm_nox}
        \begin{tabular}{l c c c c}
                \hline \hline
                Test & $\hat \theta_1 $ & $\hat \theta_2 $ & $\hat \theta_3 $ & $\sigma$ \\ \hline \hline
                        DG RMC 1 & -0.70 & 0.56 & 38.48 & 1.10\\
                        DG RMC 2 & -1.18 & 0.93 & 38.43 & 1.07\\
                        DG RMC 3 & -0.54 & 0.62 & 38.99 & 1.06\\
                        Aged RMC & -1.03 & 1.26 & 37.80 & 1.00\\ \hline
                        DG Hot FTP 1 & -2.30 & -6.11 & 37.08 & 1.28\\
                        DG Hot FTP 2 & -2.71 & -6.13 & 36.22 & 1.27\\
                        DG Hot FTP 3 & -3.92 & -6.17 & 35.81 & 1.30\\
                        Aged Hot FTP & -6.42 & -13.06 & 34.73 & 1.34\\
                \hline\hline
        \end{tabular}
\end{table}
%==
% ==================================================================================================================
Furthermore, the above delineated parameter estimation method has a limitation that the dataset must have response
of the catalyst under saturation for finite duration. If this condition is not met, the algorithm estimates the
parameters considering the maximum concentration of adsorbed ammonia as the concentration corresponding to catalyst
saturation (Appendix-\ref{sec::necessary_cond}).
% ==================================================================================================================
\subsection{Limitation: Parameter Estimation Assumes Catalyst Saturation \label{sec::necessary_cond}}
The catalyst mode detection algorithm presented fails when the catalyst is not saturated within the duration for a
significant length of time in the data. If the data-set does not have any regions of saturation, the approach fits a
saturated model as if the maximum achieved catalyst ammonia storage, $(\max\lr{\sigma(k)})$, is the storage capacity of
the catalyst $(\Gamma)$. This is demonstrated using data from high-fidelity simulation model (AVL Cruise
\cite{AVLCruise}) tuned to RMC (Ramped Mode Cycle) test on a test-cell. The RMC test is then simulated with $\pm 20\%$
change in the gain of urea-dosing controller to change the length of the data where the catalyst remains saturated.
\begin{figure}[ht]
        \centering
        \includegraphics[width=\figWidth]{./figs/5-sat_detect/3_parm_ID/eta_sim.png}
        \caption{RMC $\eta$ simulation with $\pm 20\%$ Urea Dosing}
        \label{fig::necessary_cond_sim}
\end{figure}
\begin{figure}[ht]
        \centering
        \includegraphics[width=\figWidth]{./figs/5-sat_detect/3_parm_ID/eta_sat.png}
        \caption{RMC $\eta_{sat}$ prediction with $\pm 20\%$ Urea Dosing}
        \label{fig::necessary_cond_sat}
\end{figure}
Figure~\ref{fig::necessary_cond_sim} shows no increase in $NO_x$ reduction when urea dosing is increased by $20\%$,
which indicates that maximum $NO_x$ reduction is already reached at nominal urea dosing. This could be because the
catalyst is saturated or because the inlet $NO_x$ concentration is sufficiently low. Conversely, a $20\%$ decrease in
urea dosing decreases $NO_x$ reduction. When the $-20\%$ urea-dosing data are used to estimate saturated-model
parameters, the model predicts a substantially lower $NO_x$ reduction under saturation than for the nominal and $+20\%$
dosing cases (Figure~\ref{fig::necessary_cond_sat}), the latter two giving essentially identical results. Thus for the nominal and $+20\%$ dosing cases, the catalyst is indeed saturated, and the parameter estimates show the same. However, for the $-20\%$ dosing case, the catalyst is not saturated in the data, instead the parameters fit the maximum surface coverage achieved $\lrf{\theta_{max} = \sfrac{\gamma_{max}}{\Gamma}}$ in the data, i.e.,
\begin{align}
        \eta_{sat}(k+1) &= \frac{u_1(k)}{F(k)} \tau_0 \; k_{scr} \Gamma \theta_{max} = \phi_{sat} \theta_{sat}
\end{align}

It is assumed that under ordinary drive conditions, the catalyst will reach saturation for a significant length of time, making the proposed parameter estimation approach applicable.

\subsection{Catalyst Mode Detection}
The saturated catalyst model's response is not dependent on the previous state but only on the inputs, making the
response independent of the initial conditions. Thus, using the parameter estimates, a given data point can be
classified as saturated catalyst response if the prediction error for the same inputs (inlet $NO_x$, $u_1$, and
mass flow rate, $F$) is less than the $\epsilon_{sat}$.
\begin{align}
        \abs{\hat \eta_{sat}(k) - \eta(k)} \leq \epsilon_{sat} \implies \text{Catalyst Saturation} \label{eqn::sat_cond}
\end{align}
For the current test-cell and truck dataset, $\epsilon_{sat} = 2.5\times 10^{-3} \, mol/m^{-3}$. This value is chosen
based on the variance of distribution of the prediction error for the saturated catalyst model.
\begin{figure}[!ht]
        \begin{minipage}{0.49\textwidth}
                \begin{figure}[H]
                        \centering
                        \includegraphics[width=\textwidth]{./figs/5-sat_detect/3_parm_ID/SatDetect_aged_rmc_FTIR.png}
                \end{figure}
        \end{minipage}
        \begin{minipage}{0.49\textwidth}
                \begin{figure}[H]
                        \centering
                        \includegraphics[width=\textwidth]{./figs/5-sat_detect/3_parm_ID/SatDetect_dg_rmc_1_FTIR.png}
                \end{figure}
        \end{minipage}
        \begin{minipage}{0.49\textwidth}
                \begin{figure}[H]
                        \centering
                        \includegraphics[width=\textwidth]{./figs/5-sat_detect/3_parm_ID/SatDetect_dg_rmc_2_FTIR.png}
                \end{figure}
        \end{minipage}
        \begin{minipage}{0.49\textwidth}
                \begin{figure}[H]
                        \centering
                        \includegraphics[width=\textwidth]{./figs/5-sat_detect/3_parm_ID/SatDetect_dg_rmc_3_FTIR.png}
                \end{figure}
        \end{minipage}
        \caption{Catalyst mode detection results for test-cell data}
        \label{fig::cat_mode_det}
\end{figure}
The catalyst mode detection for DG RMC 1 test-cell data using FTIR measurements is shown in
Figure~\ref{fig::cat_mode_det}. The catalyst is found to be close to saturation (or at its maximum $NO_x$ reduction)
when the mass flow rate and the urea dosing rates are high, consistent with the physical understanding of the catalyst
operation. Further, the flow rate can introduce a virtual ceiling on the maximum achievable $NO_x$ reduction due to
reduced residence time in the catalyst, which would be indistinguishable from saturated catalyst behavior as the $NO_x$
reduction would be independent of urea dosing.

A fraction of the data from the test-cell experiments under RMC and Hot-FTP test conditions for both degreened and aged
catalysts are found to be in saturated mode. The number of data points close to catalyst saturation $\lr{N_{sat}}$ and
the total number of samples ($N$) for different test conditions is tabulated in Tables~\ref{tab::sat_data_perc_ssd} and
\ref{tab::sat_data_perc_iod}.
%==============================================================
\begin{table}[ht]
        \centering
        \caption{Data Points Close to Catalyst Saturation in Test-Cell FTIR Data}
        \label{tab::sat_data_perc_ssd}
        \begin{tabular}{l r r}
                \hline \hline
                Test & $N$ & $N_{sat}$ \\\hline \hline
                DG RMC 1  & 2402 & 490  \\
                DG RMC 2  & 2402 & 493  \\
                DG RMC 3  & 2402 & 491  \\
                Aged RMC  & 2402 & 515  \\ \hline
                DG Hot FTP 1 & 634 & 186  \\
                DG Hot FTP 2 & 634 & 181  \\
                DG Hot FTP 3 & 638 & 180  \\
                Aged Hot FTP & 636 & 198  \\
                \hline\hline
        \end{tabular}
\end{table}
\begin{table}[ht]
        \centering
        \caption{Data Points Close to Catalyst Saturation in Test-Cell $NO_x$ Sensor Data}
        \label{tab::sat_data_perc_iod}
        \begin{tabular}{l r r}
                \hline \hline
                Test & $N$ & $N_{sat}$ \\\hline \hline
                DG RMC 1  & 2402 & 492  \\
                DG RMC 2  & 2402 & 494  \\
                DG RMC 3  & 2402 & 493  \\
                Aged RMC  & 2402 & 515  \\ \hline
                DG Hot FTP 1 & 634 & 194  \\
                DG Hot FTP 2 & 634 & 191  \\
                DG Hot FTP 3 & 638 & 198  \\
                Aged Hot FTP & 636 & 220  \\
                \hline\hline
        \end{tabular}
\end{table}
Note that the higher number of data points classified as close to catalyst saturation in RMC test conditions, compared
to Hot-FTP, results in better parameter estimates and performance of the aging detection test statistic in this case
(Table~\ref{tab::Tw_tst}).

\section{Catalyst Aging Detection in Test Cell Data}
The parameters of the saturated system are a function of the rate-constant for SCR reaction and the maximum surface
concentration of viable voids on the catalyst. The adsorption site concentration is hypothesized to change with aging of
the catalyst allowing the parameters to capture the aging of the catalyst. This is demonstrated using the test-cell
experiments on a new (degreened) and hydro-thermally aged catalyst with the same capacity. The predicted $NO_x$
reduction for an aged catalyst is found to be statistically lower than the predicted $NO_x$ reduction in a degreened
catalyst (Figure-\ref{fig::aging_diff}).
\begin{figure}[ht]
        \centering
        \includegraphics[width=\figWidth]{./figs/5-sat_detect/3_parm_ID/AgingDiff.png}
        \caption{Predicted $\eta_{sat}$ for Aged and Degreened Catalyst}
        \label{fig::aging_diff}
\end{figure}
%===
When the $\eta_{sat}$ response is normalized with respect to the inlet $NO_x$ concentration and flow-rate, we end up
with a quantity that is only a function of temperature and represents the product of rate-constant and maximum surface
coverage. From (\ref{eqn::regression}),
\begin{align}
        \alpha_{sat} &= \frac{F(k)}{u_1(k)} \eta_{sat}(k) = \tau_0 k_{scr}\Gamma = \phi_T^T(k)\theta_{sat}\\
        \text{where,} \quad \phi_T^T(k) &= \bm{2\lr{\frac{T-T_0}{T_r}}^2-1 & \frac{T-T_0}{T_r} & 1}
\end{align}
%===
This quantity, $\alpha_{sat}$ can be estimated and changes with aging of the catalyst based on the hypothesis that the
aging changes the adsorption site concentration. The change in $\alpha_{sat}$ with aging is demonstrated using the
test-cell experiments (Figures~\ref{fig::alpha_sat_ssd} and \ref{fig::alpha_sat_iod}).
\begin{figure}[!ht]
        \begin{minipage}{0.49\textwidth}
                \begin{figure}[H]
                        \centering
                        \includegraphics[width=\textwidth]{./figs/5-sat_detect/3_parm_ID/aging_factor_ssd.png}
                        \caption{$\hat \alpha_{sat}$ from RMC FTIR Data}
                        \label{fig::alpha_sat_ssd}
                \end{figure}
        \end{minipage}
        \begin{minipage}{0.49\textwidth}
                \begin{figure}[H]
                        \centering
                        \includegraphics[width=\textwidth]{./figs/5-sat_detect/3_parm_ID/aging_factor_iod.png}
                        \caption{$\hat \alpha_{sat}$ from RMC $NO_x$ Sensor Data}
                        \label{fig::alpha_sat_iod}
                \end{figure}
        \end{minipage}
\end{figure}
%===
\par Thus, $\theta_{sat}$ captures the aging of the catalyst. Hence, the catalyst aging detection problem can be posed
as a hypothesis testing problem with null hypothesis $\lr{\mathcal{H}_0}$ corresponding to a degreened catalyst and
alternative hypothesis $\lr{\mathcal{H}_1}$ corresponding to an aged catalyst. Since the distribution of the error is
assumed to be known (half-normal) for both hypotheses and differs by the unknown parameter $\theta_{sat}$ which can be
estimated, the following Wald test \cite{wald1943tests} is proposed to check if the parameter estimates for the given
data set correspond to an aged catalyst or a known degreened catalyst.
Decide $\mathcal{H}_1$ if,
\begin{align}
        T_w  &= \lr{\hat \theta_{sat} - \theta_{sat}^{dg}}^T I\lr{\hat \theta_{sat}} \lr{\hat \theta_{sat} - \theta_{sat}^{dg}} > \beta
\end{align}
where $\theta_{sat}^{dg}$ is the parameter estimate for the degreened catalyst under the same input conditions,
including temperature, flow-rate and urea-dosing rate ranges. The threshold $\beta$ is decided based on the experimental
results. For RMC and FTP data sets, the value of the test-statistic is tabulated in Table~\ref{tab::Tw_tst}.
% Table of test-statistic
\begin{table}[ht]
        \centering
        \caption{Wald Test Results On the Test Cell Data}
        \label{tab::Tw_tst}
        \begin{tabular}{l c c}
        \hline \hline
        Test & $T_w$ from FTIR & $T_w$ from $NO_x$ sensor \\\hline \hline
        DG RMC 1      & 0.26 & 5.54\\
        DG RMC 2      & 0.00 & 0.00\\
        DG RMC 3      & 6.71 & 5.46\\
        Aged RMC      & 47.16 & 17.68\\
        \hline
        DG Hot FTP 1  & 0.49 & 0.81\\
        DG Hot FTP 2  & 3.26 & 1.98\\
        DG Hot FTP 3  & 0.00 & 0.00\\
        Aged Hot FTP  & 4.84 & 11.44\\
        \hline \hline
        \end{tabular}
\end{table}
The distribution of the test-statistic under aging (non-central $\chi^2$, asymptotically) is different for RMC and FTP
test conditions due to the differences in the ranges of inputs (temperature, flow-rate and urea-dosing rate) experienced
by the catalyst during the tests. Based on the results, a reasonably discerning threshold that maximizes the probability of detection for Hot-FTP and RMC test conditions is found to be:
\begin{align}
        \beta_{hFTP} &= 4.00, \qquad \beta_{RMC} = 10.00
\end{align}
Further investigation with more data is needed to arrive at the probability of detection and false alarm for these
thresholds. The proposed Wald test detects the aged catalyst for both FTIR and $NO_x$ sensor measurements under both
test conditions.

