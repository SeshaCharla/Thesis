\chapter{Catalyst Saturation and Aging Detection}
The surface concentration of viable voids $(\Gamma)$ is hypothesized to decrease with aging \cite{daya2018development},
\cite{daya2020kinetic}. This effect is reflected in the lumped parameters of the dynamic model for $NO_x$ reduction
developed in previous chapter (Figure~\ref{fig::no_x_swtiching}). The impact of $\Gamma$ reduction on parameter
estimates is more pronounced in the saturated catalyst model, as this model does not exhibit the noise amplification
caused by the backward difference present in the regression model of the unsaturated system. For $NO_x$ process dynamics
(\ref{eqn::NOX_process}) under catalyst saturation, $\sigma(k) = \Gamma$.
\begin{align}
    \implies \eta_{sat} (k+1) &=  \frac{u_1(k)}{F(k)} \tau_0 \; k_{scr}(T(k)) \; \Gamma (T(k))
        \label{eqn::nox_mdl}
\end{align}
%===
The above equation~(\ref{eqn::nox_mdl}) shows that the tailpipe $NO_x$ concentration becomes independent of urea dosing
under catalyst saturation. Polynomial approximations of the Arrhenius temperature dependence of the rate constant,
$k_{scr}$, and the concentration of viable voids, $\Gamma$ (\cite{nova2014urea}, \cite{ciardelli2004scr} and ,
\cite{joo2008study}), are used to obtain a linear in parameters model for the $NO_x$ process dynamics.
\begin{align}
    k_{scr}(T) &= A_{scr} e^{\lr{-\frac{E_{scr}}{RT}}} \approx \sum k_i T^i  \qquad i = 0, 1, \hdots
    \label{eqn::rate_const} \\
    \Gamma(T) &= S_1 e^{-S_2 T} \approx \sum r_i T^i \qquad i = 0, 1, \hdots
    \label{eqn::gamma}
\end{align}
%===
The above approximate models for temperature introduce dependence of the parameter estimates based on the temperature
range in the samples. For numerical stability \cite{press2003numerical}, the given temperature range is mapped to
$[-1,1]$, and Chebyshev polynomial bases \cite{trefethen2019approximation} are used instead of standard polynomials.
Incorporating equations (\ref{eqn::residence_time_mdl}) and (\ref{eqn::gamma}) into the $NO_x$ process dynamic model
(\ref{eqn::nox_mdl}),
\begin{align}
    \eta_{sat} (k+1) &= \phi_{sat}^T(T(k)) \; \theta_{sat}
    \label{eqn::regression} \\
    %===
    \phi_{sat} (k) &= \frac{u_1(k)}{F(k)} \underbrace{\bm{2\lr{\frac{T-T_0}{T_r}}^2-1 & \frac{T-T_0}{T_r} & 1}}_{\phi_2(T)}
    \label{eqn::phi_def} \\
    %===
    \text{Where, } \quad T_0 &= \frac{T_{max} + T_{min}}{2}, \quad T_r = \frac{T_{max} - T_{min}}{2} \notag
\end{align}
Rearranging the terms of equation (\ref{eqn::regression}), we get a quantity that involves the product of the rate
constant and the surface concentration of viable voids, $k_{scr} \Gamma$, i.e.,
\begin{align}
    \alpha_{sat} (k) &= \frac{F(k)}{u_1(k)} \eta_{sat} (k) = \tau_0 k_{scr}(T(k)) \Gamma(T(k)) = \phi_2(T(k))^T \theta_{sat}
\end{align}
\begin{figure}[!ht]
    \centering
    \includegraphics[width=0.75\textwidth]{./figs/5-sat_detect/1_intro/alpha_sat_1.png}
    \caption{The time plot of $\alpha_{sat}$ for aged and degreened catalysts for RMC cycle}
    \label{fig::alpha_sat_1}
\end{figure}
The quantity $\alpha_{sat}$ can be estimated from the input-output data of the system and is expected to decrease with
aging. This is verified on the test-cell data by plotting the estimated $\alpha_{sat}$ against time
(Figure~\ref{fig::alpha_sat_1}) and temperature (Figure~\ref{fig::alpha_sat_0}) for an aged and a degreened catalyst.
The estimated $\alpha_{sat}$ for the aged catalyst is lower than that of the degreened catalyst, which is consistent
with the hypothesis of reduction in surface concentration of viable voids with aging.
\begin{figure}[!ht]
    \centering
    \includegraphics[width=0.7\textwidth]{./figs/5-sat_detect/1_intro/alpha_sat.png}
    \caption{Estimated $\alpha_{sat}$ for an aged and a degreened catalyst for Hot-FTP and RMC cycles}
    \label{fig::alpha_sat_0}
\end{figure}
The difference in $\alpha_{sat}$ for aged and degreened catalysts is statistically significant as shown in
Figure~\ref{fig::alpha_sat_0}. The estimation method that provides the variance is delineated in the subsequent
sections. Thus, the parameter estimates statistically capture the aging effects on the catalyst.

% \input{secs/5-sat_detect/paper/3-MLE_estimation.tex}
% \input{secs/5-sat_detect/paper/4-catalyst_mode_detection.tex}
% \input{secs/5-sat_detect/paper/5-catalyst_aging.tex}
% \input{secs/5-sat_detect/paper/6-trucks.tex}
