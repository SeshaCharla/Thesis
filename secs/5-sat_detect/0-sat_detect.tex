\chapter{Catalyst Saturation and Aging Detection}
The surface concentration of viable voids $(\Gamma)$ is hypothesized to decrease with aging \cite{daya2018development,daya2020kinetic}. This effect is reflected in the lumped parameters of the dynamic model for $NO_x$ reduction
developed in the previous chapter (Figure~\ref{fig::no_x_switching}). The impact of $\Gamma$ reduction on parameter
estimates is more pronounced in the saturated catalyst model, as this model does not exhibit the noise amplification
caused by the backward difference present in the regression model of the unsaturated system. For $NO_x$ process dynamics
(\ref{eqn::NOX_process}) under catalyst saturation, $\sigma(k) = \Gamma$.
\begin{align}
    \implies \eta_{sat} (k+1) &=  \frac{u_1(k)}{F(k)} \tau_0 \; k_{scr}(T(k)) \; \Gamma (T(k))
        \label{eqn::nox_mdl}
\end{align}
%===
The above equation~(\ref{eqn::nox_mdl}) shows that the tailpipe $NO_x$ concentration becomes independent of urea dosing
under catalyst saturation. Polynomial approximations of the Arrhenius temperature dependence of the rate constant,
$k_{scr}$, and the concentration of viable voids, $\Gamma$ \cite{nova2014urea,ciardelli2004scr,joo2008study}, are used to obtain a linear in parameters model for the $NO_x$ process dynamics.
\begin{align}
    k_{scr}(T) &= A_{scr} e^{\lr{-\frac{E_{scr}}{RT}}} \approx \sum k_i T^i  \qquad i = 0, 1, \hdots
    \label{eqn::rate_const} \\
    \Gamma(T) &= S_1 e^{-S_2 T} \approx \sum r_i T^i \qquad i = 0, 1, \hdots
    \label{eqn::gamma}
\end{align}
%===
The above approximate models for temperature introduce dependence of the parameter estimates based on the temperature
range in the samples. For numerical stability \cite{press2003numerical}, the given temperature range is mapped to
$[-1,1]$, and Chebyshev polynomial bases \cite{trefethen2019approximation} are used instead of standard polynomials.
Incorporating equations (\ref{eqn::residence_time_mdl}) and (\ref{eqn::gamma}) into the $NO_x$ process dynamic model
(\ref{eqn::nox_mdl}),
\begin{align}
    \eta_{sat} (k+1) &= \phi_{sat}^T(T(k)) \; \theta_{sat}
    \label{eqn::regression} \\
    %===
    \phi_{sat} (k) &= \frac{u_1(k)}{F(k)} \underbrace{\bm{2\lr{\frac{T-T_0}{T_r}}^2-1 & \frac{T-T_0}{T_r} & 1}}_{\phi_2(T)}
    \label{eqn::phi_def} \\
    %===
    \text{where } \quad T_0 &= \frac{T_{max} + T_{min}}{2}, \quad T_r = \frac{T_{max} - T_{min}}{2} \notag
\end{align}
Rearranging the terms of equation (\ref{eqn::regression}), we get a quantity that involves the product of the rate
constant and the surface concentration of viable voids, $k_{scr} \Gamma$, i.e.,
\begin{align}
    \alpha_{sat} (k) &= \frac{F(k)}{u_1(k)} \eta_{sat} (k) = \tau_0 k_{scr}(T(k)) \Gamma(T(k)) = \phi_2(T(k))^T \theta_{sat}
    \label{eqn::alpha_sat}
\end{align}
\begin{figure}[!ht]
    \centering
    \includegraphics[width=\figWidth]{./figs/5-sat_detect/1_intro/alpha_sat_1.png}
    \caption{Time plot of $\alpha_{sat}$ for aged and degreened catalysts for RMC cycle}
    \label{fig::alpha_sat_1}
\end{figure}
The parameter $\alpha_{sat}$ can be estimated from the available measurements. A plot of $\alpha_{sat}$ over time reveals a significant decrease in its magnitude for an aged catalyst compared to a degreened catalyst. The preceding equation indicates that $\alpha_{sat}$ is solely temperature-dependent. When plotted as a function of temperature, the magnitude of $\alpha_{sat}$ is consistently and statistically lower for the aged catalyst than for the degreened catalyst in both Hot-FTP and RMC tests. These findings support the hypothesis that aging reduces the surface concentration of viable voids ($\Gamma$).
\begin{figure}[!ht]
    \centering
    \includegraphics[width=0.7\textwidth]{./figs/5-sat_detect/1_intro/alpha_sat.png}
    \caption{Estimated $\alpha_{sat}$ for an aged and a degreened catalyst for Hot-FTP and RMC cycles}
    \label{fig::alpha_sat_0}
\end{figure}
Thus, the parameter estimates statistically capture the aging effects on the catalyst. The next step is to develop a
statistical test for catalyst saturation and aging detection based on the parameter estimates. The test is designed to
be implemented in real-time on the engine control unit (ECU) of a vehicle, enabling on-board diagnostics for catalyst
health monitoring.

\section{Framework for Catalyst Aging Detection: Statistical Hypothesis Testing}
The catalyst aging detection problem can be formulated as a composite hypothesis testing problem, where the null
hypothesis, $\mathcal{H}_0$, corresponds to a degreened catalyst and the alternative hypothesis, $\mathcal{H}_1$,
corresponds to an aged catalyst. Thus, the problem can be stated as follows:
\begin{align}
        \mathcal{H}_0 &: \hat \theta_{sat} = \theta^{dg}_{sat} \quad \text{(Degreened Catalyst)} \\
        \mathcal{H}_1 &: \hat \theta_{sat} \neq \theta^{dg}_{sat} \quad \text{(Aged Catalyst)}
\end{align}
where $\hat \theta_{sat}$ is the parameter estimate obtained from the available measurements and $\theta^{dg}_sat$ is the parameter value corresponding to a degreened catalyst. The test statistic for this hypothesis testing problem can be constructed based on the parameter estimates and their variances. For a given data set $\pmb x$, let $p(\pmb x; \theta)$ denote the likelihood function of the data given the parameter $\theta$. Thus, we have the generalized likelihood ratio test (GLRT) \cite{kay1998detection} as follows:
\begin{align}
\text{Decide } \mathcal{H}_1 \text{ if } & \notag \\
        L_G (\pmb x) &= \frac{p(\pmb x; \hat \theta_{sat})}{p(\pmb x; \theta^{dg}_{sat})} > \beta'
\end{align}
where $\hat \theta_{sat}$ is the maximum likelihood estimate of $\theta$ based on the data $\pmb x$ and $\beta'$ is the
threshold for the test. For large sample sizes $(N)$, the asymptotic probability density function (PDF) of the maximum likelihood estimate (MLE) is attained, and the variance of $\hat \theta_{sat}$ attains the Cramer-Rao lower bound (CRLB) \cite{kay1993estimation}. Thus:
\begin{align}
        \frac{\partial ln p(\pmb x; \theta_{sat})}{\partial \theta_{sat}} = I(\theta_{sat})(\hat \theta_{sat} - \theta_{sat})
        \label{eqn::log_ln_P}
\end{align}
where $\theta_{sat}$ is the true parameter value and $I(\theta_{sat})$ is the Fisher information matrix. As $N \rightarrow \infty$, $\hat \theta_{sat} \rightarrow \theta_{sat}$. We can use the first-order Taylor expansion of the fisher information matrix:
\begin{align}
        \lrb{I(\theta_{sat})}_{ij} &=  \lrb{I(\theta_{sat})}_{ij} + \left. \frac{\partial \lrb{I(\theta_{sat})}_{ij}}{\partial \theta_{sat}} \right|_{\theta_{sat} = \hat \theta_{sat}} (\hat \theta_{sat} - \theta_{sat})
        \label{eqn::I_taylor}
\end{align}
Neglecting the higher-order terms in the Taylor expansion, we get the asymptotic equivalence of the fisher information matrix at the true parameter value and the MLE:
\begin{align}
        I(\theta_{sat}) = I(\hat \theta_{sat}) \qquad \text{as } N \rightarrow \infty
        \label{eqn::I_asymp}
\end{align}
Thus, substituting the above equation into equation (\ref{eqn::log_ln_P}), we get:
\begin{align}
        \frac{\partial ln p(\pmb x; \theta_{sat})}{\partial \theta_{sat}} = I(\hat \theta_{sat})(\hat \theta_{sat} - \theta_{sat})
        \label{eqn::log_ln_P_2}
\end{align}
Integrating the above equation with respect to $\theta_{sat}$, we get:
\begin{align}
        p(\pmb x; \theta_{sat}) = p(\pmb x; \hat \theta_{sat}) e^{-\frac{1}{2} (\hat \theta_{sat} - \theta_{sat})^T I(\hat \theta_{sat}) (\hat \theta_{sat} - \theta_{sat})}
        \label{eqn::p_x_theta}
\end{align}
The $\theta_{sat}$ in the above derivation is $\theta^{dg}_{sat}$, the parameter value corresponding to a degreened catalyst. The GLRT can be rewritten as follows:
\begin{align}
\text{Decide } \mathcal{H}_1 \text{ if } & \notag \\
        L_G (\pmb x) &= \frac{p(\pmb x; \hat \theta_{sat})}{p(\pmb x; \theta^{dg}_{sat})} = e^{\frac{1}{2} (\hat \theta_{sat} - \theta^{dg}_{sat})^T I(\hat \theta_{sat}) (\hat \theta_{sat} - \theta^{dg}_{sat})} > \beta'
\end{align}
Taking logarithm on both sides, we get the Wald-test \cite{wald1943tests} statistic, $T_w$:
\begin{align}
\text{Decide } \mathcal{H}_1 \text{ if } & \notag \\
        T_w &= 2 \ln L_G (\pmb x) = (\hat \theta_{sat} - \theta^{dg}_{sat})^T I(\hat \theta_{sat}) (\hat \theta_{sat} - \theta^{dg}_{sat}) > \beta (= 2 \ln \beta')
        \label{eqn::wald_statistic}
\end{align}
The distribution of the test-statistic $T_w$, can be obtained from the distribution of the MLE, $\hat \theta_{sat}$. As $N \rightarrow \infty$, the MLE is asymptotically normally distributed with mean $\theta_{sat}$ and covariance matrix $I^{-1}(\theta_{sat})$ \cite{kay1993estimation}. Thus,
\begin{align}
        \hat \theta_{sat} \sim
        \begin{cases}
        \mathcal{N}(\theta^{dg}_{sat}, I^{-1}(\theta^{dg}_{sat})) & \text{under } \mathcal{H}_0 \\
        \mathcal{N}(\theta^{ag}_{sat}, I^{-1}(\theta^{ag}_{sat})) & \text{under } \mathcal{H}_1
        \end{cases}
\end{align}
Using the similar arguments previously asymptotic equivalence of fisher information matrix (\ref{eqn::I_asymp}):
\begin{align}
        I(\hat \theta_{sat}) (\hat \theta_{sat} - \theta^{dg}_{sat}) = I(\theta^{dg}_{sat}) (\hat \theta_{sat} - \theta^{dg}_{sat}) = I(\theta^{ag}_{sat}) (\hat \theta_{sat} - \theta^{dg}_{sat})
\end{align}
Thus, the distribution of the test statistic $T_w$ would be $\chi^2$ with degrees of freedom equal to the dimension of $\theta_{sat}$ under the null hypothesis $\mathcal{H}_0$ and a non-central $\chi^2$ distribution under the alternative hypothesis $\mathcal{H}_1$ \cite{kay1998detection}.
\begin{align}
        T_w &\sim
        \begin{cases}
        \chi^2_k & \text{under } \mathcal{H}_0 \\
        \chi^2_k (\lambda) & \text{under } \mathcal{H}_1
        \end{cases} \\
        \text{where,} \quad \lambda &= (\theta^{ag}_{sat} - \theta^{dg}_{sat})^T I(\theta^{ag}_{sat}) (\theta^{ag}_{sat} - \theta^{dg}_{sat})
\end{align}

The threshold $\beta$ can be chosen using the Neyman-Pearson criterion \cite{kay1998detection} to achieve a desired false alarm probability, $P_{FA}$ and maximize the detection probability, $P_D$. The false alarm probability is given by:
\begin{align}
        P_{FA} &= P(T_w > \beta | \mathcal{H}_0) = 1 - F_{\chi^2}(\beta)
\end{align}
where $F_{\chi^2_k}(\beta)$ is the cumulative distribution function (CDF) of the $\chi^2$ distribution with $k$ degrees of freedom. The detection probability is given by:
\begin{align}
        P_D &= P(T_w > \beta | \mathcal{H}_1) = 1 - F_{\chi^2_k (\lambda)}(\beta)
\end{align}
where $F_{\chi^2_k (\lambda)}(\beta)$ is the CDF of the non-central $\chi^2$ distribution with $k$ degrees of freedom and non-centrality parameter $\lambda$. The threshold $\beta$ can be chosen to achieve a desired $P_{FA}$ and maximize $P_D$. In practice, the sample size $N$ may not be large enough to achieve the asymptotic properties of the MLE, and the distribution of the test statistic may deviate from the theoretical distributions. In such cases, the threshold can be determined empirically using techniques such as bootstrapping or Monte Carlo simulations to estimate the distribution of the test statistic under both hypotheses.

A more direct and worst case performance measure for the detector is the probability of missed-detection, $P_{MD}$, which is the complement of the detection probability:
\begin{align}
        P_{MD} &= 1 - P_D = F_{\chi^2_k (\lambda)}(\beta)
\end{align}

Wald-test is used for catalyst aging detection due to its computational efficiency and asymptotic optimality properties. The test statistic can be computed using the parameter estimates and their variances, which can be obtained from the available measurements. The test can be implemented in real-time on the engine control unit (ECU) of a vehicle, enabling on-board diagnostics for catalyst health monitoring. The next step is to develop a framework for estimating the parameters and their variances from the available measurements, which will be discussed in the next section.

\section{Maximum Likelihood Estimation of the Saturated Catalyst Model Parameters}
The catalyst switches between saturation mode and normal mode based on states that are not completely controllable experimentally. The bounding condition on the saturated system's response (\ref{eqn::eta_bound}) can be used for detecting segments of the data where the catalyst's operation is normal or saturated. As the saturated catalyst's response forms a tight upper bound to the actual system's response, the parameters of the system will be the solution to the linear programming problem that minimizes the area under the curve with the lower bound as the response of the actual system.
\begin{align}
\text{min. } \, \sum \eta_{sat}(k) \quad
\text{s.t } \, \lr{\eta_{sat}(k) \geq \eta(k)} \quad \forall  k \label{eqn::lin_opt_1}
\end{align}
%===
The optimization problem (\ref{eqn::lin_opt_1}) is feasible as $\eta_{sat}, \eta \geq 0$ always. When the $NO_x$ sensor
data is used, $\eta(k)$ is replaced with $\eta_y(k)$ as the bounding condition still holds (\ref{eqn::eta_y_bound}) .
Rewriting the optimization problem using regression vector (\ref{eqn::regression}), we have the linear programming
problem for estimating the parameters of the catalyst:
\begin{align}
        \text{minimize} &\qquad \pmb 1^T \lrb{\Phi_{sat} \theta_{sat}}      \label{eqn::lin_prog}\\
        \text{subject to} &\qquad \Phi_{sat} \theta_{sat} \succeq H,  \notag\\
        \text{where, } & \qquad
        \Phi_{sat} = \bm{\phi_{sat}(1), \phi_{sat}(2), \hdots, \phi_{sat}(N-1)}^T \notag\\
        &\qquad H = \bm{ \eta(2), \eta(3), \hdots, \eta(N)}^T \notag\\
        & \text{$\pmb 1$ is a column vector of ones.} \notag
\end{align}
%===
The above estimate of $\theta_{sat}$ is the \itbf{maximum likelihood estimate (MLE) under a half-normal distribution
\cite{942636} for model structure error}, $\lr{\varepsilon_\eta}$, between the saturated system's response,
$\eta_{sat}(k)$, and the actual response of the catalyst, $\eta(k)$ (Appendix-\ref{sec::mle_derivation}). In the
case of $NO_x$-sensor data with cross-sensitivity, the model structure error includes the cross-sensitivity error
$\varepsilon_{\chi}$ and is assumed to be a half-normal random variable.
%===
\input{secs/5-sat_detect/paper/3-subs/3-0-Apx-MLE_derivation.tex}
\input{secs/5-sat_detect/paper/3-subs/3-1-theta_dist.tex}
\input{secs/5-sat_detect/paper/3-subs/3-2-eta_sat_variance.tex}
\input{secs/5-sat_detect/paper/3-subs/3-3-FTIR_results.tex}
\input{secs/5-sat_detect/paper/3-subs/3-4-NOx_sensor_results.tex}
\input{secs/5-sat_detect/paper/3-subs/3-1-Apx-necessary_condition.tex}

\section{Parameter Estimation and Aging Detection in Test Cell Data}
Using the parameter estimation approach outlined in the previous section, we analyze the test cell data to estimate the
parameters of the saturated catalyst model. The parameter estimates using FTIR measurements from the test cell
experiments are tabulated in Table-\ref{tab::parm_FTIR}. The parameters show consistency across repeated RMC (Ramped
Mode Cycle) and Hot-FTP (Federal Test Procedure) tests on the same catalyst validating the estimation approach. However,
the parameters vary significantly between RMC and Hot-FTP tests. This is predominantly due to the differences in the
temperature ranges experienced by the catalyst during the tests, which affects the Arrhenius temperature dependence of
the rate constant and the concentration of viable voids on the catalyst surface. The temperature range of the data
considered during the RMC tests is between 240°C and 360°C, while for the Hot-FTP tests it is between 200°C and 300°C.
\begin{table}[ht]
        \centering
        \caption{Parameter Estimates using FTIR Data}
        \label{tab::parm_FTIR}
        \begin{tabular}{l c c c c}
                \hline \hline
                Test & $\hat \theta_1 $ & $\hat \theta_2 $ & $\hat \theta_3 $ & $\sigma$\\ \hline \hline
                        DG RMC 1  & -0.63 & 0.55 & 38.74 & 1.14\\
                        DG RMC 2  & -0.82 & 0.69 & 38.61 & 1.08\\
                        DG RMC 3  & -0.80 & 1.12 & 38.78 & 1.10\\
                        Aged RMC  & -1.31 & 1.89 & 36.94 & 1.01\\ \hline
                        DG Hot FTP 1 & -2.14 & -5.80 & 39.53 & 1.35\\
                        DG Hot FTP 2 & -3.45 & -7.99 & 38.05 & 1.30\\
                        DG Hot FTP 3 & -3.37 & -6.42 & 38.88 & 1.33\\
                        Aged Hot FTP & -4.54 & -8.01 & 37.00 & 1.31\\
                \hline\hline
        \end{tabular}
\end{table}
% ======================================================================================================================
The parameter estimates using $NO_x$ sensor measurements from the test-cell experiments are tabulated in
Table-\ref{tab::parm_nox}. The parameter estimates are different from those obtained using FTIR measurements
(Table-\ref{tab::parm_FTIR}) due to the cross-sensitivity of the $NO_x$ sensor to ammonia. However, the parameters show
consistency across repeated tests on the same catalyst under similar test conditions. The $NO_x$ sensor cross-
sensitivity error $\varepsilon_{\chi}(T)$ introduces bias in the parameter estimates as part of the error that has the
same temperature dependence as the model structure enters into the parameter estimates. Thus, the linear programming problem estimated the parameters with an added bias term as shown below:
\begin{align}
    \varepsilon_{\chi}(T) &= \phi_{sat}(T(k))^T \theta_{\varepsilon_{\chi}} + \epsilon_{\chi}\\
    J(\theta_{sat})
                    &= \sum_{k=0}^{N-1} \lr{ \phi^T(k) \lr{\theta_{sat} + \theta_{\varepsilon_{\chi}}} - \eta (k) }
\end{align}
Thus, the estimated parameters using $NO_x$ sensor measurements will have a bias dependent on the cross-sensitivity
error characteristics of the sensor used. From, the experimental results, it is observed that $NO_x$ sensor
cross-sensitivity reduces the difference between the estimates of aging-factor
(Figure~\ref{fig::alpha_sat_ssd}, \ref{fig::alpha_sat_iod}) between aged and degreened catalysts. But the overall trend
of change in parameter values remains consistent.
%===
\begin{table}[ht]
        \centering
        \caption{Parameter Estimates using $NO_x$ Sensor Data}
        \label{tab::parm_nox}
        \begin{tabular}{l c c c c}
                \hline \hline
                Test & $\hat \theta_1 $ & $\hat \theta_2 $ & $\hat \theta_3 $ & $\sigma$ \\ \hline \hline
                        DG RMC 1 & -0.70 & 0.56 & 38.48 & 1.10\\
                        DG RMC 2 & -1.18 & 0.93 & 38.43 & 1.07\\
                        DG RMC 3 & -0.54 & 0.62 & 38.99 & 1.06\\
                        Aged RMC & -1.03 & 1.26 & 37.80 & 1.00\\ \hline
                        DG Hot FTP 1 & -2.30 & -6.11 & 37.08 & 1.28\\
                        DG Hot FTP 2 & -2.71 & -6.13 & 36.22 & 1.27\\
                        DG Hot FTP 3 & -3.92 & -6.17 & 35.81 & 1.30\\
                        Aged Hot FTP & -6.42 & -13.06 & 34.73 & 1.34\\
                \hline\hline
        \end{tabular}
\end{table}
%==
% ==================================================================================================================
Furthermore, the above delineated parameter estimation method has a limitation that the dataset must have response
of the catalyst under saturation for finite duration. If this condition is not met, the algorithm estimates the
parameters considering the maximum concentration of adsorbed ammonia as the concentration corresponding to catalyst
saturation (Appendix-\ref{sec::necessary_cond}).
% ==================================================================================================================
\subsection{Limitation: Parameter Estimation Assumes Catalyst Saturation \label{sec::necessary_cond}}
The catalyst mode detection algorithm presented fails when the catalyst is not saturated within the duration for a
significant length of time in the data. If the data-set does not have any regions of saturation, the approach fits a
saturated model as if the maximum achieved catalyst ammonia storage, $(\max\lr{\sigma(k)})$, is the storage capacity of
the catalyst $(\Gamma)$. This is demonstrated using data from high-fidelity simulation model (AVL Cruise
\cite{AVLCruise}) tuned to RMC (Ramped Mode Cycle) test on a test-cell. The RMC test is then simulated with $\pm 20\%$
change in the gain of urea-dosing controller to change the length of the data where the catalyst remains saturated.
\begin{figure}[ht]
        \centering
        \includegraphics[width=\figWidth]{./figs/5-sat_detect/3_parm_ID/eta_sim.png}
        \caption{RMC $\eta$ simulation with $\pm 20\%$ Urea Dosing}
        \label{fig::necessary_cond_sim}
\end{figure}
\begin{figure}[ht]
        \centering
        \includegraphics[width=\figWidth]{./figs/5-sat_detect/3_parm_ID/eta_sat.png}
        \caption{RMC $\eta_{sat}$ prediction with $\pm 20\%$ Urea Dosing}
        \label{fig::necessary_cond_sat}
\end{figure}
Figure~\ref{fig::necessary_cond_sim} shows no increase in $NO_x$ reduction when urea dosing is increased by $20\%$,
which indicates that maximum $NO_x$ reduction is already reached at nominal urea dosing. This could be because the
catalyst is saturated or because the inlet $NO_x$ concentration is sufficiently low. Conversely, a $20\%$ decrease in
urea dosing decreases $NO_x$ reduction. When the $-20\%$ urea-dosing data are used to estimate saturated-model
parameters, the model predicts a substantially lower $NO_x$ reduction under saturation than for the nominal and $+20\%$
dosing cases (Figure~\ref{fig::necessary_cond_sat}), the latter two giving essentially identical results. Thus for the nominal and $+20\%$ dosing cases, the catalyst is indeed saturated, and the parameter estimates show the same. However, for the $-20\%$ dosing case, the catalyst is not saturated in the data, instead the parameters fit the maximum surface coverage achieved $\lrf{\theta_{max} = \sfrac{\gamma_{max}}{\Gamma}}$ in the data, i.e.,
\begin{align}
        \eta_{sat}(k+1) &= \frac{u_1(k)}{F(k)} \tau_0 \; k_{scr} \Gamma \theta_{max} = \phi_{sat} \theta_{sat}
\end{align}

It is assumed that under ordinary drive conditions, the catalyst will reach saturation for a significant length of time, making the proposed parameter estimation approach applicable.

\subsection{Catalyst Mode Detection}
The saturated catalyst model's response is not dependent on the previous state but only on the inputs, making the
response independent of the initial conditions. Thus, using the parameter estimates, a given data point can be
classified as saturated catalyst response if the prediction error for the same inputs (inlet $NO_x$, $u_1$, and
mass flow rate, $F$) is less than the $\epsilon_{sat}$.
\begin{align}
        \abs{\hat \eta_{sat}(k) - \eta(k)} \leq \epsilon_{sat} \implies \text{Catalyst Saturation} \label{eqn::sat_cond}
\end{align}
For the current test-cell and truck dataset, $\epsilon_{sat} = 2.5\times 10^{-3} \, mol/m^{-3}$. This value is chosen
based on the variance of distribution of the prediction error for the saturated catalyst model.
\begin{figure}[!ht]
        \begin{minipage}{0.49\textwidth}
                \begin{figure}[H]
                        \centering
                        \includegraphics[width=\textwidth]{./figs/5-sat_detect/3_parm_ID/SatDetect_aged_rmc_FTIR.png}
                \end{figure}
        \end{minipage}
        \begin{minipage}{0.49\textwidth}
                \begin{figure}[H]
                        \centering
                        \includegraphics[width=\textwidth]{./figs/5-sat_detect/3_parm_ID/SatDetect_dg_rmc_1_FTIR.png}
                \end{figure}
        \end{minipage}
        \begin{minipage}{0.49\textwidth}
                \begin{figure}[H]
                        \centering
                        \includegraphics[width=\textwidth]{./figs/5-sat_detect/3_parm_ID/SatDetect_dg_rmc_2_FTIR.png}
                \end{figure}
        \end{minipage}
        \begin{minipage}{0.49\textwidth}
                \begin{figure}[H]
                        \centering
                        \includegraphics[width=\textwidth]{./figs/5-sat_detect/3_parm_ID/SatDetect_dg_rmc_3_FTIR.png}
                \end{figure}
        \end{minipage}
        \caption{Catalyst mode detection results for test-cell data}
        \label{fig::cat_mode_det}
\end{figure}
The catalyst mode detection for DG RMC 1 test-cell data using FTIR measurements is shown in
Figure~\ref{fig::cat_mode_det}. The catalyst is found to be close to saturation (or at its maximum $NO_x$ reduction)
when the mass flow rate and the urea dosing rates are high, consistent with the physical understanding of the catalyst
operation. Further, the flow rate can introduce a virtual ceiling on the maximum achievable $NO_x$ reduction due to
reduced residence time in the catalyst, which would be indistinguishable from saturated catalyst behavior as the $NO_x$
reduction would be independent of urea dosing.

A fraction of the data from the test-cell experiments under RMC and Hot-FTP test conditions for both degreened and aged
catalysts are found to be in saturated mode. The number of data points close to catalyst saturation $\lr{N_{sat}}$ and
the total number of samples ($N$) for different test conditions is tabulated in Tables~\ref{tab::sat_data_perc_ssd} and
\ref{tab::sat_data_perc_iod}.
%==============================================================
\begin{table}[ht]
        \centering
        \caption{Data Points Close to Catalyst Saturation in Test-Cell FTIR Data}
        \label{tab::sat_data_perc_ssd}
        \begin{tabular}{l r r}
                \hline \hline
                Test & $N$ & $N_{sat}$ \\\hline \hline
                DG RMC 1  & 2402 & 490  \\
                DG RMC 2  & 2402 & 493  \\
                DG RMC 3  & 2402 & 491  \\
                Aged RMC  & 2402 & 515  \\ \hline
                DG Hot FTP 1 & 634 & 186  \\
                DG Hot FTP 2 & 634 & 181  \\
                DG Hot FTP 3 & 638 & 180  \\
                Aged Hot FTP & 636 & 198  \\
                \hline\hline
        \end{tabular}
\end{table}
\begin{table}[ht]
        \centering
        \caption{Data Points Close to Catalyst Saturation in Test-Cell $NO_x$ Sensor Data}
        \label{tab::sat_data_perc_iod}
        \begin{tabular}{l r r}
                \hline \hline
                Test & $N$ & $N_{sat}$ \\\hline \hline
                DG RMC 1  & 2402 & 492  \\
                DG RMC 2  & 2402 & 494  \\
                DG RMC 3  & 2402 & 493  \\
                Aged RMC  & 2402 & 515  \\ \hline
                DG Hot FTP 1 & 634 & 194  \\
                DG Hot FTP 2 & 634 & 191  \\
                DG Hot FTP 3 & 638 & 198  \\
                Aged Hot FTP & 636 & 220  \\
                \hline\hline
        \end{tabular}
\end{table}
Note that the higher number of data points classified as close to catalyst saturation in RMC test conditions, compared
to Hot-FTP, results in better parameter estimates and performance of the aging detection test statistic in this case
(Table~\ref{tab::Tw_tst}).

\section{Catalyst Aging Detection in Test Cell Data}
The parameters of the saturated system are a function of the rate-constant for SCR reaction and the maximum surface
concentration of viable voids on the catalyst. The adsorption site concentration is hypothesized to change with aging of
the catalyst allowing the parameters to capture the aging of the catalyst. This is demonstrated using the test-cell
experiments on a new (degreened) and hydro-thermally aged catalyst with the same capacity. The predicted $NO_x$
reduction for an aged catalyst is found to be statistically lower than the predicted $NO_x$ reduction in a degreened
catalyst (Figure-\ref{fig::aging_diff}).
\begin{figure}[ht]
        \centering
        \includegraphics[width=\figWidth]{./figs/5-sat_detect/3_parm_ID/AgingDiff.png}
        \caption{Predicted $\eta_{sat}$ for Aged and Degreened Catalyst}
        \label{fig::aging_diff}
\end{figure}
%===
When the $\eta_{sat}$ response is normalized with respect to the inlet $NO_x$ concentration and flow-rate, we end up
with a quantity that is only a function of temperature and represents the product of rate-constant and maximum surface
coverage. From (\ref{eqn::regression}),
\begin{align}
        \alpha_{sat} &= \frac{F(k)}{u_1(k)} \eta_{sat}(k) = \tau_0 k_{scr}\Gamma = \phi_T^T(k)\theta_{sat}\\
        \text{where,} \quad \phi_T^T(k) &= \bm{2\lr{\frac{T-T_0}{T_r}}^2-1 & \frac{T-T_0}{T_r} & 1}
\end{align}
%===
This quantity, $\alpha_{sat}$ can be estimated and changes with aging of the catalyst based on the hypothesis that the
aging changes the adsorption site concentration. The change in $\alpha_{sat}$ with aging is demonstrated using the
test-cell experiments (Figures~\ref{fig::alpha_sat_ssd} and \ref{fig::alpha_sat_iod}).
\begin{figure}[!ht]
        \begin{minipage}{0.49\textwidth}
                \begin{figure}[H]
                        \centering
                        \includegraphics[width=\textwidth]{./figs/5-sat_detect/3_parm_ID/aging_factor_ssd.png}
                        \caption{$\hat \alpha_{sat}$ from RMC FTIR Data}
                        \label{fig::alpha_sat_ssd}
                \end{figure}
        \end{minipage}
        \begin{minipage}{0.49\textwidth}
                \begin{figure}[H]
                        \centering
                        \includegraphics[width=\textwidth]{./figs/5-sat_detect/3_parm_ID/aging_factor_iod.png}
                        \caption{$\hat \alpha_{sat}$ from RMC $NO_x$ Sensor Data}
                        \label{fig::alpha_sat_iod}
                \end{figure}
        \end{minipage}
\end{figure}
%===
\par Thus, $\theta_{sat}$ captures the aging of the catalyst. Hence, the catalyst aging detection problem can be posed
as a hypothesis testing problem with null hypothesis $\lr{\mathcal{H}_0}$ corresponding to a degreened catalyst and
alternative hypothesis $\lr{\mathcal{H}_1}$ corresponding to an aged catalyst. Since the distribution of the error is
assumed to be known (half-normal) for both hypotheses and differs by the unknown parameter $\theta_{sat}$ which can be
estimated, the following Wald test \cite{wald1943tests} is proposed to check if the parameter estimates for the given
data set correspond to an aged catalyst or a known degreened catalyst.
Decide $\mathcal{H}_1$ if,
\begin{align}
        T_w  &= \lr{\hat \theta_{sat} - \theta_{sat}^{dg}}^T I\lr{\hat \theta_{sat}} \lr{\hat \theta_{sat} - \theta_{sat}^{dg}} > \beta
\end{align}
where $\theta_{sat}^{dg}$ is the parameter estimate for the degreened catalyst under the same input conditions,
including temperature, flow-rate and urea-dosing rate ranges. The threshold $\beta$ is decided based on the experimental
results. For RMC and FTP data sets, the value of the test-statistic is tabulated in Table~\ref{tab::Tw_tst}.
% Table of test-statistic
\begin{table}[ht]
        \centering
        \caption{Wald Test Results On the Test Cell Data}
        \label{tab::Tw_tst}
        \begin{tabular}{l c c}
        \hline \hline
        Test & $T_w$ from FTIR & $T_w$ from $NO_x$ sensor \\\hline \hline
        DG RMC 1      & 0.26 & 5.54\\
        DG RMC 2      & 0.00 & 0.00\\
        DG RMC 3      & 6.71 & 5.46\\
        Aged RMC      & 47.16 & 17.68\\
        \hline
        DG Hot FTP 1  & 0.49 & 0.81\\
        DG Hot FTP 2  & 3.26 & 1.98\\
        DG Hot FTP 3  & 0.00 & 0.00\\
        Aged Hot FTP  & 4.84 & 11.44\\
        \hline \hline
        \end{tabular}
\end{table}
The distribution of the test-statistic under aging (non-central $\chi^2$, asymptotically) is different for RMC and FTP
test conditions due to the differences in the ranges of inputs (temperature, flow-rate and urea-dosing rate) experienced
by the catalyst during the tests. Based on the results, a reasonably discerning threshold that maximizes the probability of detection for Hot-FTP and RMC test conditions is found to be:
\begin{align}
        \beta_{hFTP} &= 4.00, \qquad \beta_{RMC} = 10.00
\end{align}
Further investigation with more data is needed to arrive at the probability of detection and false alarm for these
thresholds. The proposed Wald test detects the aged catalyst for both FTIR and $NO_x$ sensor measurements under both
test conditions.


\section{Aging Detection from Truck Data}
%===
The performance of the proposed aging detector is also demonstrated using the $NO_x$ sensor data from four long-haul
trucks. The road data were collected from trucks operating in the United States for two days separated by a few years
(Table~\ref{tab::truck_data_mileage}) during which the catalyst has aged based on the existing performance measures. The
data contains measurements of $NO_x$ sensors before and after the catalyst, as well as other required variables
including flow rate, temperature and urea injection rate. The preprocessed data are then partitioned into drive segments
with no breaks which correspond to a continuous driving period from engine start to engine stop. Unlike the test-cell data where the engine drive-cycles are repeatable and produce a significant fraction of samples
where the catalyst is close to saturation, the truck data drive segments constitute varied driving conditions with fewer
samples $\lrf{\leq 10\%}$ that are close to catalyst saturation. A necessary condition for the proposed parameter
estimation and aging detection framework is that the data contains a sufficiently large number of samples close to
catalyst saturation such that the asymptotic properties of the MLE and the test statistic hold (\ref{eqn::Ns_min}).
Hence, the drive segment data are further pruned to only include segments with at least $150$ samples ($120$ in the case
of Truck C) close to catalyst saturation.
\begin{align}
        N_{sat} &\geq N_{sat}^{min} \label{eqn::Ns_min} \\
        \beta_{truck} &= 250 \label{eqn::beta_truck}
\end{align}
The threshold of the Wald test statistic (\ref{eqn::beta_truck}) for aging detection in trucks is chosen such that it
maximizes the probability of detection while minimizing the probability of false alarm and missed detection for the
given limited truck data. A larger data set would help arrive at the thresholds using Neyman-Pearson criteria
\cite{kay1998detection}. The value of the test statistic for the drive segments for each truck is tabulated in
Tables~\ref{tab::adt_aging_res}, \ref{tab::mes_aging_res}, \ref{tab::wer_aging_res} and \ref{tab::trw_aging_res}. A
summary of the aging detection results is tabulated in Table~\ref{tab::truck_aging_summary}.
%===
\begin{table}[!ht]
        \centering
        \caption{Summary of Aging Detection Results on Truck Data}
        \label{tab::truck_aging_summary}
        \begin{tabular}{c c c c c}
        \hline \hline
         Truck & Total Drive & Detection & False  & Missed \\
               & Segments    &         & Alarm & Detection  \\\hline \hline
        Truck A   & 8  & 4 & 0 & 4 \\
        Truck B   & 11 & 9 & 1 & 1 \\
        Truck C   & 9  & 7 & 1 & 1 \\
        Truck D   & 8  & 6 & 2 & 0 \\
        \hline \hline
        \end{tabular}
\end{table}
The detector performs reasonably well on three of the four trucks with high probability of detection and low false alarm
and missed detection rates. Truck A has relatively lower mileage difference between the earlier and latter data years
which may explain the lower detection performance. The performance is limited by the amount of data available with
sufficient samples close to catalyst saturation. Further, the lack of ground truth on the actual individual catalyst's
aging levels limits the ability to quantify the performance of the detector. Nevertheless, the results demonstrate the
potential of the proposed aging detection framework for on-board application using $NO_x$ sensor data.

% =================================================================

\subsection{Aging Detector Results on Truck Data Drive Segments}
The value of the Wald test statistic for each drive segment that met the necessary condition of sufficient samples close
to catalyst saturation is calculated and tabulated in Tables~\ref{tab::adt_aging_res}, \ref{tab::mes_aging_res}, \ref{tab::wer_aging_res} and \ref{tab::trw_aging_res}, for each truck. The number and percentage of samples close to
catalyst saturation in each drive segment is also tabulated. There is a strong correlation between detection of aging
and the number of samples close to catalyst saturation in the drive segment.
%===
% Truck Maping
% Truck A - AD Transport (adt)
% Truck B - Mesilla Valley (mes)
% Truck C - Werner (wer)
% Truck D - Transwest (trw)
% =====
\begin{table}[!ht]
        \centering
        \caption{Aging Detection Results for Truck A}
        \label{tab::adt_aging_res}
        \begin{tabular}{l r r r}
        \hline \hline
                Drive Segment& $T_w$ & $N_{sat}$ & $\% \lr{ \sfrac{N_{sat}}{N} }$ \\ \hline \hline
                A$\_15\_0$ & 0.00 & 797 & 6.70\\
                A$\_15\_1$ & 28.95 & 305 & 2.75\\
                A$\_15\_2$ & 58.50 & 276 & 5.93\\
                A$\_15\_3$ & 89.56 & 271 & 12.77\\ \hline
                A$\_17\_0$ & 65.59 & 414 & 2.97\\
                A$\_17\_1$ & 127.79 & 614 & 6.76\\
                A$\_17\_2$ & 12.13 & 330 & 8.44\\
                A$\_17\_3$ & 61.76 & 336 & 9.73\\
                \hline \hline
        \end{tabular}
\end{table}
\begin{table}[!ht]
        \centering
        \caption{Aging Detection Results for Truck B}
        \label{tab::mes_aging_res}
        \begin{tabular}{l r r r}
        \hline \hline
                Drive Segment& $T_w$ & $N_{sat}$ & $\% \lr{ \sfrac{N_{sat}}{N} }$ \\ \hline \hline
                B$\_15\_0$ & 672.43 & 167 & 1.26\\
                B$\_15\_1$ & 106.66 & 189 & 1.81\\
                B$\_15\_3$ & 237.79 & 159 & 2.95\\
                B$\_15\_4$ & 0.00 & 226 & 6.03\\
                B$\_15\_6$ & 84.73 & 179 & 5.75\\ \hline
                B$\_18\_0$ & 543.55 & 231 & 2.10\\
                B$\_18\_1$ & 1050.24 & 308 & 3.00\\
                B$\_18\_4$ & 22.48 & 195 & 6.14\\
                B$\_18\_6$ & 7150.19 & 378 & 12.73\\
                B$\_18\_7$ & 760.89 & 224 & 8.14\\
                B$\_18\_8$ & 342.67 & 246 & 9.04\\
        \hline \hline
        \end{tabular}
\end{table}
\begin{table}[!ht]
        \centering
        \caption{Aging Detection Results for Truck C}
        \label{tab::wer_aging_res}
        \begin{tabular}{l r r r}
        \hline \hline
                Drive Segment& $T_w$ & $N_{sat}$ & $\% \lr{ \sfrac{N_{sat}}{N} }$ \\ \hline \hline
                C$\_15\_1$ & 0.000 & 186 & 1.30\\
                C$\_15\_4$ & 113.86 & 143 & 3.84\\
                C$\_15\_5$ & 4340.08 & 124 & 4.16\\ \hline
                C$\_17\_0$ & 700.88 & 186 & 1.24\\
                C$\_17\_2$ & 3993.85 & 446 & 7.23\\
                C$\_17\_3$ & 1693.49 & 199 & 4.00\\
                C$\_17\_4$ & 186.64 & 120 & 2.81\\
                C$\_17\_7$ & 1878.52 & 129 & 4.73\\
                C$\_17\_9$ & 7500.10 & 218 & 11.63\\
        \hline \hline
        \end{tabular}
\end{table}
\begin{table}[!ht]
        \centering
        \caption{Aging Detection Results for Truck D}
        \label{tab::trw_aging_res}
        \begin{tabular}{l r r r}
        \hline \hline
                Drive Segment& $T_w$ & $N_{sat}$ & $\% \lr{ \sfrac{N_{sat}}{N} }$ \\ \hline \hline
                D$\_15\_0$ & 0.00 & 214 & 1.29 \\
                D$\_15\_2$ & 98.99 & 160 & 2.95 \\
                D$\_15\_4$ & 1546.55 & 213 & 4.90 \\
                D$\_15\_6$ & 935.05 & 152 & 8.86 \\ \hline
                D$\_16\_1$ & 423.20 & 246 & 3.67 \\
                D$\_16\_2$ & 1022.06 & 199 & 5.22 \\
                D$\_16\_3$ & 697.78 & 170 & 5.36 \\
                D$\_16\_5$ & 313.62 & 200 & 9.31 \\
        \hline \hline
        \end{tabular}
\end{table}

\section{Aging Detection Using Unsaturated Catalyst Model}

Similar to saturated catalyst model, a subset of unsaturated catalyst model parameters that are a function of surface concentration of viable voids, $\Gamma$, can be used for aging detection. From, (\ref{eqn::unsat_mdl}) have the parametric model for the unsaturated catalyst system as,
\begin{align}
        \bar \eta_F (k+1) = \bar \eta_F (k) &+ u_{2F}(k-1)\phi_2(k-1) \theta_\Gamma \notag\\
        & - u_{2F}(k-1) \bar \eta_F (k) \phi_1(k-1) \theta_{\eta_{ads}} \notag\\
        & - \bar \eta_F (k) \phi_1(k-1) \theta_{\eta_{od}} \notag\\
        & - u_1(k-1) \bar \eta_F (k) \phi_1(k-1) \theta_{\eta_{scr}} \label{eqn::unsat_parametrized}
\end{align}
where, individual parameter vectors correspond to specific reaction dynamics as follows:
\begin{enumerate}
        \item $\theta_{\Gamma}$: Adoption dynamics related to the surface concentration of viable voids, $\Gamma$.
        \item $\theta_{\eta_{ads}}$: Adsorption dynamics related to adsorbed ammonia, $\sigma$.
        \item $\theta_{\eta_{od}}$: Oxidation dynamics and desorption dynamics that remove ammonia from the catalyst surface without reducing $NO_x$.
        \item $\theta_{\eta_{scr}}$: SCR dynamics that reduce $NO_x$ using the adsorbed ammonia.
\end{enumerate}
Among the above four parameter vectors, $\theta_{\Gamma}$ is the only parameter vector that is a function of the surface
concentration of viable voids, $\Gamma$. Thus, $\theta_{\Gamma}$ can be used for aging detection. However, the other
parameters, $\theta_{\eta_{ads}}, \theta_{\eta_{od}}, \theta_{\eta_{scr}}$ are nuisance parameters that affect the
estimates of $\theta_{\Gamma}$ and should be accounted for in the aging detection test design.
\begin{align}
        \theta_{\nu} = \bm{\theta_{\eta_{ads}}^T & \theta_{\eta_{od}}^T & \theta_{\eta_{scr}}^T}^T
\end{align}
Thus, the aging detector (Wald Test) should account for the effect of nuisance parameters in the test statistic. Thus, we have the following Wald test \cite{wald1943tests},\cite{kay1998detection} with nuisance parameters for aging detection using the unsaturated catalyst model.
\begin{align}
\text{Decide } \mathcal{H}_1 \text{ if}, & \qquad \\
        T_w &= \lr{\hat \theta_{\Gamma} - \theta_{\Gamma}^{dg}}^T \lrf{\lrb{I^{-1} (\hat \theta)}_{\theta_{\Gamma} \theta_{\Gamma}}}^{-1} \lr{\hat \theta_{\Gamma} - \theta_{\Gamma}^{dg}} \geq \beta
        \\
        %===
        \lrf{\lrb{I^{-1} (\hat \theta)}_{\theta_{\Gamma} \theta_{\Gamma}}}^{-1} &=
        I_{\theta_{\Gamma} \theta_{\Gamma}} - I_{\theta_{\Gamma} \theta_{\nu}} \lrb{I_{\theta_{\nu} \theta_{\nu}}}^{-1} I_{\theta_{\nu} \theta_{\Gamma}}
        \\
        % ==
        I &= \bm{I_{\theta_{\Gamma} \theta_{\Gamma}} & I_{\theta_{\Gamma} \theta_{\nu}} \\
                I_{\theta_{\nu} \theta_{\Gamma}} & I_{\theta_{\nu} \theta_{\nu}}}
\end{align}

Where, $I(\hat \theta)$ is the Fisher Information Matrix evaluated at $\hat \theta$. The subscript notation indicates the sub-matrix of the Fisher Information Matrix corresponding to the parameters indicated by the subscripts.

The performance of the above Walt test for aging detection using the unsaturated catalyst model is evaluated using the
test cell data (\ref{tab::wald_unsat_testcell}). The test statistic, $T_w$, is computed for each test and compared
against a threshold, $\beta$, to determine if the test correctly identifies the aged catalyst. The results show that the
test statistic for the aged catalyst is significantly higher than that of the degreened catalyst, indicating that the
test can effectively distinguish between aged and degreened catalysts. However, the performance of the test is affected
by the presence of nuisance parameters, which can lead to false positives or false negatives if not properly accounted
for in the test design.

\begin{table}[!ht]
\caption{Wald test with nuisance parameters on Hot-FTP and RMC test cell data}
\label{tab::wald_unsat_testcell}
\begin{minipage}{0.49\textwidth}
\begin{table}[H]
        \centering
        \begin{tabular}{l c c}
                \hline \hline
                Test & $T_w$ (FTIR)     & $T_w$ ($NO_x$) \\ \hline
                $ dg\_hftp\_1 $ & 2.95 & 2.72\\
                $ dg\_hftp\_2 $ & 13.13 & 8.81\\
                $ dg\_hftp\_3 $ & 0.000 & 0.000\\
                $ aged\_hftp\ $ & 164.14 & 165.33\\
                \hline \hline
        \end{tabular}
\end{table}
\end{minipage}
\begin{minipage}{0.49\textwidth}
\begin{table}[H]
        \centering
        \begin{tabular}{l c c}
                \hline \hline
                Test & $T_w$ (FTIR)     & $T_w$ ($NO_x$) \\ \hline
                $dg\_rmc\_1$ & 42.55 & 34.11\\
                $dg\_rmc\_2$ & 0.000 & 0.000\\
                $dg\_rmc\_3$ & 30.11 & 28.67\\
                $aged\_rmc $ & 119.61 & 111.56\\
                \hline \hline
        \end{tabular}
\end{table}
\end{minipage}
\end{table}

%===
% Truck Maping
% Truck A - AD Transport (adt)
% Truck B - Mesilla Valley (mes)
% Truck C - Werner (wer)
% Truck D - Transwest (trw)
% =====

\begin{table}[!ht]
\caption{Wald test with nuisance parameters on trucks A and B}
\label{tab::wald_unsat_truck_AB}
\begin{minipage}{0.49\textwidth}
        \begin{table}[H]
                \centering
                \begin{tabular}{l c}
                        \hline \hline
                        Test & $T_w$ \\ \hline
                        $ A\_15\_0$ & 0.000\\
                        $ A\_15\_1$ & 964.250\\
                        $ A\_15\_2$ & 446.841\\
                        $ A\_15\_3$ & 16.968\\
                        \hline
                        $ A\_17\_0$ & 60.403\\
                        $ A\_17\_1$ & 421.009\\
                        $ A\_17\_2$ & 107.126\\
                        $ A\_17\_3$ & 22.801\\
                        \hline \hline
                \end{tabular}
        \end{table}
\end{minipage}
\begin{minipage}{0.49\textwidth}
\begin{table}[H]
        \centering
        \begin{tabular}{l c}
                \hline \hline
                Test & $T_w$ \\ \hline
                $ B\_15\_0$  & 0.000\\
                $ B\_15\_1$  & 43.255\\
                $ B\_15\_2$  & 78.956\\
                $ B\_15\_3$  & 159.709\\
                \hline
                $ B\_18\_0$  & 109.657\\
                $ B\_18\_1$  & 428.107\\
                $ B\_18\_2$  & 29.036\\
                $ B\_18\_3$  & 85.096\\
                \hline \hline
        \end{tabular}
\end{table}
\end{minipage}
\end{table}

\begin{table}[!ht]
\caption{Wald test with nuisance parameters on trucks C and D}
\label{tab::wald_unsat_truck_CD}
\begin{minipage}{0.49\textwidth}
\begin{table}[H]
        \centering
        \begin{tabular}{l c}
                \hline \hline
                Test & $T_w$ \\ \hline
                $ C\_15\_0$  & 0.000\\
                $ C\_15\_1$  & 319.991\\
                $ C\_15\_2$  & 423.146\\
                $ C\_15\_3$  & 154.224\\
                \hline
                $ C\_17\_0$  & 135.084\\
                $ C\_17\_1$  & 275.070\\
                $ C\_17\_2$  & 161.445\\
                $ C\_17\_3$  & 312.514\\
                \hline \hline
        \end{tabular}
\end{table}
\end{minipage}
\begin{minipage}{0.49\textwidth}
\begin{table}[H]
        \centering
        \begin{tabular}{l c}
                \hline \hline
                Test & $T_w$ \\ \hline
                $ D\_15\_0$ & 0.000\\
                $ D\_15\_1$ & 229.066\\
                $ D\_15\_2$ & 178.935\\
                $ D\_15\_3$ & 76.475\\
                \hline
                $ D\_16\_0$ & 229.306\\
                $ D\_16\_1$ & 75.937\\
                $ D\_16\_2$ & 103.586\\
                $ D\_16\_3$ & 63.516\\
                \hline \hline
        \end{tabular}
\end{table}
\end{minipage}
\end{table}


However, the noise amplification caused by the backward difference in the regression model of the unsaturated system makes it difficult to obtain reliable parameter estimates for aging detection. Thus, a more robust statistical test that accounts for the effect of nuisance parameters is needed for aging detection using the unsaturated catalyst model.

\section{Remarks on Detector Performance}

The study introduces a parameter estimation-based hypothesis testing algorithm for catalyst aging detection, based on
the hypothesis that model parameters associated with catalyst storage capacity change with aging
\cite{daya2018development} and \cite{daya2020kinetic}. The hypothesis testing framework requires not only the parameter
estimates but also their distributions. To address this, the maximum likelihood estimation (MLE) framework and its
asymptotic properties are employed. The linear programming problem (\ref{eqn::lin_prog}) for estimating the parameters
of the saturated model is demonstrated to be equivalent to MLE under a half-normal distribution of the model-structure
error between the actual system response and the saturated model prediction. For the unsaturated model, the
least-squares problem corresponds to MLE under the Gaussian distribution of the model-structure error. Utilizing the
parameter distributions, a Wald test-based detector (\ref{eqn::wald_statistic}) is proposed to quantify the distance
between the estimated parameters and the known degreened parameters, scaled by the Fisher information of the parameter
estimate distribution. The influence of nuisance parameters is also considered for the unsaturated model
(\ref{eqn::wald_nus}). The performance of the detector is evaluated using both test-cell and truck data sets.

The saturated model-based detector demonstrates greater reliability compared to the unsaturated model-based detector.
This improved performance can be attributed to the following factors:
\begin{enumerate}
        \item The parameter estimates of the saturated model do not require backward differencing, in contrast to the unsaturated model, which amplifies noise through this process.
        \item The saturated model contains fewer parameters than the unsaturated model, resulting in a less stringent persistence of excitation condition and reducing the risk of overfitting noise.
        \item The parameter estimation method for the saturated model is more robust to cross-sensitivity errors because it utilizes highly selective data for estimation. Only data points near saturation are employed for parameter estimation.
\end{enumerate}
Therefore, the saturated model parameter estimation-based detector represents a logical choice for catalyst aging
detection in SCR-ASC systems.

