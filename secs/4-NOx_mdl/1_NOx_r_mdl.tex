The discrete nonlinear recursive model that is linear in parameters can be derived from the molar conservation equations
directly using control volume (Figure~\ref{fig::ctrl_vol}). The model derivation involves three main steps:
\begin{figure}[!ht]
        \centering
        \includegraphics[width = 0.5\textwidth]{./figs/4-NOx_mdl/plug_flow_discrete.png}
        \caption{SCR-ASC system abstraction}
        \label{fig::ctrl_vol}
\end{figure}
\begin{enumerate}
        \item We have the molar-conservation equations across the control volume within one residence time.
                \begin{multline}
                        \underbrace{\mol{NO_x}^{out} }_{ \con{NO_x}^{out} \tau F_{vol} } (t + (i+1) \tau) =
                                \underbrace{\mol{NO_x}^{in} }_{ \con{NO_x}^{in} \tau F_{vol} } (t + i \tau)
                                + V_{scr} \int_0^{\tau}
                                                \underbrace{\frac{d}{dt} \con{NO_x}^{scr}}_
                                                        {k_{s2v} k_{scr} \con{NO_x}^{in} \con{NH_3}^{ads}}
                                           dt
                        \label{eqn::nox_bal}
                \end{multline}
                \begin{multline}
                       \mol{NH_3}^{ads} (t + (i+1) \tau) =
                                \underbrace{\mol{NH_3}^{ads} }_{A_{scr} \con{NH_3}^{ads}} (t + i \tau)
                                + A_{scr} \int_0^{\tau}
                                               \frac{d}{dt} \con{NH_3}^{ads}
                                               % \underbrace{}_{
                                               %         - \con{NH_3}^{ads}
                                               %                \lrf{
                                               %                k_{ads} \con{NH_3}^{in}
                                               %                + k_{scr} \con{NO_x}^{in}
                                               %                + k_{od}}
                                               %         + \Gamma k_{ads} \con{NH_3}^{in}
                                               %                }
                                                dt
                        \label{eqn::ads_bal}
                \end{multline}
        \item By summing the equations from (\ref{eqn::nox_bal}) for  $i = 0$ to $n-1$, where $n$ is the total number of residence times within one sampling period $(= t_s/\tau)$, we get the relation between the total number of moles of $NO_x$ that entered, reduced and left the SCR-ASC chamber. Finally, writing the equation in terms of the inlet and outlet concentrations we get the equation relating inlet and outlet concentrations at the end of sampling period as:
                \begin{multline}
                        \underbrace{\con{NO_x}^{out}}_{x_1}(t + t_s) =
                                \underbrace{\con{NO_x}^{in}}_{u_1}(t) - \tau(t) k_{s2v} k_{scr}(t) \con{NO_x}^{in}(t) \underbrace{\lrf{\frac{1}{n} \sum_{i = 0}^{n-1} \con{NH_3}^{ads}(t + i \tau)}}_{\sigma}
                        \label{eqn::nox_avg}
                \end{multline}

        \item The dynamics of $\sigma$, the average surface concentration of adsorbed ammonia on the catalyst in the
                given sampling period, can be obtained by summing the equations from (\ref{eqn::ads_bal}) for $i = 0$ to
                $n-1$ which is the sampling period. This results in a telescopic cancellation of the terms resulting in
                an equation relating the moles in the current sample to the moles in the next sample
                (section~\ref{sec::sigma_deriv}). As the ammonia desorption and ammonia oxidation have the same effect
                ($\sigma$ reduction) on the surface concentration of ammonia the rate constants can
                be lumped by addition, i.e., $k_{od} = k_{oxi} + k_{des}$. We have,
                \begin{multline}
                        \sigma(t + t_s) = \sigma(t) - \sigma(t) t_s \lrf{k_{ads}(t) \con{NH_3}^{in}(t)
                                                                                + k_{scr}(t) \con{NO_x}^{in}(t)
                                                                                + k_{od}(t)}
                                                \\
                                                + \Gamma t_s k_{ads}(t) \con{NH_3}^{in}(t)
                        \label{eqn::ads_avg}
                \end{multline}
        \item Finally, using (\ref{eqn::nox_avg}), $\sigma(t+t_s)$ and $\sigma(t)$ can be eliminated from
                (\ref{eqn::ads_avg}) to get a dynamic model for the $NO_x$-reduction process that explicitly depends
                on the measured inputs and states alone.
                \begin{align}
                        \text{Let }\quad \eta(k) = u_1(k-1) - x_1(k)
                \end{align}
\begin{align}
        \eta(k+1) =& \eta(k) \lrb{\frac{\tau(k)}{\tau(k-1)}}
                                \lrb{\frac{u_1(k)}{u_1(k-1)}}
                                \lrb{\frac{k_{scr}(k)}{k_{scr}(k-1)}} \notag \\
                &-\eta(k) \lrb{\frac{\tau(k)}{\tau(k-1)}}
                                \lrb{\frac{u_1(k)}{u_1(k-1)}}
                                \lrb{\frac{k_{scr}(k)}{k_{scr}(k-1)}}
                t_s k_{ads}(k-1) \con{NH_3}^{in}(k-1)
                \notag \\
                &-\eta(k) \lrb{\frac{\tau(k)}{\tau(k-1)}}
                                \lrb{\frac{u_1(k)}{u_1(k-1)}}
                                \lrb{\frac{k_{scr}(k)}{k_{scr}(k-1)}}
                t_s k_{scr}(k-1) u_1(k-1)
                \notag \\
                &-\eta(k) \lrb{\frac{\tau(k)}{\tau(k-1)}}
                                \lrb{\frac{u_1(k)}{u_1(k-1)}}
                                \lrb{\frac{k_{scr}(k)}{k_{scr}(k-1)}}
                t_s k_{od}(k-1)
                \notag \\
                &+ t_s k_{s2v} \times \lrf{\Gamma(k-1) \tau(k) u_1(k) \con{NH_3}^{in}(k-1)} \times \lr{k_{scr}(k) k_{ads}(k-1)}
                \label{eqn::disc_NOx}
\end{align}

\end{enumerate}

The above equation (\ref{eqn::disc_NOx}) is parametrized using the models for the physical properties and
urea-dosing and used for parameter estimation and validation.

Steps 1, 2 and 3 are discussed in detail in the previous sections. The step-4 derivation is presented below followed by the parametrization.


\section{Modelling using average concentration change of $NO_x$ due to reduction in a sample}
Let, \textbf{$\eta$ denote the change in concentration from inlet to outlet $NO_x$ in one sample}. This is the change in concentration in the exhaust due to the $NO_x$ reduction.
%
\begin{align}
        \eta(k) &= \con{NO_x}^{in}(k-1) - \con{NO_x}^{out}(k) = u_1(k-1) - x_1(k)
\end{align}
%
Thus, rewriting the $NO_x$ process dynamics (\ref{eqn::nox_avg}) in terms of $\eta$, we have,
\begin{align}
        \eta(k+1) &= \tau(k) k_{s2v} k_{scr}(k) u_1(k) \sigma(k)\\
        %===
        \implies \sigma(k) &= \frac{\eta(k+1)}{\tau(k) k_{s2v} k_{scr}(k) u_1(k)}
        \label{eqn::sigma_elim}
\end{align}
%
The above equation (\ref{eqn::sigma_elim}) can be used to eliminate the unknown quantity $(\sigma)$ from equation (\ref{eqn::ads_avg}). We have,
\begin{multline}
        \sigma(k+1) = \sigma(k) - \sigma(k) t_s k_{ads}(k) \con{NH_3}^{in}(t)
                        - \sigma(k) t_s k_{scr}(k) u_1(k)
                        - \sigma(k) t_s k_{od}(k)
                        \\ + \Gamma(k) t_s k_{ads}(k) \con{NH_3}^{in}(k)
\end{multline}
writing the above equation for $\sigma(k)$:
\begin{multline*}
         \sigma(k) = \sigma(k-1)\lrf{1 -  t_s k_{ads}(k-1) \con{NH_3}^{in}(t-1)
                        -  t_s k_{scr}(k-1) u_1(k-1)
                        -  t_s k_{od}(k-1)}
                        \\ + \Gamma(k-1) t_s k_{ads}(k-1) \con{NH_3}^{in}(k-1)
\end{multline*}
Let,
\begin{align}
        \gamma_{proc}(k-1) &= \lrf{1 -  t_s k_{ads}(k-1) \con{NH_3}^{in}(t-1)
                        -  t_s k_{scr}(k-1) u_1(k-1)
                        -  t_s k_{od}(k-1)}
        \label{eqn::gamma_proc}
        \\
        \implies \sigma(k) &= \sigma(k-1) \gamma_{proc}(k-1) + \Gamma(k-1) t_s k_{ads}(k-1) \con{NH_3}^{in}(k-1)
        \label{eqn::sig_gamma_proc}
\end{align}
Eliminating $\sigma(k)$ and $\sigma(k-1)$ from the above equation (\ref{eqn::sig_gamma_proc}) using equation (\ref{eqn::sigma_elim}):
\begin{multline*}
        \frac{\eta(k+1)}{\tau(k) k_{s2v} k_{scr}(k) u_1(k)}
        = \frac{\eta(k)}{\tau(k-1) k_{s2v} k_{scr}(k-1) u_1(k-1)} \times \gamma_{proc}(k-1)\\
                        \\ + \Gamma(k-1) t_s k_{ads}(k-1) \con{NH_3}^{in}(k-1)
\end{multline*}
Thus, we have the recursive equation for change in concentration due to reduction:
\begin{multline}
        \eta(k+1) = \eta(k) \times \lr{\frac{\tau(k)}{\tau(k-1)}}
                                \times \lr{\frac{u_1(k)}{u_1(k-1)}}
                                \times \lr{\frac{k_{scr}(k)}{k_{scr}(k-1)}}
                                \times \gamma_{proc}(k-1)
                                \\
                        + t_s k_{s2v} \times \lrf{\con{NH_3}^{in} \Gamma(k-1) \tau(k) u_1(k)} \times \lr{k_{scr}(k) k_{ads}(k-1)}
\end{multline}
Explicitly writing the individual terms:
\begin{align*}
        \eta(k+1) =& \eta(k) \lrb{\frac{\tau(k)}{\tau(k-1)}}
                                \lrb{\frac{u_1(k)}{u_1(k-1)}}
                                \lrb{\frac{k_{scr}(k)}{k_{scr}(k-1)}} \\
                &-\eta(k) \lrb{\frac{\tau(k)}{\tau(k-1)}}
                                \lrb{\frac{u_1(k)}{u_1(k-1)}}
                                \lrb{\frac{k_{scr}(k)}{k_{scr}(k-1)}}
                t_s k_{ads}(k-1) \con{NH_3}^{in}(k-1)
                \\
                &-\eta(k) \lrb{\frac{\tau(k)}{\tau(k-1)}}
                                \lrb{\frac{u_1(k)}{u_1(k-1)}}
                                \lrb{\frac{k_{scr}(k)}{k_{scr}(k-1)}}
                t_s k_{scr}(k-1) u_1(k-1)
                \\
                &-\eta(k) \lrb{\frac{\tau(k)}{\tau(k-1)}}
                                \lrb{\frac{u_1(k)}{u_1(k-1)}}
                                \lrb{\frac{k_{scr}(k)}{k_{scr}(k-1)}}
                t_s k_{od}(k-1)
                \\
                &+ t_s k_{s2v} \times \lrf{\Gamma(k-1) \tau(k) u_1(k) \con{NH_3}^{in}(k-1)} \times \lr{k_{scr}(k) k_{ads}(k-1)}
\end{align*}
\begin{equation}
       \label{eqn::nox_reduction_govern}
\end{equation}
The above equation can be simplified using the following two assumptions:
\begin{itemize}
        \item[$A1.$] The temperature doesn't change significantly across contiguous samples, i.e., $T(k-1) \approx T(k)$.
        \begin{align}
                \implies \frac{k_{scr}(k)}{k_{scr}(k-1)} \approx 1
        \end{align}
        \item[$A2.$] The product of rate constants results in a combined rate-constant,
        \begin{align}
                k_{scr}(k) k_{ads}(k-1) &\approx k_{scr}(k-1) k_{ads}(k-1)  = A_{scr}A_{ads} \exp\lrf{\frac{E_{scr}+E_{ads}}{R T(k-1)}} = k_{scr/ads}(k-1)
        \end{align}
\end{itemize}
Incorporating the above assumptions we have the dynamic model for change in concentration due to $NO_x$ reduction:
\begin{align*}
         \eta(k+1) =& \eta(k) \lrb{\frac{\tau(k)}{\tau(k-1)}}
                                \lrb{\frac{u_1(k)}{u_1(k-1)}}
                \\
                &-\eta(k) \lrb{\frac{\tau(k)}{\tau(k-1)}}
                                \lrb{\frac{u_1(k)}{u_1(k-1)}}
                t_s k_{ads}(k-1) \con{NH_3}^{in}(t-1)
                \\
                &-\eta(k) \lrb{\frac{\tau(k)}{\tau(k-1)}}
                                \lrb{\frac{u_1(k)}{u_1(k-1)}}
                t_s k_{od}(k-1)
                \\
                &-\eta(k) \lrb{\frac{\tau(k)}{\tau(k-1)}}
                t_s k_{scr}(k) u_1(k)
                \\
                &+ t_s k_{s2v} \times \lrf{\Gamma(k-1) \tau(k) u_1(k)\con{NH_3}^{in}(k-1)} \times k_{scr/ads}(k-1)
\end{align*}
\begin{equation}
       \label{eqn::nox_govern}
\end{equation}

\subsection{Parametrizing the $\eta$ dynamics}
The equation (\ref{eqn::nox_govern}) can be written in the following structure, with $g_i$'s denoting the corresponding expressions in each of the terms.
\begin{align}
        \eta(k+1) &= \eta(k) \lrf{ g_{\eta} - g_{ads} - g_{od} - g_{scr}} + g_\Gamma
\end{align}
The individual terms are parametrized based on the following relevant set of assumptions:
\begin{itemize}
        \item[$A3.$] The rate constant is linear or (quadratic) for a given operating range of temperature.
        %===
        \item[$A4.$] $\Gamma$ is a constant for a given operating range of temperature and only changes with aging.
        % ===
        %===
        \item[$A5.$] The model for $\con{NH_3}^{in}$ based on urea injection is given by equation (\ref{eqn::urea_inj}), i.e., $\con{NH_3}^{in}$ depends only on the flow-rate and urea injection but not the temperature (as the urea is preheated).
        \begin{align}
                \con{NH_3}^{in}(k) &= \nu_u \times \frac{u_2(k)}{F(k)}
                \label{eqn::urea_mdl}
        \end{align}
        \item[$A6.$] The model for residence time is given by equation (\ref{eqn::residence_time_mdl}), i.e., the
                residence time depends only on the flow-rate and effect of change in density (due to change in
                temperature) is negligible.
        \begin{align}
                \tau(k) &= \frac{V \rho_0}{F(k)} = \frac{\tau_0}{F(k)}
                \label{eqn::residence_time_mdl}
        \end{align}
\end{itemize}

Further, for numerical stability \cite{press2003numerical}, the given temperature range is mapped to $[-1,1]$ and Chebyshev polynomial basis functions \cite{trefethen2019approximation} are used instead of standard polynomials. We have the first and second order Chebyshev basis as follows:
\begin{align}
    \phi_1 (k) &= \bm{ \frac{T-T_0}{T_r} & 1}
    \label{eqn::phi_1} \\
    \phi_2 (k) &= \bm{2\lr{\frac{T-T_0}{T_r}}^2-1 & \frac{T-T_0}{T_r} & 1}
    \label{eqn::phi_2} \\
    %===
    \text{where } \quad T_0 &= \frac{T_{max} + T_{min}}{2}, \quad T_r = \frac{T_{max} - T_{min}}{2} \notag
\end{align}
The Chebyshev basis order is chosen based on the expected nonlinearity in the temperature dependence of the rate
constants. A first order basis is used when terms contain only rate constants, which are expected to vary nearly
linearly with temperature over the operating range. A second order basis is used when both rate constants and the total
surface concentration of voids ($\Gamma$) appear, since this dependence is expected to include an inflection point.
Using the above assumptions, we have the parametrization of individual terms:

\begin{enumerate}
\item \begin{align*}
        g_\Gamma &= t_s k_{s2v} \times \lrf{\Gamma(k-1) \tau(k) u_1(k) \con{NH_3}^{in}(k-1)} \times k_{scr/ads}(k-1)\\
                &= t_s k_{s2v} \Gamma \times \frac{\tau_0}{F(k)} \times u_1 (k) \times \nu_u \frac{u_2(k-1)}{F(k-1)}\times \pmb \phi^T(k-1) \pmb \theta_{scr/ads}
                \qquad \lrb{\because \ref{eqn::gamma_mdl}, \ref{eqn::residence_time_mdl}, \ref{eqn::k_mdl}, \ref{eqn::urea_mdl}}
\end{align*}
\begin{align}
        g_\Gamma &= \lrf{ \frac{u_1(k)}{F(k)} \frac{u_2(k-1)}{F(k-1)} \pmb \phi^T(k-1) } \times \lrf{ t_s k_{s2v} \nu_u \Gamma \tau_0 \pmb \theta_{scr/ads} }
\end{align}

\item \begin{align*}
        g_{\eta} &= \lrb{\frac{\tau(k)}{\tau(k-1)}}
                                \lrb{\frac{u_1(k)}{u_1(k-1)}}
                = \lrb{\frac{\frac{\tau_0}{F(k)}}{\frac{\tau_0}{F(k-1)}}} \lrb{\frac{u_1(k)}{u_1(k-1)}}
                \qquad \bm{\because \ref{eqn::residence_time_mdl}}
\end{align*}
\begin{align}
        g_{\eta} &= \lrb{\frac{u_1(k)}{F(k)}} \lrb{\frac{F(k-1)}{u_1(k-1)}}
\end{align}

\item \begin{align*}
        g_{ads} &= \lrb{\frac{\tau(k)}{\tau(k-1)}}
                                \lrb{\frac{u_1(k)}{u_1(k-1)}}
                t_s k_{ads}(k-1) \con{NH_3}^{in}(k-1)\\
        % ===
        &=\lrb{\frac{F(k-1)}{F(k)}}\lrb{\frac{u_1(k)}{u_1(k-1)}}
                t_s \times \phi_1^T(k-1) \theta_{ads} \times \nu_u \frac{u_2(t-1)}{F(t-1)}
                \qquad
                \bm{\because \ref{eqn::residence_time_mdl}, \ref{eqn::k_mdl}, \ref{eqn::urea_mdl}}
\end{align*}
\begin{align}
       g_{ads} &=  \lrf{\lrb{\frac{u_1(k)}{F(k)}}\lrb{\frac{u_2(k-1)}{u_1(k-1)}} \phi_1^T(k-1) }
                \times \lrf{\nu_u t_s \theta_{ads}}
\end{align}

\item \begin{align*}
        g_{od} &= \lrb{\frac{\tau(k)}{\tau(k-1)}}
                                \lrb{\frac{u_1(k)}{u_1(k-1)}}
                t_s k_{od}(k-1)\\
                &= \lrb{\frac{F(k-1)}{F(k)}}
                                \lrb{\frac{u_1(k)}{u_1(k-1)}}
                t_s \phi_1^T(k-1) \theta_{od}
                \qquad \bm{\because \ref{eqn::residence_time_mdl}, \ref{eqn::phi_1}}
\end{align*}
\begin{align}
        g_{od} &= \lrf{ \lrb{\frac{u_1(k)}{F(k)}} \lrb{\frac{F(k-1)}{u_1(k-1)}} \phi_1^T(k-1) }
                \times \lrf{t_s \theta_{od}}
\end{align}

\item \begin{align*}
        g_{scr} &= \lrb{\frac{\tau(k)}{\tau(k-1)}}
                t_s k_{scr}(k) u_1(k)\\
                &= \lrb{\frac{F(k-1)}{F(k)}} u_1(k) \phi_1^T(k) \theta_{scr} t_s
                \qquad \bm{\because \ref{eqn::residence_time_mdl}, \ref{eqn::k_mdl}}
\end{align*}
\begin{align}
        g_{scr} &= \lrf{\lrb{\frac{u_1(k)}{F(k)}} F(k-1) \phi_1^T(k)}
                \times \lrf{ t_s \theta_{scr} }
\end{align}

\end{enumerate}

Thus, we have the parametric form of the $NO_x$ reduction dynamics:

\begin{align}
        \eta(k+1) &= \eta(k) \lrb{\frac{u_1(k)}{F(k)}} \lrb{\frac{F(k-1)}{u_1(k-1)}}
                    - \phi^T_{\eta}(k) \theta_{\eta}  +  \phi_{\Gamma}^T(k)  \theta_{\Gamma}
        \label{eqn::eta_parm}
\end{align}
where,

\begin{minipage}{0.49\textwidth}
        \begin{align}
                \phi_{\eta}(k) &= \eta(k) \lrb{\frac{u_1(k)}{F(k)}}
                                \bm{\lrb{\frac{u_2(k-1)}{u_1(k-1)}}  \phi_1^T(k-1) \\
                                         \lrb{\frac{F(k-1)}{u_1(k-1)}} \phi_1^T(k-1)     \\
                                                 F(k-1) \phi_1^T(k)
                                                }
        \end{align}
\end{minipage}
\begin{minipage}{0.49\textwidth}
        \begin{align}
        \theta_{\eta} &= \bm{\nu_u t_s  \theta_{ads}\\
                                        t_s  \theta_{od} \\
                                        t_s  \theta_{scr}}
        \end{align}
\end{minipage}

\begin{minipage}{0.49\textwidth}
        \begin{align}
                \phi_{\Gamma} (k) &= \lrb{\frac{u_1(k)}{F(k)} } \lrb{ \frac{u_2(k-1)}{F(k-1)}} \phi_2(k-1)
        \end{align}
\end{minipage}
\begin{minipage}{0.49\textwidth}
        \begin{align}
                \theta_{\Gamma} (k) &= \bm{ t_s k_{s2v} \nu_u \Gamma \tau_0 \theta_{scr/ads} }
        \end{align}
\end{minipage}

\bigskip

Hence, we have the $NO_x$ dynamics:
\begin{align}
        x(k+1) &= u_1(k) - \eta(k) \lrb{\frac{u_1(k)}{F(k)}} \lrb{\frac{F(k-1)}{u_1(k-1)}}
                        + \phi^T_{\eta}(k) \theta_{\eta}  - \phi_{\Gamma}^T(k) \theta_{\Gamma}
        \label{eqn::nox_sim_mdl}\\
%===
        x(k+1) &= u_1(k) - \lrf{1 - \frac{x_1(k)}{u_1(k-1)}} \lrb{\frac{u_1(k)}{F(k)}} F(k-1)
                        + \phi^T_{\eta}(k) \theta_{\eta}  - \phi_{\Gamma}^T(k) \theta_{\Gamma}
\end{align}

%==============

$ \phi_\eta$ can be further simplified to avoid the multiplication and division of the same signals that result in noise amplification as follows:

\begin{align}
     \phi_{\eta}(k)
                        &= \lrb{\frac{u_1(k)}{F(k)}}
                                \bm{\lrb{1 - \frac{x_1(k)}{u_1(k-1)}} u_2(k-1) \phi_1^T(k-1) \\
                                     \lrb{1 - \frac{x_1(k)}{u_1(k-1)}}   F(k-1) \phi_1^T(k-1)     \\
                                        \eta(k) F(k-1) \phi_1^T(k)
                                                }
\end{align}

\subsection{Identification Model}
The parametric model presented previously can be converted to a regression form with minimal noise amplification as follows:
let
\begin{align}
        \phi_{\eta r}(k) &= \bm{\lrb{1 - \frac{x_1(k)}{u_1(k-1)}} u_2(k-1) \phi_1^T(k-1) \\
                                     \lrb{1 - \frac{x_1(k)}{u_1(k-1)}}   F(k-1) \phi_1^T(k-1)     \\
                                        \eta(k) F(k-1) \phi_1^T(k)
                                                }\\
        \phi_{\Gamma r}(k) &= \lrb{ \frac{u_2(k-1)}{F(k-1)}} \phi_2(k-1)
\end{align}
Thus, rewriting equation (\ref{eqn::eta_parm}):
\begin{align*}
        \lrb{\frac{F(k)}{u_1(k)}} \eta(k+1) &= \eta(k) \lrb{\frac{F(k-1)}{u_1(k-1)}}
                    -  \phi^T_{\eta r}(k) \theta_{\eta}  + \phi_{\Gamma r}^T(k) \theta_{\Gamma}
\end{align*}
Let
\begin{align}
        y_{NO_x}(k) &= \lrb{\frac{F(k)}{u_1(k)}} \eta(k+1) - \eta(k) \lrb{\frac{F(k-1)}{u_1(k-1)}} \\
        %====
        \phi_{NO_x}(k) &= \bm{-  \phi_{\eta r}(k) \\
                           \phi_{\Gamma r}(k)}
                        = \bm{\lrb{\frac{x_1(k)}{u_1(k-1)} - 1} u_2(k-1)  \phi^T(k-1) \\
                                     \lrb{\frac{x_1(k)}{u_1(k-1)} - 1}   F(k-1)  \phi^T(k-1)     \\
                                        - \eta(k) F(k-1)  \phi^T(k) \\
                                                \lrb{ \frac{u_2(k-1)}{F(k-1)}}  \phi(k-1)}
        \label{eqn::phi_NOx}\\
        %===
         \theta_{NO_x} &= \bm{ \theta_{\eta} \\
                                   \theta_{\Gamma}}
                            = \bm{\nu_u t_s  \theta_{ads}\\
                                        t_s  \theta_{od} \\
                                        t_s  \theta_{scr} \\
                                        t_s k_{s2v} \nu_u \Gamma \tau_0  \theta_{scr/ads}}
\end{align}
Thus, we have the regression model,
\begin{align}
        y_{NO_x}(k) &=  \phi_{NO_x}^T  \theta_{NO_x}
        \label{eqn::ident_form}
\end{align}

\subsection{Physical Interpretation of the Model}
The model predicts the tailpipe $NO_x$ from the engine-out $NO_x$ and the previous change in $NO_x$ concentration due to
reduction. The change in adsorbed ammonia concentration due to oxidation, SCR reaction and change in void concentration
is implicitly considered.
\begin{multline}
        \underbrace{x(k+1)}_{\bm{\text{tailpipe (outlet) $NO_x$}\\ \text{at next time step}}}
        % ===
        = \underbrace{u_1(k)}_{\bm{\text{engine-out (inlet) $NO_x$ }\\
                                   \text{concentration at current time step}}}
                - \underbrace{ \eta(k) \lrb{\frac{u_1(k)}{F(k)}} \lrb{\frac{F(k-1)}{u_1(k-1)}}}
                                _{\bm{\text{$NO_x$ reduction }\\
                                        \text{from previous step}\\
                                        \text{corrected for current flow-rate}}}
                                \\+ \underbrace{ \lrb{\frac{u_1(k)}{F(k)}}
                                \bm{\lrb{1 - \frac{x_1(k)}{u_1(k-1)}} u_2(k-1) \phi_1^T(k-1) \\
                                     \lrb{1 - \frac{x_1(k)}{u_1(k-1)}}   F(k-1) \phi_1^T(k-1)     \\
                                         \eta(k) F(k-1) \phi_1^T(k) \\
                                         - \lrb{ \frac{u_2(k-1)}{F(k-1)}} \phi_2(k-1)}
                                \bm{\nu_u t_s \theta_{ads}\\
                                        t_s \theta_{od} \\
                                        t_s  \theta_{scr} \\
                                        t_s k_{s2v} \nu_u \Gamma \tau_0 \theta_{scr/ads}}
                                }_{
                                        \bm{\text{change in $NO_x$ reduction due to} \\
                                                \text{change in adsorbed ammonia from} \\
                                                \text{oxidation, urea-injection and}\\
                                                \text{change in void concentration}
                                        }
                                }
\end{multline}

