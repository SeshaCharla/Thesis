\subsection{Parametrizing the $\eta$ dynamics}
The equation (\ref{eqn::nox_govern}) can be written in the following structure, with $g_i$'s denoting the corresponding expressions in each of the terms.
\begin{align}
        \eta(k+1) &= \eta(k) \lrf{ g_{\eta} - g_{ads} - g_{od} - g_{scr}} + g_\Gamma
\end{align}
The individual terms are parametrized based on the following relevant set of assumptions:
\begin{itemize}
        \item[$A3.$] The rate constant is linear or (quadratic) for a given operating range of temperature.
        %===
        \item[$A4.$] $\Gamma$ is a constant for a given operating range of temperature and only changes with aging.
        % ===
        %===
        \item[$A5.$] The model for $\con{NH_3}^{in}$ based on urea injection is given by equation (\ref{eqn::urea_inj}), i.e., $\con{NH_3}^{in}$ depends only on the flow-rate and urea injection but not the temperature (as the urea is preheated).
        \begin{align}
                \con{NH_3}^{in}(k) &= \nu_u \times \frac{u_2(k)}{F(k)}
                \label{eqn::urea_mdl}
        \end{align}
        \item[$A6.$] The model for residence time is given by equation (\ref{eqn::residence_time_mdl}), i.e., the
                residence time depends only on the flow-rate and effect of change in density (due to change in
                temperature) is negligible.
        \begin{align}
                \tau(k) &= \frac{V \rho_0}{F(k)} = \frac{\tau_0}{F(k)}
                \label{eqn::residence_time_mdl}
        \end{align}
\end{itemize}

Further, for numerical stability \cite{press2003numerical}, the given temperature range is mapped to $[-1,1]$ and Chebyshev polynomial basis functions \cite{trefethen2019approximation} are used instead of standard polynomials. We have the first and second order Chebyshev basis as follows:
\begin{align}
    \phi_1 (k) &= \bm{ \frac{T-T_0}{T_r} & 1}
    \label{eqn::phi_1} \\
    \phi_2 (k) &= \bm{2\lr{\frac{T-T_0}{T_r}}^2-1 & \frac{T-T_0}{T_r} & 1}
    \label{eqn::phi_2} \\
    %===
    \text{where } \quad T_0 &= \frac{T_{max} + T_{min}}{2}, \quad T_r = \frac{T_{max} - T_{min}}{2} \notag
\end{align}
The Chebyshev basis order is chosen based on the expected nonlinearity in the temperature dependence of the rate
constants. A first order basis is used when terms contain only rate constants, which are expected to vary nearly
linearly with temperature over the operating range. A second order basis is used when both rate constants and the total
surface concentration of voids ($\Gamma$) appear, since this dependence is expected to include an inflection point.
Using the above assumptions, we have the parametrization of individual terms:

\begin{enumerate}
\item \begin{align*}
        g_\Gamma &= t_s k_{s2v} \times \lrf{\Gamma(k-1) \tau(k) u_1(k) \con{NH_3}^{in}(k-1)} \times k_{scr/ads}(k-1)\\
                &= t_s k_{s2v} \Gamma \times \frac{\tau_0}{F(k)} \times u_1 (k) \times \nu_u \frac{u_2(k-1)}{F(k-1)}\times \pmb \phi^T(k-1) \pmb \theta_{scr/ads}
                \qquad \lrb{\because \ref{eqn::gamma_mdl}, \ref{eqn::residence_time_mdl}, \ref{eqn::k_mdl}, \ref{eqn::urea_mdl}}
\end{align*}
\begin{align}
        g_\Gamma &= \lrf{ \frac{u_1(k)}{F(k)} \frac{u_2(k-1)}{F(k-1)} \pmb \phi^T(k-1) } \times \lrf{ t_s k_{s2v} \nu_u \Gamma \tau_0 \pmb \theta_{scr/ads} }
\end{align}

\item \begin{align*}
        g_{\eta} &= \lrb{\frac{\tau(k)}{\tau(k-1)}}
                                \lrb{\frac{u_1(k)}{u_1(k-1)}}
                = \lrb{\frac{\frac{\tau_0}{F(k)}}{\frac{\tau_0}{F(k-1)}}} \lrb{\frac{u_1(k)}{u_1(k-1)}}
                \qquad \bm{\because \ref{eqn::residence_time_mdl}}
\end{align*}
\begin{align}
        g_{\eta} &= \lrb{\frac{u_1(k)}{F(k)}} \lrb{\frac{F(k-1)}{u_1(k-1)}}
\end{align}

\item \begin{align*}
        g_{ads} &= \lrb{\frac{\tau(k)}{\tau(k-1)}}
                                \lrb{\frac{u_1(k)}{u_1(k-1)}}
                t_s k_{ads}(k-1) \con{NH_3}^{in}(k-1)\\
        % ===
        &=\lrb{\frac{F(k-1)}{F(k)}}\lrb{\frac{u_1(k)}{u_1(k-1)}}
                t_s \times \phi_1^T(k-1) \theta_{ads} \times \nu_u \frac{u_2(t-1)}{F(t-1)}
                \qquad
                \bm{\because \ref{eqn::residence_time_mdl}, \ref{eqn::k_mdl}, \ref{eqn::urea_mdl}}
\end{align*}
\begin{align}
       g_{ads} &=  \lrf{\lrb{\frac{u_1(k)}{F(k)}}\lrb{\frac{u_2(k-1)}{u_1(k-1)}} \phi_1^T(k-1) }
                \times \lrf{\nu_u t_s \theta_{ads}}
\end{align}

\item \begin{align*}
        g_{od} &= \lrb{\frac{\tau(k)}{\tau(k-1)}}
                                \lrb{\frac{u_1(k)}{u_1(k-1)}}
                t_s k_{od}(k-1)\\
                &= \lrb{\frac{F(k-1)}{F(k)}}
                                \lrb{\frac{u_1(k)}{u_1(k-1)}}
                t_s \phi_1^T(k-1) \theta_{od}
                \qquad \bm{\because \ref{eqn::residence_time_mdl}, \ref{eqn::phi_1}}
\end{align*}
\begin{align}
        g_{od} &= \lrf{ \lrb{\frac{u_1(k)}{F(k)}} \lrb{\frac{F(k-1)}{u_1(k-1)}} \phi_1^T(k-1) }
                \times \lrf{t_s \theta_{od}}
\end{align}

\item \begin{align*}
        g_{scr} &= \lrb{\frac{\tau(k)}{\tau(k-1)}}
                t_s k_{scr}(k) u_1(k)\\
                &= \lrb{\frac{F(k-1)}{F(k)}} u_1(k) \phi_1^T(k) \theta_{scr} t_s
                \qquad \bm{\because \ref{eqn::residence_time_mdl}, \ref{eqn::k_mdl}}
\end{align*}
\begin{align}
        g_{scr} &= \lrf{\lrb{\frac{u_1(k)}{F(k)}} F(k-1) \phi_1^T(k)}
                \times \lrf{ t_s \theta_{scr} }
\end{align}

\end{enumerate}

Thus, we have the parametric form of the $NO_x$ reduction dynamics:

\begin{align}
        \eta(k+1) &= \eta(k) \lrb{\frac{u_1(k)}{F(k)}} \lrb{\frac{F(k-1)}{u_1(k-1)}}
                    - \phi^T_{\eta}(k) \theta_{\eta}  +  \phi_{\Gamma}^T(k)  \theta_{\Gamma}
        \label{eqn::eta_parm}
\end{align}
where,

\begin{minipage}{0.49\textwidth}
        \begin{align}
                \phi_{\eta}(k) &= \eta(k) \lrb{\frac{u_1(k)}{F(k)}}
                                \bm{\lrb{\frac{u_2(k-1)}{u_1(k-1)}}  \phi_1^T(k-1) \\
                                         \lrb{\frac{F(k-1)}{u_1(k-1)}} \phi_1^T(k-1)     \\
                                                 F(k-1) \phi_1^T(k)
                                                }
        \end{align}
\end{minipage}
\begin{minipage}{0.49\textwidth}
        \begin{align}
        \theta_{\eta} &= \bm{\nu_u t_s  \theta_{ads}\\
                                        t_s  \theta_{od} \\
                                        t_s  \theta_{scr}}
        \end{align}
\end{minipage}

\begin{minipage}{0.49\textwidth}
        \begin{align}
                \phi_{\Gamma} (k) &= \lrb{\frac{u_1(k)}{F(k)} } \lrb{ \frac{u_2(k-1)}{F(k-1)}} \phi_2(k-1)
        \end{align}
\end{minipage}
\begin{minipage}{0.49\textwidth}
        \begin{align}
                \theta_{\Gamma} (k) &= \bm{ t_s k_{s2v} \nu_u \Gamma \tau_0 \theta_{scr/ads} }
        \end{align}
\end{minipage}

\bigskip

Hence, we have the $NO_x$ dynamics:
\begin{align}
        x(k+1) &= u_1(k) - \eta(k) \lrb{\frac{u_1(k)}{F(k)}} \lrb{\frac{F(k-1)}{u_1(k-1)}}
                        + \phi^T_{\eta}(k) \theta_{\eta}  - \phi_{\Gamma}^T(k) \theta_{\Gamma}
        \label{eqn::nox_sim_mdl}\\
%===
        x(k+1) &= u_1(k) - \lrf{1 - \frac{x_1(k)}{u_1(k-1)}} \lrb{\frac{u_1(k)}{F(k)}} F(k-1)
                        + \phi^T_{\eta}(k) \theta_{\eta}  - \phi_{\Gamma}^T(k) \theta_{\Gamma}
\end{align}

%==============

$ \phi_\eta$ can be further simplified to avoid the multiplication and division of the same signals that result in noise amplification as follows:

\begin{align}
     \phi_{\eta}(k)
                        &= \lrb{\frac{u_1(k)}{F(k)}}
                                \bm{\lrb{1 - \frac{x_1(k)}{u_1(k-1)}} u_2(k-1) \phi_1^T(k-1) \\
                                     \lrb{1 - \frac{x_1(k)}{u_1(k-1)}}   F(k-1) \phi_1^T(k-1)     \\
                                        \eta(k) F(k-1) \phi_1^T(k)
                                                }
\end{align}
