\section{Switching Condition Parametrization for the $NO_x$ Reduction Model}
An alternative and concise derivation of the above model provides greater insight into the parametric switching conditions. We have the following equations describing the $NO_x$ reduction and ammonia adsorption in the SCR-ASC system (from \ref{eqn::NOX_process}, \ref{eqn::sigma_process}, and \ref{eqn::sat_func}):
\begin{align}
        \eta(k+1) &= \frac{u_1(k)}{F(k)}\underbrace{\lr{k_{s2v}\tau_0 k_{scr}(T(k))}}_{Q(k)} \sigma(k)\\
        \sigma(k+1)&= \sigma(k) + t_s k_{ads} \nu_0 \frac{u_2(k)}{F(k)} \lrb{\Gamma - \sigma(k)}
                        - t_s k_{od} \sigma(k)
                        - t_s k_{scr} u_1(k) \sigma(k)\\
        [NH_3]^{in}(k) &= \nu_0 \frac{u_2(k)}{F(k)}
\end{align}
Let,
\begin{align*}
        u_{1F}(k) &= \frac{u_1(k)}{F(k)} \\
        u_{2F}(k) &= \frac{u_2(k)}{F(k)}
\end{align*}
We have,
\begin{align*}
        Q(k) \sigma(k) &=  \frac{\eta(k+1)}{u_{1F}(k)} \\
\end{align*}
Multiplying both sides of the $\sigma(k+1)$ equation by $Q(k)$, we get
\begin{align*}
        \frac{Q(k)}{Q(k+1)} Q(k+1) \sigma(k+1) &= Q(k) \sigma(k) + t_s k_{ads} \nu_0 \frac{u_2(k)}{F(k)} \lrb{Q(k)\Gamma - Q(k)\sigma(k)} \\
        % ==
        & - t_s k_{od} Q(k) \sigma(k)\\
        & - t_s k_{scr} u_1(k) Q(k) \sigma(k)
\end{align*}
\itbf{Assumption 1:} $T(k) \approx T(k+1)$, which implies $Q(k) \approx Q(k+1)$.
\begin{align*}
        \frac{\eta(k+2)}{u_{1F}(k+1)} &= \frac{\eta(k+1)}{u_{1F}(k)} + t_s k_{ads} \nu_0 u_{2F}(k) \lrb{Q(k) \Gamma - \frac{\eta(k+1)}{u_{1F}(k)}} \\
        % ==
        & - t_s k_{od} \frac{\eta(k+1)}{u_{1F}(k)} \\
        & - t_s k_{scr} u_1(k) \frac{\eta(k+1)}{u_{1F}(k)}
\end{align*}
\itbf{Definition:} We have, the \itbf{flow-scaled fractional $NO_x$ reduction} defined as,
\begin{align*}
        \bar \eta_F (k) = \frac{\eta(k)}{u_{1F}(k-1)} = \frac{F(k-1)}{u_1(k-1)} \eta(k)
\end{align*}
\begin{align*}
        \bar \eta_F (k+2) = \bar \eta_F (k+1) &+ t_s k_{ads}(k) \nu_0 u_{2F}(k) \lrb{Q(k) \Gamma - \bar \eta_F (k+1)} \\
        % ===
        & - t_s k_{od}(k) \bar \eta_F (k+1) \\
        & - t_s k_{scr}(k) u_1(k) \bar \eta_F (k+1)
\end{align*}
Moving the time back by one step, we have
\begin{align*}
        \bar \eta_F (k+1) = \bar \eta_F (k) &+ t_s \nu_0 k_{ads}(k-1)  u_{2F}(k-1) \lrb{Q(k-1) \Gamma - \bar \eta_F (k)} \\
        % ===
        & - t_s k_{od}(k-1) \bar \eta_F (k) \\
        & - t_s k_{scr}(k-1) u_1(k-1) \bar \eta_F (k)
\end{align*}
\begin{align*}
        \bar \eta_F (k+1) = \bar \eta_F (k) &+ t_s k_{ads}(k-1) \nu_0 u_{2F}(k-1) Q(k-1) \Gamma(k-1) \\
        & - t_s k_{ads}(k-1) \nu_0 u_{2F}(k-1) \bar \eta_F (k) \\
        & - t_s k_{od}(k-1) \bar \eta_F (k) \\
        & - t_s k_{scr}(k-1) u_1(k-1) \bar \eta_F (k)
\end{align*}
Expanding $Q(k)$:
\begin{align*}
        \bar \eta_F (k+1) = \bar \eta_F (k) &+ t_s  k_{s2v}\tau_0 \nu_0 k_{ads}(k-1) k_{scr}(k-1) u_{2F}(k-1) \Gamma(k-1) \\
        & - t_s \nu_0 u_{2F}(k-1) k_{ads}(k-1) \bar \eta_F (k) \\
        & - t_s k_{od}(k-1) \bar \eta_F (k) \\
        & - t_s k_{scr}(k-1) u_1(k-1) \bar \eta_F (k)
\end{align*}
Parametrizing the above equation with a Chebyshev basis polynomial temperature model for the rate constants (linear) and the product of rate constants and $\Gamma$ (quadratic), we have the following:
\begin{align}
        \bar \eta_F (k+1) = \bar \eta_F (k) &+ u_{2F}(k-1)\phi_2(k-1) \theta_\Gamma \label{eqn::t2}\\
        & - u_{2F}(k-1) \bar \eta_F (k) \phi_1(k-1) \theta_{\eta_{ads}} \label{eqn::t3}\\
        & - \bar \eta_F (k) \phi_1(k-1) \theta_{\eta_{od}} \label{eqn::t4}\\
        & - u_1(k-1) \bar \eta_F (k) \phi_1(k-1) \theta_{\eta_{scr}} \label{eqn::t5}\\
        \text{where,} \qquad  & \notag\\
        \phi_2(k) &= \bm{2\lr{\frac{T-T_0}{T_r}}^2-1 & \frac{T-T_0}{T_r} & 1} \notag\\
        \phi_1(k) &= \bm{\frac{T-T_0}{T_r} & 1} \notag\\
        T_r &= \frac{T_{max} - T_{min}}{2} \quad \text{and} \quad
        T_0 = \frac{T_{max} + T_{min}}{2} \notag
\end{align}
Terms \ref{eqn::t2} and \ref{eqn::t3} on the right-hand side form the adsorption dynamics, term \ref{eqn::t4}
forms the oxidation dynamics, and term \ref{eqn::t5} forms the SCR dynamics. The adsorption dynamics are constrained
and switch based on the surface coverage $\sigma$ as follows:
\begin{align*}
        \Gamma \geq \sigma(k) \geq 0
\end{align*}
\begin{align*}
        u_{2F}(k-1)\phi_2(k-1) \theta_\Gamma - u_{2F}(k-1) \bar \eta_F (k) \phi_1(k-1) \theta_{\eta_{ads}} \geq 0
\end{align*}
\begin{align*}
        \implies  \bm{- \phi_2^T(k-1) &  \bar \eta_F(k) \phi_1^T(k-1)} \bm{\theta_\Gamma \\ \theta_{\eta_{ads}}} \leq 0
\end{align*}
\begin{align}
        \implies  \underbrace{\bm{ \bar \eta_F(k) \phi_1^T(k-1)
                        & \pmb 0 & \pmb 0 & - \phi_2^T(k-1) }}_{\phi_g(k)^T} \underbrace{\bm{\theta_{\eta_{ads}} \\ \theta_{\eta_{od}} \\ \theta_{\eta_{scr}} \\ \theta_\Gamma }}_{\theta_{NO_x}} \leq 0
        \label{eqn::gaurd_condition}
\end{align}
The above condition is the condition for normal operation of the catalyst. When this condition is violated, the
adsorption dynamics switch to zero, and the dynamics switch to saturated catalyst mode. Thus, we end up with a switching condition that is parametrized linearly in terms of the model parameters, resulting in a \itbf{self-excited threshold nonlinear autoregressive with exogenous input} \cite{hansen1991institutional} model structure.
% =========================================

Also, $NO_x$ reduction in a given time-step is always positive, i.e.,
\begin{align*}
        \bar \eta_F (k) \geq 0 \qquad \forall \quad k
\end{align*}
This condition is implicitly satisfied.
% =========================================
Incorporating the switching condition and the saturated and unsaturated $NO_x$ reduction dynamics, we have the state diagram for the $NO_x$ reduction dynamics as shown in Figure \ref{fig::no_x_swtiching}.

\begin{figure}[!ht]
        \centering
        \includegraphics[width=\textwidth]{./figs/4-NOx_mdl/hybrid_model.png}
        \caption{State Diagram for the $NO_x$ Reduction Model with Switching Conditions}
        \label{fig::no_x_swtiching}
\end{figure}
