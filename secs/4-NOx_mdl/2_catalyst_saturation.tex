\section{Catalyst Saturation}

The model developed up until now considers the situation where the urea-dosing is such that the catalyst is neither
saturated nor empty. The ideal case which is also an assumption in the linearized CSTR model. The dynamic model for
catalyst's surface concentration, when considering the saturation and zero concentration limit will be
($\gamma_{proc}(k)$ is defined in \ref{eqn::gamma_proc}):
%=====
\begin{align}
        \sigma^{ub}(k) &= \sigma (k-1) \gamma_{proc}(k-1) + t_s \Gamma k_{ads}(k-1) \con{NH_3}^{in}(k-1)
        \label{eqn::unbounded_sigma_mdl}\\
        \sigma(k) &= \begin{cases}
                                \sigma^{ub}(k) & \text{if }  0 \leq \sigma^{ub}(k) \leq \Gamma\\
                                \Gamma         & \text{if }  \sigma^{ub}(k) > \Gamma \\
                                0              & \text{if }  \sigma^{ub}(k) < 0
                        \end{cases}
        \label{eqn::actual_sigma_mdl}
\end{align}

Thus substituting equation \ref{eqn::sigma_elim} into above equations \ref{eqn::unbounded_sigma_mdl} and \ref{eqn::actual_sigma_mdl}:
%===
\begin{align}
        &\sigma^{ub}(k) = \lrf{\frac{\eta(k)}{\tau(k-1) k_{s2v} k_{scr}(k-1) u_1(k-1)}} \gamma_{proc}(k-1) + t_s \Gamma k_{ads}(k-1) \con{NH_3}^{in}(k-1)\\
        % ====
        &\frac{\eta(k+1)}{\tau(k) k_{s2v} k_{scr}(k) u_1(k)} =
        \begin{cases}
                                \sigma^{ub}(k)     & \text{if }  0 \leq \sigma^{ub}(k) \leq \Gamma\\
                                \Gamma             & \text{if }  \sigma^{ub}(k) > \Gamma \\
                                0                  & \text{if }  \sigma^{ub}(k) < 0
        \end{cases} \\
        % =====
        &\implies \eta(k+1) =
        \begin{cases}
        \lrf{\tau(k) k_{s2v} k_{scr}(k) u_1(k)}\sigma^{ub}(k)         & \text{if }
                0 \leq \sigma^{ub}(k) \leq \Gamma\\
        \lrf{\tau(k) k_{s2v} k_{scr}(k) u_1(k)} \Gamma                & \text{if }
                \sigma^{ub}(k) > \Gamma \\
        0                                                             & \text{if }
                \sigma^{ub}(k) < 0
        \end{cases}
        \label{eqn::eta_cases}
\end{align}
%====
From, the derivation and the parametrization in the last section, specifically, equations \ref{eqn::eta_parm} and
\ref{eqn::phi_NOx}:
\begin{align}
        \lrf{\tau(k) k_{s2v} k_{scr}(k) u_1(k)}\sigma^{ub}(k) &=
        \eta(k) \lrb{\frac{u_1(k)}{F(k)}} \lrb{\frac{F(k-1)}{u_1(k-1)}}
        +\lrb{\frac{u_1(k)}{F(k)}} \pmb \phi_{NO_x}^T \pmb \theta_{NO_x}
        \label{eqn::sigma_ub_param}
\end{align}
%===
Similarly, the second case can be parametrized using the same set of approximations ($Ai$'s) and models of physical quantities (\ref{eqn::residence_time_mdl}, \ref{eqn::k_mdl}),
\begin{align*}
        \lrf{\tau(k) k_{s2v} k_{scr}(k) u_1(k)} \Gamma  &= \frac{\tau_0}{F(k)} \times k_{s2v} \times \pmb \phi(k)^T \pmb \theta_{scr} \times u_1(k) \times \Gamma\\
        % ==
        &= \lrf{\lrb{\frac{u_1(k)}{F(k)}} \pmb \phi^T(k) } \lrf{\Gamma \tau_0 k_{s2v} \pmb \theta_{scr}}
\end{align*}
\begin{align}
        Let, \qquad  \Gamma \tau_0 k_{s2v} \pmb \theta_{scr} &= \pmb \theta_{\Gamma scr}\\
        \lrf{\tau(k) k_{s2v} k_{scr}(k) u_1(k)} \Gamma  &=
        \lrb{\frac{u_1(k)}{F(k)}} \pmb \phi^T(k)  \pmb \theta_{\Gamma scr}
        \label{eqn::gamma_scr}
\end{align}
%===
The conditions for case switching can also be parametrized to the same expressions as (\ref{eqn::sigma_ub_param}) and (\ref{eqn::gamma_scr}) by multiplying both sides of the inequalities by $\lrf{\tau(k) k_{s2v} k_{scr}(k) u_1(k)} (>0)$ which is always positive.

Let,
\begin{align}
        f_{\sigma}(k) &= \eta(k) \lrb{\frac{u_1(k)}{F(k)}} \lrb{\frac{F(k-1)}{u_1(k-1)}}
        +\lrb{\frac{u_1(k)}{F(k)}} \pmb \phi_{NO_x}^T \pmb \theta_{NO_x}\\
        %===
        f_{\Gamma}(k) &= \lrb{\frac{u_1(k)}{F(k)}} \pmb \phi^T(k)  \pmb \theta_{\Gamma scr}
\end{align}

Substituting the above parametrization into equation (\ref{eqn::eta_cases}):
%===
\begin{align}
        \eta(k+1) &=
        \begin{cases}
                f_{\sigma}(k) & \text{if } 0 \leq f_{\sigma}(k) \leq f_{\Gamma}(k)\\
                f_{\Gamma}(k) & \text{if } f_{\sigma}(k) > f_{\Gamma}(k) \\
                0             & \text{if } f_{\sigma}(k) < 0
        \end{cases}
        \label{eqn::eta_cases}
\end{align}
The above cases can be written in min-max form as follows:
\begin{align}
        \eta\lr{k + 1} &= \max \lrf{ 0, \min \lrf{f_{\sigma}(k), f_{\Gamma}(k)}}
        \label{eqn::eta_max_min}
\end{align}
Thus, the $NO_x$ reduction dynamics are bounded by:
\begin{align}
        0 \leq \eta(k+1) \leq f_\Gamma(k)
\end{align}
% ====
The above equation can be interpreted as follows: $\eta$ at any given time step is bounded by the change in concentration due to $NO_x$ reduction when the catalyst is saturated $\lr{\eta_{sat}}$ under the same conditions of flow-rate and inlet $NO_x$ concentration. This relationship can be used to find the optimal values of the parameter $\pmb \theta_{\Gamma scr}$ by solving the following constrained linear programming problem:
\begin{align*}
        \text{Minimize:}& \qquad    \lrb{ \sum_{k = 0}^{N-2} \lrb{\frac{u_1(k)}{F(k)}} \phi^T(k) } \theta_{\Gamma scr}\\
        \text{Subject to:}& \qquad
                \bm{\lrb{\frac{u_1(0)}{F(0)}} \phi^T(0) \\
                    \lrb{\frac{u_1(1)}{F(1)}} \phi^T(1) \\
                    \vdots \\
                    \lrb{\frac{u_1(N-2)}{F(N-2)}} \phi^T(N-2) \\
                }
                \theta_{\Gamma scr}
                \geq
                \bm{\eta(1) \\
                    \eta(2)\\
                    \vdots \\
                    \eta(N-1)}
\end{align*}
\begin{align}
        \label{eqn::optimization_prob}
\end{align}
Effectively we are minimizing the area under the curve of the parametrized saturated system dynamics under the
constraints of the actual system dynamics. This results in the tightly upper bound curve that describes the dynamics of
the system which is always saturated.

The upper bound of the actual change in concentration due to $NO_x$ reduction $(\eta(k))$ is the inlet concentration of $NO_x$ $(u_1(k-1))$.

%===
Writing the equation (\ref{eqn::eta_max_min}) in terms of the tail-pipe $NO_x$ concentration,
\begin{align}
        x_1(k+1) &= \max \lrf{0, u_1(k) - \max \lrf{ 0, \min \lrf{f_{\sigma}(k), f_{\Gamma}(k)}}}
\end{align}
The above equation also accounts for the case when more $NO_x$ reduction is possible but the inlet $NO_x$ is lower.
%===
Thus, the tailpipe $NO_x$ dynamics are bounded by:
\begin{align}
        \max \lrf{0, u_1(k) - f_{\Gamma}(k)} \leq x_1(k+1) \leq u_1(k)
\end{align}
%====

\subsection{Saturated Catalyst Model Parameter Estimation}

The linear programming problem is solved using a quadratic polynomial approximation for $\pmb \phi \pmb \theta_{\Gamma
scr}$. Both $k_{scr/ads}$ and $\Gamma$ are assumed to be monotonic functions of temperature but in opposing directions.
Thus, a linear model for the product will not capture the change in monotonicity of the product that happens at a
particular temperature. After a systematic analysis of various polynomial orders and temperature partitions, a quadratic
model with two partitions was found to be the best approximation with minimum prediction error. The parameter estimates
of the saturated model are tabulated below.

\begin{table}[!ht]
        \centering
        \caption{Parameter estimates of the saturated catalyst model for the SCR-ASC system}
        \begin{tabular}{l l c c c c}
                \hline \hline
                Age & Test & Temp. Zone &
                $ \theta_{sat}[2]$ &
                $ \theta_{sat}[1]$ &
                $ \theta_{sat}[0]$ \\ \hline \hline
                % ============================================
                Degreened & RMC & high & -0.06 & 1.43 & 31.19 \\
                % =============================================
                Aged & RMC & high & -0.07 & 1.77 & 27.82\\      \hline
                % =============================================
                Degreened & hot-FTP & high & -0.49 & 2.94 & 40.94 \\
                % =============================================
                Aged & hot-FTP & high & -0.54 & 3.40 & 39.59 \\         \hline
                % =============================================
                Degreened & cold-FTP & high & -0.19 & 0.97 & 40.69 \\
                % =============================================
                Aged & cold-FTP & high & -0.28 & 1.82 & 38.97\\         \hline
                % =============================================
                Degreened & cold-FTP & low & 0.26 &  5.97 & 45.00 \\
                % =============================================
                Aged & cold-FTP & low & 0.39 & 8.25 & 50.63 \\
                % =============================================
                \hline \hline
                % =============================================
        \end{tabular}

        $ \theta_{\Gamma scr} [i]$ is the coefficient of $T^i$ in the polynomial.
\end{table}

The parameter values are directionally different for aged and degreened catalysts. Their variance provides better
discernibility of catalyst aging based on these parameter values. The responses of the saturated system to the inputs
from the data are shown in Figures~\ref{fig::eta_bounds_cftp}, \ref{fig::eta_bounds_hftp}, and \ref{fig::eta_bounds_rmc}.

\begin{figure}[!ht]
        \begin{minipage}{0.49\textwidth}
                \begin{figure}[H]
                        \centering
                        \includegraphics[width=\textwidth]{./figs/4-NOx_mdl/2_figs/bounded_eta_plots/eta_bounds_dg_cftp.png}
                \end{figure}
        \end{minipage}
        \begin{minipage}{0.49\textwidth}
                \begin{figure}[H]
                        \centering
                        \includegraphics[width=\textwidth]{./figs/4-NOx_mdl/2_figs/bounded_eta_plots/eta_bounds_aged_cftp.png}
                \end{figure}
        \end{minipage}
        \caption{Saturated system response for cold FTP data}
        \label{fig::eta_bounds_cftp}
\end{figure}

The plots show the response of the system under continuous catalyst saturation with ammonia. That is, irrespective of
the urea dosing, the catalyst remains saturated in this model. This is essentially the maximum
possible $NO_x$ reduction at current flow rate and temperature conditions. One observation of note that is consistent
with intuition is that the maximum $NO_x$ reduction shown above is inversely related to the flow rate of the system.
When the flow rate is low, the residence time is high and there is more time for the $NO_x$ to be reduced at the same
reaction rate, resulting in higher $NO_x$ reduction.

\begin{figure}[!ht]
        \begin{minipage}{0.49\textwidth}
                \begin{figure}[H]
                        \centering
                        \includegraphics[width=\textwidth]{./figs/4-NOx_mdl/2_figs/bounded_eta_plots/eta_bounds_dg_hftp.png}
                \end{figure}
        \end{minipage}
        \begin{minipage}{0.49\textwidth}
                \begin{figure}[H]
                        \centering
                        \includegraphics[width=\textwidth]{./figs/4-NOx_mdl/2_figs/bounded_eta_plots/eta_bounds_aged_hftp.png}
                \end{figure}
        \end{minipage}
        \caption{Saturated system response for hot FTP data}
        \label{fig::eta_bounds_hftp}
\end{figure}

\begin{figure}[!ht]
        \begin{minipage}{0.49\textwidth}
                \begin{figure}[H]
                        \centering
                        \includegraphics[width=\textwidth]{./figs/4-NOx_mdl/2_figs/bounded_eta_plots/eta_bounds_dg_rmc.png}
                \end{figure}
        \end{minipage}
        \begin{minipage}{0.49\textwidth}
                \begin{figure}[H]
                        \centering
                        \includegraphics[width=\textwidth]{./figs/4-NOx_mdl/2_figs/bounded_eta_plots/eta_bounds_aged_rmc.png}
                \end{figure}
        \end{minipage}
        \caption{Saturated system response for RMC data}
        \label{fig::eta_bounds_rmc}
\end{figure}

When this quantity is normalized with respect to the flow-rate and inlet concentration, we obtain a quantity that is
purely a function of temperature and catalyst aging. This is verified in the next section.

% \input{\froot/secs/14_subs/prelim_aging.tex}
