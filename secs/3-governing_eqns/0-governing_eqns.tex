\chapter{SCR-ASC Reacting Flow Dynamics: State Space Model}

The sensor limitations and identifiability requirements outlined in previous chapters necessitate the development of a dynamic model for the SCR-ASC system. This model must possess parameters that are identifiable from sensor data and exhibit dynamics within the frequency range captured by the sensors. Accordingly, the model's validity region in the $\tau-t_s$ (residence time versus sampling time) plane should encompass the decimated test-cell and truck data residence times [Figure]. Therefore, this chapter proposes a discrete nonlinear state-space model for the SCR-ASC dynamics that alternates between catalyst saturation and unsaturation states.

\begin{figure}[!ht]
        \centering
        \includegraphics[width=0.8\textwidth]{./figs/3-governing_eqns/mdl_valid.png}
        \caption{Validity region of the discrete nonlinear switched model for SCR-ASC dynamics}
\end{figure}

The model development employs a control volume approach, formulating molar conservation equations for the reacting species between the inlet and outlet. The primary innovation involves integrating these conservation equations over the residence time and summing them telescopically according to the number of residence times occurring within a single sampling period. This methodology explicitly incorporates both residence time and sampling time into the model.

The chapter begins by detailing the reactions occurring within the SCR-ASC system and outlining the assumptions adopted to reduce model order while maintaining identifiability, causality, and parsimony. Dynamic equations for each reacting species are then derived using this approach, with all assumptions explicitly stated and justified. The resulting nonlinear state-space model for the SCR-ASC system is subsequently presented. The transformation of this nonlinear model into an identifiable input-output model, along with the parameter estimation algorithm and model validation, is addressed in the following chapter.

% =============================================================

\section{SCR/ASC Reactions}
The following lists all the reactions that take place inside the SCR-ASC chamber.

\begin{align*}
    NH_2 - CO - NH_2 (liquid) &\longrightarrow NH_2 - CO - NH_2^* + x H_2 O
                & &[\text{AdBlue evaporation}] \\
    NH_2 - CO - NH_2^*  &\longrightarrow  HNCO + NH_3
                & &[\text{Urea decomposition}] \\
    HNCO + H_2O &\longrightarrow NH_3 + CO_2
                & &[\text{Isocyanic acid hydrolysis}] \\
    %===
    NH_3 + \Theta_{free} &\leftrightharpoons NH_3(ads)
                         & &[\text{Ammonia Adsorption/Desorption}]\\
    %===
    4 NH_3 (ads) + 4 NO + O_2 &\longrightarrow 4 N_2 + 6 H_2O
                              & &[\text{Standard SCR reaction}]\\
    %===
    2 NH_3 (ads) +  NO + NO_2 &\longrightarrow 2 N_2 + 3 H_2O
                              & &[\text{Fast SCR reaction}]\\
    %===
    4 NH_3 (ads) + 3NO_2 &\longrightarrow 3.5 N_2 + 6 H_2O
                              & &[\text{Slow SCR reaction}]\\
    %===
    4 NH_3 + 3 O_2 &\longrightarrow 2 N_2 + 6 H_2O
                         & &[\text{AMOX with/without ASC}]\\
    4 NH_3 + 5 O_2 &\longrightarrow 4 NO + 6 H_2 O
                         & &[\text{AMOX with/without ASC}]\\
    2 NH_3 + 2 O_2 &\longrightarrow N_2O + 3 H_2O
                         & &[\text{AMOX with/without ASC}]\\
    %==
    2 NO + O_2 &\longrightarrow 2 NO_2
                        & &[\text{NO oxidation}]
\end{align*}

Note that ammonia oxidation (AMOX) happens both on the surfaces of the SCR catalyst and the ASC catalyst through a
similar mechanism, but the SCR catalyst favors $NO_x$ reduction while ASC is specifically designed for ammonia
oxidation. The Eley-Rideal reaction mechanism \cite{yuan2015diesel,hsieh2011development,nova2014urea} is
considered for interpreting the SCR reactions, where one reactant $(NO_x)$ is gaseous, and the other is adsorbed on the
catalyst surface $(NH_3)$.

Further, in order to keep the model order reasonably low, only the following three prominent reactions are considered:
\begin{enumerate}
    \item Standard SCR reaction:
    \begin{align}
        4 NH_3 ^{ads} + 4 NO + O_2 &\xrightarrow[]{k_{scr}} 4 N_2 + 6 H_2O \label{eqn::std_scr}
    \end{align}
    \item Ammonia Oxidation:
    \begin{align}
        4 NH_3^{ads} + 3 O_2 &\xrightarrow[]{k_{oxi}} 2 N_2 + 6 H_2O \label{eqn::amox}
    \end{align}
    \item Ammonia Adsorption/Desorption:
        \begin{align}
            NH_3 + \Theta_{free} &\xrightleftharpoons[k_{des}]{k_{ads}} NH_3^{ads}
            \label{eqn::ads}
        \end{align}
\end{enumerate}


\subsection{Polynomial Approximation of Temperature Model for Rate Constants}
The rate constants $(k)$ of the reactions are temperature dependent and follow the Arrhenius equation:
\begin{align*}
    k = A \exp\left(-\frac{E}{RT}\right)
\end{align*}
where $A$ is the pre-exponential factor, $E$ is the activation energy, and $R$ is the universal gas constant. The small
perturbation form of the above equation is given by:
\begin{align*}
    \delta k &= A e^{-\frac{E}{RT}} \underbrace{\lr{\frac{E}{RT^2}}}_p \delta T = k p \delta T
\end{align*}
Thus,
\begin{align*}
    k(T) \approx k(T_0) + \delta k = k(T_0) + k(T_0) p(T_0) \underbrace{\lr{T - T_0}}_{\delta T}
\end{align*}

Hence, within a certain range of temperatures, the rate constant can be assumed to be varying linearly with temperature.
Thus, the rate constant can be:
\begin{align*}
    k(T) &= mT + c \qquad  \text{for } \: T \in [T_0 - \Delta T_{max}, T_0 + \Delta T_{max}]
\end{align*}

Based on the available data, $T_0$ is chosen as $250 \lx{^o}{C}$ and $\Delta T_{max}$ such that it spans all the
available data. From linear model fitting, the validity of the linear model was found to be limited to $\pm 50
\lx{^o}{C}$.

\subsubsection{Quadratic temperature model for rate constants}
Using second-order Taylor series approximation, the expression for rate constant can be rewritten as:
\begin{align*}
    k(T) &\approx k(T_0) + k'(T_0) (T - T_0) + \frac{1}{2} k''(T_0) (T - T_0)^2\\
    \text{Let} \qquad
    &m + 2qT_0 = k'(T_0) = \lr{\frac{AE}{RT_0^2}} e^{-\frac{E}{RT_0}} \\
    &2 q = k''(T_0) = \lr{\frac{A E^2}{R^2 T_0^4} - \frac{A E}{2 R T_0^3}} e^{-\frac{E}{RT_0}} \\
    \implies k(T) &\approx k(T_0) + \lr{m + 2qT_0} (T - T_0) + q (T - T_0)^2
                   = q T^2 + m T + \underbrace{\lr{ -  q T_0^2 - mT_0  + k(T_0)}}_c
\end{align*}
Thus, we can have a quadratic approximation model for the rate constant:
\begin{align}
    k(T) &= q T^2 + m T + c \qquad \text{for } \: T \in [T_0 - \Delta T_{max}, T_0 + \Delta T_{max}]
\end{align}

%===============================================================================
\subsection{Model reduction: Lumping ASC reaction dynamics into SCR reaction dynamics}

\begin{figure}[H]
    \centering
    \includegraphics[width=0.9\textwidth]{./figs/3-governing_eqns/SCR-ASC_ModelReduction_horizontal.png}
    \caption{Schematic of the SCR-ASC reduced model}
    \label{fig:scr-asc}
\end{figure}

The ammonia oxidation (AMOX) reactions are similar on SCR and ASC catalysts. First, the gaseous ammonia adsorbs onto the
catalyst system. Then, the adsorbed ammonia oxidizes into $N_2$ or $NO_x$ based on the temperature and other conditions.
As the AMOX reactions on both catalysts are similar, it is not possible to distinguish the origin of the products
from the outlet measurements alone unless measurements between SCR and ASC sections are available.

Thus, oxidation of adsorbed ammonia on both SCR and ASC catalysts can be combined into a single reaction, lumping the
rate constants and concentrations of the products into parameters and states of a single oxidation reaction. Further
the nitrogen selectivity of that single AMOX reaction is assumed to be $100\%$. This is valid for temperatures greater
than $225 \lx{^o}{C}$ \cite{jain2023diagnostics}, i.e., the temperature range of interest for most of the test and road
conditions.

This aggregation of reactions results in errors in the rate constant estimates of all the SCR-ASC reactions.
Specifically, the rate constant estimate for the $NO_x$ reduction will be lower than the actual value as only a part of
the total adsorbed ammonia (on SCR catalyst) alone is involved in the reduction reaction.

\section{Ammonia Adsorption/Desorption Process Dynamics \label{sec::sigma_deriv}}
\begin{figure}[H]
    \centering
    \includegraphics[width=0.5\textwidth]{./figs/3-governing_eqns/plug_flow_discrete.png}
    \caption{Discrete plug-flow reactor model}
    \label{fig:plug_flow_discrete}
\end{figure}
%===
The ammonia adsorption/desorption process dynamics involve all the three above-mentioned reactions. The gaseous ammonia
that enters the catalyst chamber gets adsorbed onto the free sites on the catalyst surface at a rate proportional to the
volumetric concentration of the gaseous ammonia and the surface concentration of the free sites. The adsorbed ammonia
then either reacts with the gaseous $NO_x$ (Eley-Rideal Mechanism) releasing $N_2$ and $H_2O$ or decomposes to form
$N_2$ and $H_2O$ (Surface Decomposition). In either case, the process frees up the adsorption sites for the next cycle
of gaseous ammonia. We first derive the dynamics of the adsorption/desorption process when the catalyst remains unsaturated and then extend it to the saturated case later in Section \ref{sec::sat_cat}.
%===
\begin{align*}
    4 NH_3 ^{ads} + 4 NO + O_2 &\xrightarrow[]{k_{scr}} 4 N_2 + 6 H_2O \\
    4 NH_3^{ads} + 3 O_2 &\xrightarrow[]{k_{oxi}} 2 N_2 + 6 H_2O \\
    NH_3 + \Theta_{free} &\xrightleftharpoons[k_{des}]{k_{ads}} NH_3^{ads}
\end{align*}
Thus, the rate of ammonia adsorption on the catalyst surface can be modelled as:
\begin{align*}
    \frac{d \con{NH_3}^{ads}}{dt} &= r_{ads} - r_{des} - r_{scr} - r_{oxi}\\
    r_{ads} &= k_{ads} \con{NH_3}^{in} \lr{\Gamma - \con{NH_3}^{ads}} \qquad \lrb{\Gamma = \frac{\Theta_{free} + \Theta_{occupied}}{A_{scr}}}\\
    r_{des} &= k_{des} \con{NH_3}^{ads}\\
    r_{scr} &= k_{scr} \con{NH_3}^{ads} \con{NO_x}^{in}\\
    r_{oxi} &= k_{oxi} \con{NH_3}^{ads}\\
    \dot{\con{NH_3}}^{ads} &= \Gamma \underbrace{k_{ads} \con{NH_3}^{in}}_{\gamma_{ads}} - \con{NH_3}^{ads} \underbrace{\lr{k_{ads} \con{NH_3}^{in} + k_{des} + k_{scr} \con{NO_x}^{in} + k_{oxi}}}_{\gamma_{des}}
\end{align*}
\itbf{Note:} $\Gamma$ and $\con{NH_3}^{ads}$ are surface concentrations (in $moles/m^2$) while the rest are volumetric
concentrations (in $moles/m^3$). The rate constant units are adjusted appropriately to make the rate equations
consistent.

The units of all the rates and rate constants are tabulated below:
\begin{align*}
    r_{ads} &= \frac{moles}{m^2 \cdot s} &
    k_{ads} &= \frac{m^3}{moles \cdot s} \\
    r_{des} &= \frac{moles}{m^2 \cdot s} &
    k_{des} &= s^{-1} \\
    r_{scr} &= \frac{moles}{m^2 \cdot s} &
    k_{scr} &= \frac{m^3}{moles \cdot s} \\
    r_{oxi} &= \frac{moles}{m^2 \cdot s} &
    k_{oxi} &= s^{-1}
\end{align*}
%===
\itbf{Note:} In this section we are interested in the Surface rates where the products are assumed to linger just on the
surface of the catalyst. The surface rates of the gaseous products/reactants can be converted to volumetric rates by
using the conversion factor $A_{scr}/V$ where $A_{scr}$ is the area of the SCR catalyst and $V$ is the volume of the
catalyst chamber. This assumes that there is instantaneous mixing of the close-to-surface moles with the gaseous moles.
Thus,
\begin{align}
    k^{vol} = \underbrace{\lr{\frac{A_{scr}}{V}}}_{k_{s2v} \, (m^{-1})} k^{surf}
\end{align}
% ==============================================================================
\subsection{Molar conservation at the scale of residence time}

The molar storage on the catalyst surface changes at the end of every residence time and a "fresh" set of gaseous
reactants enters the catalyst chamber. And, within the sample time $t_s$, the volumetric concentrations of the gaseous
reactants can be considered constant. \itbf{Let there be $n$ residence times within one sample time.} It is implicitly
assumed that there are an integer number of residence times within one sample time. When that is not the case, the
resulting error is attributed to model structure error.
\begin{align}
    t_s &= n \tau
    \label{eqn::ts_2_tau}
\end{align}
%===
Noting that $\lrf{\bullet}$ denotes moles and $\lrb{\bullet}$ denotes the concentration of the species, the molar conservation for the adsorption/desorption process can be written as:
%===
\begin{align*}
    \mol{NH_3}^{ads} (k + \tau) &= \mol{NH_3}^{ads} (k) + A_{scr} \int_{0}^{\tau} \dot{\con{NH_3}}^{ads} (k) dt\\
    \mol{NH_3}^{ads} (k + 2\tau) &= \mol{NH_3}^{ads} (k + \tau) + A_{scr} \int_{0}^{\tau} \dot{\con{NH_3}}^{ads} (k + \tau) dt\\
    \vdots&\\
    \mol{NH_3}^{ads} (k + n\tau) &= \mol{NH_3}^{ads} (k + (n-1)\tau) + A_{scr} \int_{0}^{\tau} \dot{\con{NH_3}}^{ads} (k + (n-1)\tau) dt
\end{align*}
%===
\begin{align*}
    \mol{NH_3}^{ads}(k + 1) = \mol{NH_3}^{ads} (k + n\tau) &= \mol{NH_3}^{ads} (k) + A_{scr} \sum_{i=0}^{n-1} \int_{0}^{\tau} \dot{\con{NH_3}}^{ads} (k + i\tau) dt
\end{align*}

Writing the above equation in terms of surface concentrations:
\begin{align*}
    \con{NH_3}^{ads}(k + 1) = \con{NH_3}^{ads} (k + n\tau) &= \con{NH_3}^{ads} (k) + \underbrace{\sum_{i=0}^{n-1} \int_{0}^{\tau} \dot{\con{NH_3}}^{ads} (k + i\tau)}_{\Omega(k)}   dt
\end{align*}

$\Omega(k)$ is the total change in the surface concentration of adsorbed ammonia
within one sample time.

% ==============================================================================
\subsection{Calculating the total surface concentration change of adsorbed ammonia within one sample time $\Omega(k)$}

The volumetric concentrations of the gaseous reactants are assumed to be
constant within the sample time, while the surface concentrations change at the
end of every residence time.

\begin{align*}
    \dot{\con{NH_3}}^{ads} (k + i\tau) &= r_{ads} (k + i \tau) - r_{des} (k + i \tau) - r_{scr} (k + i \tau) - r_{oxi} (k + i \tau)\\
    %===
    r_{ads} (k + i \tau) &= k_{ads} \con{NH_3}^{in}(k) \lr{\Gamma - \con{NH_3}^{ads} (k + i \tau)}\\
    r_{des} (k + i \tau) &= k_{des} \con{NH_3}^{ads}(k + i \tau)\\
    r_{scr} (k + i \tau) &= k_{scr} \con{NH_3}^{ads}(k + i \tau) \con{NO_x}^{in}(k)\\
    r_{oxi} (k + i \tau) &= k_{oxi} \con{NH_3}^{ads}(k + i \tau)\\
    \gamma_{ads} (k) &= k_{ads} \con{NH_3}^{in}(k)\\
    \gamma_{des} (k) &= \lr{k_{ads} \con{NH_3}^{in}(k) + k_{des} + k_{scr} \con{NO_x}^{in}(k) + k_{oxi}}\\
    %===
    \dot{\con{NH_3}}^{ads} (k + i\tau) &= \Gamma \gamma_{ads} (k) - \con{NH_3}^{ads} (k + i \tau) \gamma_{des} (k)
\end{align*}

Calculating the expression for $\Omega(k)$:
\begin{align*}
    \Omega(k) &= \sum_{i=0}^{n-1} \int_{0}^{\tau} \dot{\con{NH_3}}^{ads} (k + i\tau) dt\\
    &= \sum_{i=0}^{n-1} \dot{\con{NH_3}}^{ads} (k + i\tau) \tau \qquad \lrb{\because \text{The rate is assumed to be constant within the residence time}}\\
    &= \sum_{i=0}^{n-1} \lr{\Gamma \gamma_{ads} (k) - \con{NH_3}^{ads} (k + i \tau) \gamma_{des} (k)} \tau\\
    &= \tau n \Gamma \gamma_{ads} (k) - \tau \gamma_{des} (k) \underbrace{\sum_{i=0}^{n-1} \con{NH_3}^{ads} (k + i \tau) }_{n \sigma(k)} \\
    &= n \tau \lr{\Gamma \gamma_{ads} (k) - \sigma(k) \gamma_{des} (k)}
\end{align*}

The term $\sigma(k)$ is unknown and unobservable. It is the average surface
concentration at the end of every residence time within the sample time.

Moreover, we can get the expression for average rate of change of the surface
concentrations within the sample time as:
\begin{align*}
    \dot{\con{NH_3}}^{ads} (k) &= \frac{\Omega(k)}{\tau n} = \Gamma \gamma_{ads} (k) - \sigma(k) \gamma_{des} (k)\\
    &= \Gamma k_{ads} \con{NH_3}^{in}(k) - \sigma(k) \lr{k_{ads} \con{NH_3}^{in}(k) + k_{des} + k_{scr} \con{NO_x}^{in}(k) + k_{oxi}}
\end{align*}

% ==============================================================================

\subsection{Ammonia adsorption/desorption process dynamics model}
Substituting the expression for $\Omega(k)$ in the equation for
$\con{NH_3}^{ads}(k + 1)$:

\begin{align*}
    \con{NH_3}^{ads}(k + 1) &= \con{NH_3}^{ads} (k) + n \tau \lr{\Gamma \gamma_{ads} (k) - \sigma(k) \gamma_{des} (k)}\\
    &= \con{NH_3}^{ads}(k) + n \tau \Gamma k_{ads} \con{NH_3}^{in}(k) \notag\\
    &\qquad - n \tau \sigma(k) \lr{k_{ads} \con{NH_3}^{in}(k) + k_{des} + k_{scr} \con{NO_x}^{in}(k) + k_{oxi}}
\end{align*}

\subsection{Approximation of $\sigma(k)$ as $NH_3^{ads}(k)$ (Zero-order hold)}
\itbf{Note:} $\sigma(k)$ can be approximated as the surface concentration at the beginning of the sample time.

The average surface concentration $\sigma(k)$ in the sample time is approximated as the surface concentration at the
beginning of the sample time. This approximation will introduce an error in the model whose effects will be analyzed in
sequential sections. Thus,
\begin{align}
    \Omega(k) &\approx n \tau \lr{\Gamma \gamma_{ads} (k) - \con{NH_3}^{ads}(k) \gamma_{des} (k)}
\end{align}
%===
Thus, we have the approximate ammonia adsorption/desorption process dynamics model as:
\begin{align*}
    \con{NH_3}^{ads}(k + 1) &= \con{NH_3}^{ads} (k) + n \tau \lr{\Gamma \gamma_{ads} (k) - \con{NH_3}^{ads}(k) \gamma_{des} (k)}\\
    &= n \tau \Gamma \gamma_{ads} (k) + \con{NH_3}^{ads} (k) \lr{1 - n \tau \gamma_{des} (k)}\\
    &= n \tau \Gamma k_{ads} \con{NH_3}^{in}(k)  + \con{NH_3}^{ads} (k) \lr{1 - n \tau \lr{k_{ads} \con{NH_3}^{in}(k) + k_{des} + k_{scr} \con{NO_x}^{in}(k) + k_{oxi}}}
\end{align*}
\begin{multline}
    \con{NH_3}^{ads}(k + 1) = n \tau \Gamma k_{ads} \con{NH_3}^{in}(k)  + \con{NH_3}^{ads} (k) \lr{1 - n \tau \lr{k_{ads} \con{NH_3}^{in}(k) + k_{des} + k_{scr} \con{NO_x}^{in}(k) + k_{oxi}}}
\end{multline}
%===
Examining the individual terms:
\begin{align*}
    \con{NH_3}^{ads}(k + 1) =& \con{NH_3}^{ads}(k)\\
        &+ n\tau k_{ads} \con{NH_3}^{in} \lr{\Gamma - \con{NH_3}^{ads}(k)} \\
        &- n \tau \lr{k_{oxi} + k_{des}} \con{NH_3}^{ads} (k)\\
        &- n \tau k_{scr} \con{NO_x}^{in}(k) \con{NH_3}^{ads}(k)
\end{align*}
%===
The above equation is nothing but multiplying the rate equation with the sampling interval to get the measurement
update.

Similarly, if we flip the approximation, that is if,
\begin{align*}
    \con{NH_3}^{ads}(k) &\approx \sigma(k)
\end{align*}
then,
\begin{align*}
    \sigma(k + 1) =& \sigma(k)\\
        &+ n\tau k_{ads} \con{NH_3}^{in} \lr{\Gamma - \sigma(k)} \\
        &- n \tau \lr{k_{oxi} + k_{des}} \sigma(k)\\
        &- n \tau k_{scr} \con{NO_x}^{in}(k) \sigma(k)
\end{align*}

as, $n\tau = t_s$ is a constant for the process, the above equation can be rewritten as:
\begin{align*}
    \sigma(k + 1) =& \sigma(k)\\
        &+ t_s k_{ads} \con{NH_3}^{in} (k) \lr{\Gamma - \sigma(k)} \\
        &- t_s \lr{k_{oxi} + k_{des}} \sigma(k)\\
        &- t_s k_{scr} \con{NO_x}^{in}(k) \sigma(k)
\end{align*}


\section{$NO_x$ Process Dynamics}

The $NO_x$ process dynamics involve only the selective catalytic reduction reaction. The gaseous $NO_x$ that enters the
catalyst chamber reacts with the adsorbed ammonia on the surface and forms $N_2$ and $H_2O$ (Eiley-Rideal Mechanism).
The process frees up the adsorption sites for the next cycle of gaseous ammonia. The reaction rate is proportional to
the volumetric concentration of the gaseous $NO_x$ and the surface concentration of the adsorbed ammonia.
\begin{align}
    4 NH_3 ^{ads} + 4 NO + O_2 &\xrightarrow[]{k_{scr}} 4 N_2 + 6 H_2O
\end{align}

Thus, the rate of $NO_x$ reduction on the catalyst surface can be modeled as:
\begin{align}
    \frac{d \con{NO_x}^{scr}}{dt} &= - k_{s2v} r_{scr} \\
    r_{scr} &= k_{scr} \con{NH_3}^{ads} \con{NO_x}^{in}\\
    \dot{\con{NO_x}}^{scr} &= -k_{s2v} k_{scr} \con{NH_3}^{ads} \con{NO_x}^{in}
\end{align}

% ==============================================================================
\subsection{Molar conservation at the scale of residence time}

Similar to the ammonia adsorption/desorption process, at the end of every residence time, a fresh parcel of gaseous
reactants enters the catalyst chamber. Thus, the molar conservation for the $NO_x$ reduction process can be
correlated with the inlet and outlet molar concentrations of the $NO_x$ at the beginning and the end of residence time. Thus,
\begin{align*}
    \mol{NO_x}^{out} (k + \tau) &= \mol{NO_x}^{in} (k) + V \int_{0}^{\tau}  \dot{\con{NO_x}}^{scr} (k) dt\\
    \mol{NO_x}^{out} (k + 2\tau) &= \mol{NO_x}^{in} (k + \tau) + V \int_{0}^{\tau} \dot{\con{NO_x}}^{scr} (k+\tau)  dt\\
    \vdots &\\
    \mol{NO_x}^{out} (k + n\tau) &= \mol{NO_x}^{in} (k + (n-1)\tau) + V \int_{0}^{\tau} \dot{\con{NO_x}}^{scr} (k+(n-1)\tau) dt
\end{align*}

The above equations show that the measurement of $NO_x$ concentration at the outlet depends only on the measurement of
the $NO_x$ concentration at the inlet one residence time before. Thus, there is no integrating effect of the $NO_x$
within the sample time. Writing in terms of volumetric concentrations, we have:

\begin{align*}
    \con{NO_x}^{out} (k + n\tau) &= \con{NO_x}^{in} (k + (n-1)\tau) + \int_{0}^{\tau}  \dot{\con{NO_x}}^{scr} (k + (n-1)\tau) dt
\end{align*}

Introducing the following two approximations:
\begin{enumerate}
    \item Zero-order-hold for the inlet concentration of $NO_x$:
        \begin{align*}
            \con{NO_x}^{in} (k + i \tau) &\approx \con{NO_x}^{in} (k) \qquad \forall i < n
        \end{align*}
    \item Using average surface concentration at the sample for the surface concentration of the adsorbed ammonia:
        \begin{align*}
            \con{NH_3}^{ads} (k + i \tau) &\approx \sigma(k) \qquad \forall i < n
        \end{align*}
\end{enumerate}

Thus,
\begin{align*}
    \dot{\con{NO_x}}^{scr} (k + i \tau) =  \dot{\con{NO_x}}^{scr} (k) = -k_{s2v} k_{scr} \sigma(k) \con{NO_x}^{in} (k) \qquad \forall i < n
\end{align*}


Thus, we have the following expression for the $NO_x$ process dynamics:

\begin{align*}
    \con{NO_x}^{out} (k + n\tau) &= \con{NO_x}^{in} (k) - k_{s2v} k_{scr} \sigma(k) \con{NO_x}^{in} (k) \tau\\
    \implies \con{NO_x}^{out} (k + 1) &= \con{NO_x}^{in} (k) \lr{1 - k_{s2v} k_{scr} \sigma(k) \tau}
\end{align*}


% ==============================================================================

\section{Gaseous Ammonia Process Dynamics}
The gaseous ammonia process dynamics involve the adsorption and desorption dynamics. The gaseous ammonia that enters the
catalyst chamber gets adsorbed on the free sites on the catalyst surface at a rate proportional to the volumetric
concentration of the gaseous ammonia and the surface concentration of the free sites. A part of adsorbed ammonia then
desorbs back to the gas phase at a rate that is proportional to the surface concentrations of the adsorbed ammonia.
\begin{align*}
    NH_3 + \Theta_{free} &\xrightleftharpoons[k_{des}]{k_{ads}} NH_3^{ads}
\end{align*}

Thus, the rate of change of gaseous $NH_3$ on the catalyst surface can be modeled as:
\begin{align*}
    \frac{d \con{NH_3}^{scr}}{dt} &= k_{s2v} (-r_{ads} + r_{des}) \\
    r_{ads} &= k_{ads} \con{NH_3}^{in} \lr{\Gamma - \con{NH_3}^{ads}}\\
    r_{des} &= k_{des} \con{NH_3}^{ads}\\
    \dot{\con{NH_3}}^{scr} &= -k_{s2v}k_{ads} \con{NH_3}^{in} \lr{\Gamma - \con{NH_3}^{ads}} + k_{s2v} k_{des} \con{NH_3}^{ads}\\
\end{align*}

% ==============================================================================
\subsection{Molar conservation at the scale of residence time}

Similar to the ammonia adsorption/desorption process, at the end of every residence time, a fresh parcel of gaseous
reactants enters the catalyst chamber. Thus, the molar conservation for the gaseous $NH_3$ adsorption/desorption process
can be correlated with the inlet and outlet molar concentrations of the $NH_3$ at the beginning and the end of residence
time. Similar to the $NO_x$ process dynamics the measurement of $NH_3$ concentration at the outlet depends only on the
measurement of the $NH_3$ concentration at the inlet one residence time before. That is, there is no integrating effect
of the $NH_3$ within the sample time either. Writing in terms of volumetric concentrations, we have:

\begin{align*}
    \con{NH_3}^{out} (k + n\tau) &= \con{NH_3}^{in} (k + (n-1)\tau) + \int_{0}^{\tau}  \dot{\con{NH_3}}^{scr} (k + (n-1)\tau) dt
\end{align*}

Introducing the following two approximations:
\begin{enumerate}
    \item Zero-order-hold for the inlet concentration of $NH_3$:
        \begin{align*}
            \con{NH_3}^{in} (k + i \tau) &\approx \con{NH_3}^{in} (k) \qquad \forall i < n
        \end{align*}
    \item Using average surface concentration at the sample for the surface concentration of the adsorbed ammonia:
        \begin{align*}
            \con{NH_3}^{ads} (k + i \tau) &\approx \sigma(k) \qquad \forall i < n
        \end{align*}
\end{enumerate}

Thus,
\begin{align*}
    \dot{\con{NH_3}}^{scr} (k + i \tau) &= -k_{s2v}k_{ads} \con{NH_3}^{in} \lr{\Gamma - \sigma(k)} + k_{s2v} k_{des} \sigma(k)
    \qquad \forall i < n
\end{align*}


Thus, we have the following expression for the $NH_3$ process dynamics:

\begin{align*}
    \con{NH_3}^{out} (k + n\tau) &= \con{NH_3}^{in} (k) - \tau k_{s2v}k_{ads} \con{NH_3}^{in}(k) \lr{\Gamma - \sigma(k)} + \tau k_{s2v} k_{des} \sigma(k)\\
    \con{NH_3}^{out} (k + 1) &= \con{NH_3}^{in}(k) \lr{1 - \tau k_{s2v}k_{ads} \lr{\Gamma - \sigma(k)}} + \tau k_{s2v} k_{des} \sigma(k)
\end{align*}

\section{Urea Dosing Process Dynamics}
The urea dosing dynamics involve the following reactions:
\begin{align*}
    NH_2 - CO - NH_2 (liquid) &\longrightarrow NH_2 - CO - NH_2^* + x H_2 O
                & &[\text{AdBlue evaporation}] \\
    NH_2 - CO - NH_2^*  &\longrightarrow  HNCO + NH_3
                & &[\text{Urea decomposition}] \\
    HNCO + H_2O &\longrightarrow NH_3 + CO_2
                & &[\text{Isocynic acid hydrolysis}] \\
\end{align*}

The above reactions can be aggregated into:
\begin{align*}
    NH_2 - CO - NH_2 (liquid) &\longrightarrow 2 NH_3 + CO_2 + x H_2 O
\end{align*}

The rate of ammonia production is twice the rate of decomposition of the urea in the solution (AdBlue). Thus,
\begin{align*}
    \frac{d \con{NH_3}^{in}}{dt} &= 2 r_{u}\\
    r_{u} &= k_{u} \con{NH_2 - CO - NH_2} \qquad (constant)
\end{align*}

As the concentration of urea in the solution is constant, the above rate becomes a constant. Thus, the total moles of
ammonia produced at the inlet within a sample time would be:
\begin{align*}
    \mol{NH_3}^{in} (k) &= t_s \times \underbrace{2 r_{u}}_{\text{Rate of decomposition}} \times \underbrace{t_s u_{inj} (k)}_{\text{Volume injected}}
\end{align*}
where $u_{inj}$ is urea-solution injection rate in $ml/s$.

This is the total moles of ammonia produced in $n$ residence times during the reaction process. Thus, the number of moles of ammonia in the chamber at a given residence time would be $\mol{NH_3}^{in}(k)/n$.
Thus, we have the inlet volumetric concentration of ammonia as:
\begin{align*}
    \con{NH_3}^{in} (k) &= \frac{\mol{NH_3}^{in} (k)}{nV}
                          = \frac{2 r_u \tau}{t_s V} \times t_s^2 \times u_{inj} (k)
                          = \frac{2 r_u t_s}{V} \times \tau \times u_{inj} (k)
\end{align*}
Thus the ammonia concentration at the inlet is a function of urea dosing rate and the residence time of the reaction.

Substituting, $\tau = \frac{V \rho_0}{F}$, we get:
\begin{align}
    \con{NH_3}^{in} (k) &= 2r_u t_s \rho_0 \times \frac{u_{inj}(k)}{F}
\end{align}

The above reciprocal relationship breaks down at zero flow rate which makes sense physically.

% ======================================================================================================================
The sensitivity of change in ammonia to the change residence time (due to change in flow-rate) or change in urea dosing rate can be calculated as follows:
\begin{align*}
    \frac{\partial \con{NH_3}^{in} }{\partial F} &= -2 r_u t_s \rho_0 \times \frac{u_{inj}}{F^2} \\
    \implies \frac{\delta \con{NH_3}^{in}}{\con{NH_3}^{in}} &= \frac{-\delta F}{F}
    \qquad \text{at constant } u_{inj}
\end{align*}
Similarly,
\begin{align*}
    \frac{\partial \con{NH_3}^{in} }{\partial u_{inj}} &= \frac{2 r_u t_s \rho_0}{F} \\
    \implies \frac{\delta \con{NH_3}^{in}}{\con{NH_3}^{in}} &= \frac{ \delta u_{inj}}{u_{inj}} \qquad \text{at constant } F
\end{align*}

From practical range of changes in data for $F$ and $u_{inj}$,
\begin{align*}
    \frac{ \delta u_{inj}}{u_{inj}}, \qquad \frac{\delta F}{F} \qquad \text{have the same order of magnitude.}
\end{align*}
Thus, the effect of change in the urea dosing is as prominent on inlet ammonia concentration as the changes in residence time due to flow rate changes. Thus, the effects of changes in residence time on ammonia generation can be neglected for the approximate model, resulting in:
\begin{align}
    \con{NH_3}^{in}(k) &= \nu_u \frac{u_{inj}}{F}    \label{eqn::urea_inj}
\end{align}
where,
\begin{align*}
    \nu_u &= 2 r_u t_s \rho_0\\
    T_{min} &\leq T \leq T_{max}\\
    F_{min} &\leq F \leq F_{max}
\end{align*}

\section{Complete Process Dynamics Model for SCR-ASC System}

From previous derivations, we have the complete process dynamics of SCR-ASC dynamics with urea dosing as:

\begin{enumerate}
    \item Urea dosing dynamics:
    \begin{align}
    \con{NH_3}^{in} (k) &= 2r_u t_s \rho_0 \times \frac{u_{inj}(k)}{F}
    \label{eqn::urea_dosing}
    \end{align}

    \item $NO_x$ reduction dynamics:
    \begin{align}
    \con{NO_x}^{out} (k + 1) &= \con{NO_x}^{in} (k) \lr{1 - k_{s2v} k_{scr} \sigma(k) \tau}
    \label{eqn::NOX_process}
    \end{align}

    \item Gaseous ammonia dynamics:
    \begin{align}
    \con{NH_3}^{out} (k + 1) &= \con{NH_3}^{in}(k) \lr{1 - \tau k_{s2v}k_{ads} \lr{\Gamma - \sigma(k)}} + \tau k_{s2v} k_{des} \sigma(k)
    \label{eqn::NH3_gas_process}
    \end{align}

    \item Ammonia adsorption/desorption dynamics:
    \begin{align}
        \sigma^{ub}(k+1) &= \sigma(k) + t_s k_{ads} \con{NH_3}^{in} (k) \lr{\Gamma - \sigma(k)} \notag \\
                         & \quad - t_s \lr{k_{oxi} + k_{des}} \sigma(k) - t_s k_{scr} \con{NO_x}^{in}(k) \sigma(k)
                \label{eqn::sigma_process}\\
        \sigma(k) &= g_{sat} \lr{\sigma^{ub}(k)}  \label{eqn::sigma_sat}\\
        \text{where, }\qquad & \notag \\
        g_{sat} (\sigma) &= \begin{cases}
        \sigma & \text{if } 0 \leq \sigma \leq \Gamma \\
        \Gamma & \text{if } \sigma > \Gamma \\
        0 & \text{if } \sigma < 0
    \end{cases} \label{eqn::sat_func}
    \end{align}
\end{enumerate}

The above four equations are the complete process dynamics of SCR-ASC dynamics with urea dosing. The above equations are
nonlinear and coupled with implicit dependence on temperature and flow rate. In this section the above equations are
rewritten in a parametric form with standard state-space representation. For convenience and clarity '(k)' is dropped
from the variables.

Note that the tailpipe (gaseous) ammonia dynamics don't affect the $NO_x$ reduction dynamics and we don't have the
necessary measurements for trucks for model parameter estimation. Moreover, we only need the models for $NO_x$ reduction
and Ammonia adsorption for characterizing the I/O dynamics of the system using the measurements.

