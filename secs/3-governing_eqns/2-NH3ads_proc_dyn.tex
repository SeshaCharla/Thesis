\section{Ammonia Adsorption/Desorption Process Dynamics \label{sec::sigma_deriv}}
\begin{figure}[H]
    \centering
    \includegraphics[width=0.5\textwidth]{./figs/3-governing_eqns/plug_flow_discrete.png}
    \caption{Discrete plug-flow reactor model}
    \label{fig:plug_flow_discrete}
\end{figure}
%===
The ammonia adsorption/desorption process dynamics involve all the three above-mentioned reactions. The gaseous ammonia
that enters the catalyst chamber gets adsorbed onto the free sites on the catalyst surface at a rate proportional to the
volumetric concentration of the gaseous ammonia and the surface concentration of the free sites. The adsorbed ammonia
then either reacts with the gaseous $NO_x$ (Eley-Rideal Mechanism) releasing $N_2$ and $H_2O$ or decomposes to form
$N_2$ and $H_2O$ (Surface Decomposition). In either case, the process frees up the adsorption sites for the next cycle
of gaseous ammonia. We first derive the dynamics of the adsorption/desorption process when the catalyst remains unsaturated and then extend it to the saturated case later in Section \ref{sec::sat_cat}.
%===
\begin{align*}
    4 NH_3 ^{ads} + 4 NO + O_2 &\xrightarrow[]{k_{scr}} 4 N_2 + 6 H_2O \\
    4 NH_3^{ads} + 3 O_2 &\xrightarrow[]{k_{oxi}} 2 N_2 + 6 H_2O \\
    NH_3 + \Theta_{free} &\xrightleftharpoons[k_{des}]{k_{ads}} NH_3^{ads}
\end{align*}
Thus, the rate of ammonia adsorption on the catalyst surface can be modelled as:
\begin{align*}
    \frac{d \con{NH_3}^{ads}}{dt} &= r_{ads} - r_{des} - r_{scr} - r_{oxi}\\
    r_{ads} &= k_{ads} \con{NH_3}^{in} \lr{\Gamma - \con{NH_3}^{ads}} \qquad \lrb{\Gamma = \frac{\Theta_{free} + \Theta_{occupied}}{A_{scr}}}\\
    r_{des} &= k_{des} \con{NH_3}^{ads}\\
    r_{scr} &= k_{scr} \con{NH_3}^{ads} \con{NO_x}^{in}\\
    r_{oxi} &= k_{oxi} \con{NH_3}^{ads}\\
    \dot{\con{NH_3}}^{ads} &= \Gamma \underbrace{k_{ads} \con{NH_3}^{in}}_{\gamma_{ads}} - \con{NH_3}^{ads} \underbrace{\lr{k_{ads} \con{NH_3}^{in} + k_{des} + k_{scr} \con{NO_x}^{in} + k_{oxi}}}_{\gamma_{des}}
\end{align*}
\itbf{Note:} $\Gamma$ and $\con{NH_3}^{ads}$ are surface concentrations (in $moles/m^2$) while the rest are volumetric
concentrations (in $moles/m^3$). The rate constant units are adjusted appropriately to make the rate equations
consistent.

The units of all the rates and rate constants are tabulated below:
\begin{align*}
    r_{ads} &= \frac{moles}{m^2 \cdot s} &
    k_{ads} &= \frac{m^3}{moles \cdot s} \\
    r_{des} &= \frac{moles}{m^2 \cdot s} &
    k_{des} &= s^{-1} \\
    r_{scr} &= \frac{moles}{m^2 \cdot s} &
    k_{scr} &= \frac{m^3}{moles \cdot s} \\
    r_{oxi} &= \frac{moles}{m^2 \cdot s} &
    k_{oxi} &= s^{-1}
\end{align*}
%===
\itbf{Note:} In this section we are interested in the Surface rates where the products are assumed to linger just on the
surface of the catalyst. The surface rates of the gaseous products/reactants can be converted to volumetric rates by
using the conversion factor $A_{scr}/V$ where $A_{scr}$ is the area of the SCR catalyst and $V$ is the volume of the
catalyst chamber. This assumes that there is instantaneous mixing of the close-to-surface moles with the gaseous moles.
Thus,
\begin{align}
    k^{vol} = \underbrace{\lr{\frac{A_{scr}}{V}}}_{k_{s2v} \, (m^{-1})} k^{surf}
\end{align}
% ==============================================================================
\subsection{Molar conservation at the scale of residence time}

The molar storage on the catalyst surface changes at the end of every residence time and a "fresh" set of gaseous
reactants enters the catalyst chamber. And, within the sample time $t_s$, the volumetric concentrations of the gaseous
reactants can be considered constant. \itbf{Let there be $n$ residence times within one sample time.} It is implicitly
assumed that there are an integer number of residence times within one sample time. When that is not the case, the
resulting error is attributed to model structure error.
\begin{align}
    t_s &= n \tau
    \label{eqn::ts_2_tau}
\end{align}
%===
Noting that $\lrf{\bullet}$ denotes moles and $\lrb{\bullet}$ denotes the concentration of the species, the molar conservation for the adsorption/desorption process can be written as:
%===
\begin{align*}
    \mol{NH_3}^{ads} (k + \tau) &= \mol{NH_3}^{ads} (k) + A_{scr} \int_{0}^{\tau} \dot{\con{NH_3}}^{ads} (k) dt\\
    \mol{NH_3}^{ads} (k + 2\tau) &= \mol{NH_3}^{ads} (k + \tau) + A_{scr} \int_{0}^{\tau} \dot{\con{NH_3}}^{ads} (k + \tau) dt\\
    \vdots&\\
    \mol{NH_3}^{ads} (k + n\tau) &= \mol{NH_3}^{ads} (k + (n-1)\tau) + A_{scr} \int_{0}^{\tau} \dot{\con{NH_3}}^{ads} (k + (n-1)\tau) dt
\end{align*}
%===
\begin{align*}
    \mol{NH_3}^{ads}(k + 1) = \mol{NH_3}^{ads} (k + n\tau) &= \mol{NH_3}^{ads} (k) + A_{scr} \sum_{i=0}^{n-1} \int_{0}^{\tau} \dot{\con{NH_3}}^{ads} (k + i\tau) dt
\end{align*}

Writing the above equation in terms of surface concentrations:
\begin{align*}
    \con{NH_3}^{ads}(k + 1) = \con{NH_3}^{ads} (k + n\tau) &= \con{NH_3}^{ads} (k) + \underbrace{\sum_{i=0}^{n-1} \int_{0}^{\tau} \dot{\con{NH_3}}^{ads} (k + i\tau)}_{\Omega(k)}   dt
\end{align*}

$\Omega(k)$ is the total change in the surface concentration of adsorbed ammonia
within one sample time.

% ==============================================================================
\subsection{Calculating the total surface concentration change of adsorbed ammonia within one sample time $\Omega(k)$}

The volumetric concentrations of the gaseous reactants are assumed to be
constant within the sample time, while the surface concentrations change at the
end of every residence time.

\begin{align*}
    \dot{\con{NH_3}}^{ads} (k + i\tau) &= r_{ads} (k + i \tau) - r_{des} (k + i \tau) - r_{scr} (k + i \tau) - r_{oxi} (k + i \tau)\\
    %===
    r_{ads} (k + i \tau) &= k_{ads} \con{NH_3}^{in}(k) \lr{\Gamma - \con{NH_3}^{ads} (k + i \tau)}\\
    r_{des} (k + i \tau) &= k_{des} \con{NH_3}^{ads}(k + i \tau)\\
    r_{scr} (k + i \tau) &= k_{scr} \con{NH_3}^{ads}(k + i \tau) \con{NO_x}^{in}(k)\\
    r_{oxi} (k + i \tau) &= k_{oxi} \con{NH_3}^{ads}(k + i \tau)\\
    \gamma_{ads} (k) &= k_{ads} \con{NH_3}^{in}(k)\\
    \gamma_{des} (k) &= \lr{k_{ads} \con{NH_3}^{in}(k) + k_{des} + k_{scr} \con{NO_x}^{in}(k) + k_{oxi}}\\
    %===
    \dot{\con{NH_3}}^{ads} (k + i\tau) &= \Gamma \gamma_{ads} (k) - \con{NH_3}^{ads} (k + i \tau) \gamma_{des} (k)
\end{align*}

Calculating the expression for $\Omega(k)$:
\begin{align*}
    \Omega(k) &= \sum_{i=0}^{n-1} \int_{0}^{\tau} \dot{\con{NH_3}}^{ads} (k + i\tau) dt\\
    &= \sum_{i=0}^{n-1} \dot{\con{NH_3}}^{ads} (k + i\tau) \tau \qquad \lrb{\because \text{The rate is assumed to be constant within the residence time}}\\
    &= \sum_{i=0}^{n-1} \lr{\Gamma \gamma_{ads} (k) - \con{NH_3}^{ads} (k + i \tau) \gamma_{des} (k)} \tau\\
    &= \tau n \Gamma \gamma_{ads} (k) - \tau \gamma_{des} (k) \underbrace{\sum_{i=0}^{n-1} \con{NH_3}^{ads} (k + i \tau) }_{n \sigma(k)} \\
    &= n \tau \lr{\Gamma \gamma_{ads} (k) - \sigma(k) \gamma_{des} (k)}
\end{align*}

The term $\sigma(k)$ is unknown and unobservable. It is the average surface
concentration at the end of every residence time within the sample time.

Moreover, we can get the expression for average rate of change of the surface
concentrations within the sample time as:
\begin{align*}
    \dot{\con{NH_3}}^{ads} (k) &= \frac{\Omega(k)}{\tau n} = \Gamma \gamma_{ads} (k) - \sigma(k) \gamma_{des} (k)\\
    &= \Gamma k_{ads} \con{NH_3}^{in}(k) - \sigma(k) \lr{k_{ads} \con{NH_3}^{in}(k) + k_{des} + k_{scr} \con{NO_x}^{in}(k) + k_{oxi}}
\end{align*}

% ==============================================================================

\subsection{Ammonia adsorption/desorption process dynamics model}
Substituting the expression for $\Omega(k)$ in the equation for
$\con{NH_3}^{ads}(k + 1)$:

\begin{align*}
    \con{NH_3}^{ads}(k + 1) &= \con{NH_3}^{ads} (k) + n \tau \lr{\Gamma \gamma_{ads} (k) - \sigma(k) \gamma_{des} (k)}\\
    &= \con{NH_3}^{ads}(k) + n \tau \Gamma k_{ads} \con{NH_3}^{in}(k) \notag\\
    &\qquad - n \tau \sigma(k) \lr{k_{ads} \con{NH_3}^{in}(k) + k_{des} + k_{scr} \con{NO_x}^{in}(k) + k_{oxi}}
\end{align*}

\subsection{Approximation of $\sigma(k)$ as $NH_3^{ads}(k)$ (Zero-order hold)}
\itbf{Note:} $\sigma(k)$ can be approximated as the surface concentration at the beginning of the sample time.

The average surface concentration $\sigma(k)$ in the sample time is approximated as the surface concentration at the
beginning of the sample time. This approximation will introduce an error in the model whose effects will be analyzed in
sequential sections. Thus,
\begin{align}
    \Omega(k) &\approx n \tau \lr{\Gamma \gamma_{ads} (k) - \con{NH_3}^{ads}(k) \gamma_{des} (k)}
\end{align}
%===
Thus, we have the approximate ammonia adsorption/desorption process dynamics model as:
\begin{align*}
    \con{NH_3}^{ads}(k + 1) &= \con{NH_3}^{ads} (k) + n \tau \lr{\Gamma \gamma_{ads} (k) - \con{NH_3}^{ads}(k) \gamma_{des} (k)}\\
    &= n \tau \Gamma \gamma_{ads} (k) + \con{NH_3}^{ads} (k) \lr{1 - n \tau \gamma_{des} (k)}\\
    &= n \tau \Gamma k_{ads} \con{NH_3}^{in}(k)  + \con{NH_3}^{ads} (k) \lr{1 - n \tau \lr{k_{ads} \con{NH_3}^{in}(k) + k_{des} + k_{scr} \con{NO_x}^{in}(k) + k_{oxi}}}
\end{align*}
\begin{multline}
    \con{NH_3}^{ads}(k + 1) = n \tau \Gamma k_{ads} \con{NH_3}^{in}(k)  + \con{NH_3}^{ads} (k) \lr{1 - n \tau \lr{k_{ads} \con{NH_3}^{in}(k) + k_{des} + k_{scr} \con{NO_x}^{in}(k) + k_{oxi}}}
\end{multline}
%===
Examining the individual terms:
\begin{align*}
    \con{NH_3}^{ads}(k + 1) =& \con{NH_3}^{ads}(k)\\
        &+ n\tau k_{ads} \con{NH_3}^{in} \lr{\Gamma - \con{NH_3}^{ads}(k)} \\
        &- n \tau \lr{k_{oxi} + k_{des}} \con{NH_3}^{ads} (k)\\
        &- n \tau k_{scr} \con{NO_x}^{in}(k) \con{NH_3}^{ads}(k)
\end{align*}
%===
The above equation is nothing but multiplying the rate equation with the sampling interval to get the measurement
update.

Similarly, if we flip the approximation, that is if,
\begin{align*}
    \con{NH_3}^{ads}(k) &\approx \sigma(k)
\end{align*}
then,
\begin{align*}
    \sigma(k + 1) =& \sigma(k)\\
        &+ n\tau k_{ads} \con{NH_3}^{in} \lr{\Gamma - \sigma(k)} \\
        &- n \tau \lr{k_{oxi} + k_{des}} \sigma(k)\\
        &- n \tau k_{scr} \con{NO_x}^{in}(k) \sigma(k)
\end{align*}

as, $n\tau = t_s$ is a constant for the process, the above equation can be rewritten as:
\begin{align*}
    \sigma(k + 1) =& \sigma(k)\\
        &+ t_s k_{ads} \con{NH_3}^{in} (k) \lr{\Gamma - \sigma(k)} \\
        &- t_s \lr{k_{oxi} + k_{des}} \sigma(k)\\
        &- t_s k_{scr} \con{NO_x}^{in}(k) \sigma(k)
\end{align*}

