\section{Complete Process Dynamics Model for SCR-ASC System}

From previous derivations, we have the complete process dynamics of SCR-ASC dynamics with urea dosing as:

\begin{enumerate}
    \item Urea dosing dynamics:
    \begin{align}
    \con{NH_3}^{in} (k) &= 2r_u t_s \rho_0 \times \frac{u_{inj}(k)}{F}
    \label{eqn::urea_dosing}
    \end{align}

    \item $NO_x$ reduction dynamics:
    \begin{align}
    \con{NO_x}^{out} (k + 1) &= \con{NO_x}^{in} (k) \lr{1 - k_{s2v} k_{scr} \sigma(k) \tau}
    \label{eqn::NOX_process}
    \end{align}

    \item Gaseous ammonia dynamics:
    \begin{align}
    \con{NH_3}^{out} (k + 1) &= \con{NH_3}^{in}(k) \lr{1 - \tau k_{s2v}k_{ads} \lr{\Gamma - \sigma(k)}} + \tau k_{s2v} k_{des} \sigma(k)
    \label{eqn::NH3_gas_process}
    \end{align}

    \item Ammonia adsorption/desorption dynamics:
    \begin{align}
        \sigma^{ub}(k+1) &= \sigma(k) + t_s k_{ads} \con{NH_3}^{in} (k) \lr{\Gamma - \sigma(k)} \notag \\
                         & \quad - t_s \lr{k_{oxi} + k_{des}} \sigma(k) - t_s k_{scr} \con{NO_x}^{in}(k) \sigma(k)
                \label{eqn::sigma_process}\\
        \sigma(k) &= g_{sat} \lr{\sigma^{ub}(k)}  \label{eqn::sigma_sat}\\
        \text{where, }\qquad & \notag \\
        g_{sat} (\sigma) &= \begin{cases}
        \sigma & \text{if } 0 \leq \sigma \leq \Gamma \\
        \Gamma & \text{if } \sigma > \Gamma \\
        0 & \text{if } \sigma < 0
    \end{cases} \label{eqn::sat_func}
    \end{align}
\end{enumerate}

The above four equations are the complete process dynamics of SCR-ASC dynamics with urea dosing. The above equations are
nonlinear and coupled with implicit dependence on temperature and flow rate. In this section the above equations are
rewritten in a parametric form with standard state-space representation. For convenience and clarity '(k)' is dropped
from the variables.

Note that the tailpipe (gaseous) ammonia dynamics don't affect the $NO_x$ reduction dynamics and we don't have the
necessary measurements for trucks for model parameter estimation. Moreover, we only need the models for $NO_x$ reduction
and Ammonia adsorption for characterizing the I/O dynamics of the system using the measurements.
