\cleardoublepage
\chapter{Introduction}

Diesel engine after-treatment systems reduce the concentrations of harmful gases such as $NO_x$ and $CO$ from exhaust
emissions. The Selective Catalytic Reduction (SCR) system reduces the engine-out $NO_x$ into $N_2$ and $H_2O$, using
ammonia from urea dosing in the presence of a catalyst. This catalytic conversion process is regulated to decrease the
levels of ammonia in the exhaust (ammonia slip), through two methods. The first is feedback control, which adjusts the
urea injection rate based on the exhaust $NO_x$ concentrations. The second method involves an additional catalytic
reaction, the Ammonia Slip Catalysis(ASC), which is designed to oxidize any excess ammonia at the end of the SCR bed.
Figure-\ref{fig:exhaust_scheme} shows a schematic of the SCR-ASC system. It is found that $NO_x$-reduction capacity of
the catalyst reduces with time due to hydro-thermal aging and deposition of contaminants such as excess urea, platinum,
sulfur and phosphorous \cite{matsumoto2016model}. The aging of the catalyst leads to increased emissions of $NO_x$ and
ammonia slip, which can result in non-compliance with environmental regulations. A fault detection system for the aging
of the catalyst would provide better control over the maintenance of the system and improve the overall reduction in
emissions.

\begin{figure}[H]
        \centering
        \includegraphics[width=0.75\textwidth]{./figs/1-intro/SCR-ASC_model.png}
        \caption{Schematic of the SCR-ASC system}
        \label{fig:exhaust_scheme}
\end{figure}

Thus, modern diesel after-treatment systems, particularly those that integrate Selective Catalytic Reduction (SCR) with
Ammonia Slip Catalyst (ASC), necessitate advanced on-board diagnostics (OBD) tools for accurate assessment of catalyst
aging. However, the effectiveness of traditional OBD approaches for this purpose has been impeded by the absence of
model validation with real-world catalyst degradation data, the limitations imposed by existing commercial $NO_x$
sensors' cross-sensitivity to ammonia and the absence of ammonia sensors in commercial vehicles.

Numerous studies have been conducted on modeling the SCR-ASC systems and their control \cite{yuan2015diesel}. A
prevalent modeling approach is to approximate the PDE model for the plug flow reaction into a set of ODEs using sequence
of Continuously Stirred Tank Reactors (CSTRs) approximation (\cite{hsieh2011development} , and \cite{nova2014urea}). The
single CSTR approach was first justified in \cite{devarakonda2008adequacy} and a nonlinear model was developed using
these assumptions, which was then linearized for feedback control design (\cite{devarakonda2009model}). With this model,
observers were designed to estimate the states corresponding to the catalyst's storage (\cite{ma2017observer},
\cite{jain2020term}). A method for detecting the catalyst's aging using an observer for the the change in the maximum
storage capacity of the catalyst, modeled as an exponential function of temperature, was also proposed in
\cite{ma2017observer} and \cite{jiang2019hydrothermal}. Alternatively, an onboad diagnostics (OBD) logic for catalyst
aging based on the tailpipe $NO_x$ and ammonia is proposed in \cite{matsumoto2016model} based on the hypthesis of
reduction in adsorption cites and demonstrated on real world trucks. A common theme in these studies is that the system
identification results in a non-convex, nonlinear parameter estimation problem. Moreover, these studies assume the
availability of all the gaseous states at the tail-pipe for state estimation and to eliminate the effects of
cross-sensitivity of the $NO_x$ sensors, which is not always the case in real-world applications as tailpipe ammonia
measurements are not generally available in commercial vehicles. One other fundamental issue with the CSTR model in the
context of sensor sampling is that the discretization of the continuous model does not account for the resisdance time
of the reactants resulting in high sampling frequency requirements that are not feasible in practice. Additionally, the
discretization requires at least two CSTRs to capture the system dynamics and causality \cite{charla2024reduced},
thereby increasing the model order.

An alternative approach to the problem would be discarding the CSTR assumption and modelling the time evolution of the
sensor signals when a "plug" or "parcel" of the exhaust gases flows through the chamber in discrete time considering the
interplay between sampling and residence time of the reactants.

Such a time evolution introduces constraints on the model due to sampling limitations. To capture the transient
dynamics, the sampling time should be significantly smaller than the "residence time" of the reactants inside the
SCR-ASC chamber. If that is not the case, as is the situation with the present available test and truck data, time
integrated states assuming zero-order holds during the transients need to be introduced into the model and the
input-output model can be derived from the resulting state-space model.

The catalyst saturation that is inherently considered in the CSTR approach needs to be explicitly included as separate
mode of the system. The present work developed such a model under catalyst saturation which becomes one of the modes of
the complete switched nonlinear model. Following the model development we present the parameter estimation algorithm
that estimates the parameters of the saturated catalyst model using the real-world data whose operating mode is unknown
and has additional uncertainties due to $NO_x$ sensor's cross-sensitivity to ammonia.

\begin{figure}[H]
    \centering
    \includegraphics[width = 0.9\textwidth]{./figs/1-intro/ModellingApproach.png}
    \caption{Summary of Modelling Approach}
\end{figure}

Finally, we develope a statistical test for detecting the catalyst aging based on the estimated parameters of the
saturated catalyst model. The performance of the proposed aging detector is demonstrated using both test-cell and
real-world truck data.


\input{secs/1-intro/3-cstr/0-cstr.tex}
