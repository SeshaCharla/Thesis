\chapter{Introduction}

Diesel engines offer several practical benefits, such as lower operating costs and higher fuel efficiency. However,
alongside these advantages, their combustion process produces emissions including unburnt hydrocarbons, carbon monoxide,
nitrogen oxides, and particulate matter, which are hazardous to health and the environment. Some of these pollutants
result from incomplete fuel combustion, combustion of engine lubricating oils and other non-hydrocarbon components,
while others are unavoidable byproducts of the combustion process.

To mitigate these pollutants, an aftertreatment system is installed between the engine outlet and the tailpipe. The
diesel engine aftertreatment system comprises several subsystems, each designed to target specific exhaust pollutants.
These subsystems include:
\begin{enumerate}
    \item Diesel Oxidation Catalyst (DOC): This component oxidizes unburnt hydrocarbons and carbon monoxide. It also converts NO to NO2, adjusting the NO/NO2 ratio from approximately 9:1 to 1:1 to enhance Selective Catalytic Reduction (SCR) performance \cite{hiremath2015development}.

    \item Diesel Particulate Filter (DPF): This filter removes particulate matter from the exhaust stream. It is positioned downstream of the DOC to utilize the oxidation heat for regeneration.

    \item Selective Catalytic Reduction (SCR) and Ammonia Slip Catalysis (ASC) System: This subsystem catalytically reduces nitrogen oxides (NOx) to nitrogen and water vapor.
\end{enumerate}
%============================================================
\begin{figure}[H]
    \centering
    \includegraphics[width=0.75\textwidth]{./figs/1-intro/SCR-ASC_Exhaust_Flow.png}
    \caption{Diesel Aftertreatment System Exhaust Flow Path and Sensor Locations \cite{jain2023model}}
    \label{fig:exhaust_flow}
\end{figure}
%============================================================
Figure~\ref{fig:exhaust_flow} illustrates the exhaust flow path and sensor locations within the aftertreatment system.
Both SCR and ASC are constructed as cylindrical bricks, with catalyst coated on the inner surfaces of hollow channels.
These channels are densely arranged to form the brick structure. Multiple SCR catalyst bricks are stacked within a metal
canister (Figure~\ref{fig:aftertreatment_system}), with the ASC brick positioned downstream. The placement and flow
distribution of these components are key design variables.
%============================================================
\begin{figure}[H]
    \centering
    \includegraphics[width=0.75\textwidth]{./figs/1-intro/SCR-ASC_chamber.png}
    \caption{Diesel Aftertreatment System}
    \label{fig:aftertreatment_system}
\end{figure}
% ============================================================
The primary function of the SCR system is to convert engine-out NOx into nitrogen and water using ammonia derived from
urea dosing, which is adsorbed onto the catalyst. Regulation of this catalytic reduction process aims to minimize
ammonia slip in tailpipe exhaust by two approaches. The first approach employs feedback control, adjusting urea dosing
based on NOx concentrations at the SCR inlet and tailpipe. The second approach utilizes Ammonia Slip Catalysis, which
oxidizes excess ammonia at the end of the SCR bed (Figure~\ref{fig:exhaust_scheme}).
% ============================================================
\begin{figure}[H]
        \centering
        \includegraphics[width=0.75\textwidth]{./figs/1-intro/SCR-ASC_model.png}
        \caption{Schematic of the SCR-ASC system}
        \label{fig:exhaust_scheme}
\end{figure}
% ============================================================
The NOx reduction capacity of the catalyst diminishes over time due to hydrothermal aging and the accumulation of
contaminants, including excess urea, platinum, sulfur, and phosphorus \cite{matsumoto2016model}. While a robust
urea-dosing controller may attempt to offset catalyst degradation
\cite{chen2016estimation},\cite{herman2010diagnostic},\cite{hu2018failure} by increasing urea dosing, this strategy can
accelerate aging and further elevate NOx and ammonia emissions. This has severe environmental and financial
implications. Cummins Inc. had to recall 500,000 medium and heavy-duty trucks in 2018 due to faster SCR catalyst
degradation, resulting in excessive NOx emissions \cite{jain2023model}. On-board diagnosis (OBD) of SCR catalyst aging
under real-world conditions would help meet increasingly stringent regulations. Some real-world limitations include the
unavailability of ammonia sensors, the unobservability of catalyst state, and NOx sensor cross-sensitivity to ammonia.

\noindent Accordingly, the principal contributions of this research are as follows:
\begin{enumerate}
    \item Development of a new discrete nonlinear switching model that describes the temporal evolution of sensor signals based on control volume and molar conservation principles.

    \item Transformation of the developed model into an identifiable framework for NOx reduction dynamics. Development of a novel parameter estimation algorithm for the switched nonlinear model. Validation of the model using test-cell data.

    \item Identification of parameters sensitive to aging within the nonlinear model. Development of a parameter estimation-based detector using statistical hypothesis testing for the saturated catalyst dynamic model. Validation of detector performance using both test-cell and real-world truck data.
\end{enumerate}

% ============================================================
\section{Aging Mechanisms and On-Board Diagnostics Review}

Catalyst aging occurs through several distinct mechanisms, each reducing the concentration of adsorption sites required for catalytic reduction. The literature outlines multiple methodologies for investigating these processes:
\begin{enumerate}
        \item \itbf{Hydrothermal Aging}: Exposure to heat and water vapor from the exhaust leads to irreversible structural and chemical changes on the catalyst surface, reducing ammonia storage capacity \cite{kim2007modeling}.

        \item \itbf{Chemical Deposits}: Deposits of sulphur, phosphorus, and urea-related compounds obstruct active sites for ammonia adsorption, thereby decreasing storage capacity \cite{kim2007modeling}.

        \item \itbf{Platinum Deposits}: Trace amounts of volatilized platinum from the diesel oxidation catalyst (DOC) can accumulate on the front section of the SCR catalyst \cite{toops2010deactivation},\cite{jen2009detection}.

        \item \itbf{Catalyst Formulation-Specific Aging} \cite{daya2018development}\cite{daya2020kinetic}: Aging mechanisms depend on the catalyst's chemical composition. Examples include dealumination in Fe-zeolite catalysts \cite{toops2010deactivation} and the sintering of anatase TiO2 (higher surface area per mass) to rutile TiO2 (lower surface area per mass) in titanium dioxide-based catalysts \cite{gieshoff2000improved}.
\end{enumerate}

Catalyst aging is spatially heterogeneous \cite{cheng2007laboratory}. On-engine dynamometer-accelerated aging results in
greater degradation at the front of the catalyst compared to the aft \cite{kim2007modeling}. This variation is
attributed not only to elevated temperatures \cite{toops2010deactivation} but also to the accumulation of metal and
chemical deposits \cite{kim2007modeling}. Cummins' research on high-fidelity aging models
\cite{daya2018development},\cite{daya2020kinetic} identifies several types of active sites for ammonia storage,
including Bronsted acid sites, copper sites, and physisorbed ammonia sites. Two primary sites, S1 and S2, are involved
in the aging process, and their behavior evolves as aging progresses. Mild aging increases S2 while maintaining a
constant sum of S1 and S2, whereas severe aging reduces both S1 and S2. The reaction rates for adsorption and desorption
remain unchanged during mild aging. Standard and fast SCR reactions are largely unaffected by aging
\cite{daya2018development}. The standard reaction predominantly occurs on S1 sites in a fresh catalyst and shifts to S2
sites in an aged catalyst. Ammonia oxidation increases with aging on S2 sites. Therefore, it can be concluded that aging
primarily reduces the concentration of adsorption sites while preserving similar reaction kinetics.

On-board aging diagnostics for SCR-ASC systems are categorized as intrusive or non-intrusive. Intrusive diagnostics
require intervention in the catalyst's normal operation, using controlled excitation \cite{herman2010diagnostic} of
specific system dynamics and comparing the resulting responses to a known reference
\cite{pezzini2009methodology},\cite{hu2017improving}. Non-intrusive diagnostics, by contrast, assess the aging state
without actively modifying system inputs.

Given the significant impact of On-Board Diagnostics (OBD) performance on public health and the environment, the Environmental Protection Agency (EPA) and California Air Resources Board (CARB) have established specific requirements for SCR systems \cite{jain2023model}. Only the regulations relevant to SCR are paraphrased here, as this is the focus of the present review:
\begin{enumerate}
        \item \itbf{Malfunction Criterion}: For 2016 and subsequent models, OBD must detect catalyst degradation during Supplemental Emission Test (SET) or Federal Test Procedure (FTP) for a degraded catalyst with NOx emission exceeding by 0.2 g/bhp-hr.

        \item \itbf{Enable Conditions}: The enable monitoring conditions must be designed such that:
        \begin{enumerate}
                \item OBD must be activated at least once in FTP
                \item The conditions are expected to occur during normal operations.
                \item The In-use Performance Ratio (IUMPR) must not be less than 0.3 for 2024 and subsequent models.
        \end{enumerate}

        \item Intrusive Diagnostics: Intrusive diagnostics are allowed if they have minimal impact on tailpipe emissions or if another diagnostics algorithm has already indicated a fault.
\end{enumerate}

The In-Use Performance Ratio (IUMPR) is formally defined as the ratio of the number of diagnostic algorithm activations
to the total number of drive cycles evaluated. The California Code of Regulations specifies a drive cycle as a trip that
satisfies any of the following four criteria: (1) initiation with engine start and termination with engine shut-off; (2)
initiation with engine start and conclusion after four hours of continuous engine operation; (3) initiation at the
conclusion of a preceding four-hour continuous engine operation period; or (4) initiation at the end of a previous
four-hour continuous engine-on period and termination with engine shut-off. This regulatory framework ensures consistent
assessment of diagnostic algorithm performance across varying operational scenarios.

\section{SCR-ASC Modelling: State of the Art}

A deductive approach to aging diagnostics models the dynamics of the signal-generating system, specifically the SCR-ASC
system, and incorporates aging effects by tracking changes in selected model parameters. Numerous studies have addressed
the modeling and control of SCR-ASC systems \cite{yuan2015diesel}. A common modeling strategy involves approximating the
partial differential equation (PDE) model for the plug flow reaction as a set of ordinary differential equations (ODEs)
using a series of Continuously Stirred Tank Reactors (CSTRs) (\cite{hsieh2011development}; \cite{nova2014urea}). The
number of CSTRs implemented often serves as an indicator of model fidelity. For instance, \cite{chen2016estimation},
\cite{ma2017observer}, \cite{devarakonda2009model}, \cite{hsieh2011development} employ four-cell CSTR models,
\cite{hu2018failure},  \cite{hu2017improving}, \cite{stadlbauer2015adaptive}, \cite{schar2004control} utilize three-cell
models, and \cite{jiang2019hydrothermal}, \cite{upadhyay2002modeling}, \cite{devarakonda2008adequacy} adopt two-cell
models. The single CSTR approach was initially justified in \cite{devarakonda2008adequacy}, where a nonlinear model was
developed based on these assumptions and subsequently linearized for feedback control design
(\cite{devarakonda2009model}). Observers have been developed to estimate the states associated with catalyst storage
(\cite{ma2017observer}; \cite{jain2020term}). A method for detecting catalyst aging by observing changes in the
catalyst's maximum storage capacity, modeled as an exponential function of temperature, was also introduced in
\cite{ma2017observer} and \cite{jiang2019hydrothermal}. Additionally, an on-board diagnostics (OBD) approach for
catalyst aging, utilizing tailpipe $NO_x$ and ammonia measurements, was proposed in \cite{matsumoto2016model}. This
method is based on the hypothesis of reduced adsorption sites and has been demonstrated on real-world trucks.

A persistent challenge in these studies is that system identification leads to a nonconvex, nonlinear parameter
estimation problem. Moreover, many approaches assume the availability of all gaseous states at the tailpipe for state
estimation and to mitigate cross-sensitivity effects in $NO_x$ sensors. This assumption is not valid in practical
applications, as tailpipe ammonia measurements are generally unavailable in commercial vehicles. Another notable
limitation of the CSTR model, regarding sensor sampling, is that its discretization does not consider reactant residence
time, which results in impractically high sampling frequencies. Furthermore, accurate discretization requires at least
two CSTRs to adequately capture system dynamics and causality \cite{charla2024reduced}, thereby increasing the model
order. The remainder of the section derives the CSTR model based on the three principal reactions and examines its
limitations.

%===========================================================

\section{Approach and Thesis Outline}
This research conceptualizes catalyst aging detection as a parameter estimation-based hypothesis testing problem,
employing the NOx reduction dynamics of the system. This framework leads to several sub-problems, which are addressed in
the subsequent chapters:
\begin{enumerate}
    \item Chapter 2 analyzes the limitations of sensor data and characterizes the associated noise.

    \item Chapters 3 and 4 present the development of a dynamic model for NOx reduction in the SCR-ASC system, with parameters that are both identifiable and validatable using experimental data.

    \item Chapter 5 identifies the subset of parameters sensitive to aging, develops estimators and detectors for aging diagnostics, and validates these methods using test-cell and truck data.
\end{enumerate}
The thesis concludes with a summary of contributions, an outline of future work, and a proposal for implementing the developed detector.
