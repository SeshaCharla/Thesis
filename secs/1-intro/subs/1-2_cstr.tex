\section{SCR-ASC Modelling: State of the Art}

A deductive approach to aging diagnostics models the dynamics of the signal-generating system, specifically the SCR-ASC
system, and incorporates aging effects by tracking changes in selected model parameters. Numerous studies have addressed
the modeling and control of SCR-ASC systems \cite{yuan2015diesel}. A common modeling strategy involves approximating the
partial differential equation (PDE) model for the plug flow reaction as a set of ordinary differential equations (ODEs)
using a series of Continuously Stirred Tank Reactors (CSTRs) (\cite{hsieh2011development}; \cite{nova2014urea}). The
number of CSTRs implemented often serves as an indicator of model fidelity. For instance, \cite{chen2016estimation},
\cite{ma2017observer}, \cite{devarakonda2009model}, \cite{hsieh2011development} employ four-cell CSTR models,
\cite{hu2018failure},  \cite{hu2017improving}, \cite{stadlbauer2015adaptive}, \cite{schar2004control} utilize three-cell
models, and \cite{jiang2019hydrothermal}, \cite{upadhyay2002modeling}, \cite{devarakonda2008adequacy} adopt two-cell
models. The single CSTR approach was initially justified in \cite{devarakonda2008adequacy}, where a nonlinear model was
developed based on these assumptions and subsequently linearized for feedback control design
(\cite{devarakonda2009model}). Observers have been developed to estimate the states associated with catalyst storage
(\cite{ma2017observer}; \cite{jain2020term}). A method for detecting catalyst aging by observing changes in the
catalyst's maximum storage capacity, modeled as an exponential function of temperature, was also introduced in
\cite{ma2017observer} and \cite{jiang2019hydrothermal}. Additionally, an on-board diagnostics (OBD) approach for
catalyst aging, utilizing tailpipe $NO_x$ and ammonia measurements, was proposed in \cite{matsumoto2016model}. This
method is based on the hypothesis of reduced adsorption sites and has been demonstrated on real-world trucks.

A persistent challenge in these studies is that system identification leads to a nonconvex, nonlinear parameter
estimation problem. Moreover, many approaches assume the availability of all gaseous states at the tailpipe for state
estimation and to mitigate cross-sensitivity effects in $NO_x$ sensors. This assumption is not valid in practical
applications, as tailpipe ammonia measurements are generally unavailable in commercial vehicles. Another notable
limitation of the CSTR model, regarding sensor sampling, is that its discretization does not consider reactant residence
time, which results in impractically high sampling frequencies. Furthermore, accurate discretization requires at least
two CSTRs to adequately capture system dynamics and causality \cite{charla2024reduced}, thereby increasing the model
order. The remainder of the section derives the CSTR model based on the three principal reactions and examines its
limitations.


\subsection{Continuous Stirred Tank Reactor Model}

\begin{figure}[!ht]
    \centering
    \includegraphics[width=0.5\textwidth]{./figs/1-intro/CSTR.png}
    \caption{Schematic of CSTR model for SCR-ASC System}
    \label{fig::cstr_schematic}
\end{figure}

The model development is based on specific assumptions that reflect typical operating conditions and system behavior.
The CSTR assumption underpins the construction of the reaction flow model for SCR reactions. Drawing on reduced order
modeling results from previous studies, including \cite{devarakonda2008adequacy} and \cite{jain2023diagnostics}, the
number of reactions considered is reduced from the full set to three (Table~\ref{tab::react_rates}). The analysis
focuses exclusively on the standard SCR reaction, with the assumption that all $NO_x$ in the exhaust gas is $NO$, since
commercially available $NO_x$ sensors cannot differentiate between $NO$ and $NO_2$ (\cite{nova2014urea}). The slow SCR
reaction is omitted, as the exhaust flow rate ensures it does not significantly affect the concentration of tailpipe
exhaust components. Mass transfer is neglected, indicating that the catalyst's chemical kinetics are reaction controlled
because the standard SCR reaction rate exceeds the exhaust fluid flow rate. A $100\%$ nitrogen selectivity for ammonia
oxidation is assumed for both SCR and ASC (\cite{jain2023diagnostics}). Furthermore, the reaction rates are assumed to
depend solely on the gas-phase concentrations of $NO_x$, $NH_3$, the adsorbed ammonia, and the available adsorption
sites.

\begin{table}[!ht]
    \begin{center}
    \caption{SCR-ASC Reactions and Rates Considerd for Model Development}
    \label{tab::react_rates}
    \begin{tabular}{|l|c|l|}
        \hline
        Reaction & Reaction Equation & Reaction Rate \\ \hline
        % ====
        Standard SCR &
        $4 NH_3 ^{(ads)} + 4 NO + O_2 \longrightarrow 4 N_2 + 6 H_2O$ &
        $r_{scr} = k_{scr} \con{NO_x}^{out} \con{NH_3}^{ads}$ \\ \hline
        % =======
        AMOX &
        $4 NH_3 + 3 O_2 \longrightarrow 2 N_2 + 6 H_2O $ &
        $r_{ox} = k_{ox} \con{NH_3}^{out}$ \\ \hline
        % =======
        Adsorption&
        $NH_3 + \theta_{free} \longrightarrow NH_3(ads)$ &
        $r_{ads} = k_{ads} \con{NH_3}^{out} \lr{\Gamma - \con{NH_3}^{ads}}$ \\ \hline
        %=======
        Desorption &
        $NH_3 ^{(ads)} \longrightarrow NH_3 + \theta_{free}$ &
        $r_{des} = k_{des} \con{NH_3}^{ads}$ \\ \hline
    \end{tabular}
    \end{center}
    *$\Gamma$ is the total adsorption site concentration.
\end{table}

In continuous stirred-tank reactors (CSTRs), reaction rates are governed by outlet concentrations, as defined by
reaction rate expressions. This relationship arises from the uniform mixing assumption at steady state: fluid with a
specified inlet concentration enters the CSTR, undergoes complete mixing, and reacts, resulting in outlet concentrations
that match those within the reactor (Figure~\ref{fig::cstr_schematic}). When a plug-flow reactor is modeled as a single
CSTR, the causality of the reaction rates is reversed. Employing multiple CSTRs introduces intermediate states, which
alleviates the impact of this causality reversal. As a result, a valid CSTR model generally contains more states than
can be measured, making the system unobservable. This limitation is one of the two key motivations behind the discrete
nonlinear model developed in this thesis.

One additonal consideration is the saturation of adsorption sites on the catalyst surface. The total concentration of adsorption sites, denoted as $\Gamma$, is the sum of occupied and free adsorption sites. Thus, at any give time,
\begin{align}
    \Gamma \geq \con{NH_3}^{ads} \geq 0
\end{align}
Thus, the system dynamics are conditioned on the saturation state of the catalyst. This phenomenon is captured in the model through the following switching function:

\begin{align}
    g_{sat} (\sigma) = \begin{cases}
        \sigma & \text{if } 0 \leq \sigma \leq \Gamma \\
        \Gamma & \text{if } \sigma > \Gamma \\
        0 & \text{if } \sigma < 0
    \end{cases}
    \label{eqn::g_sat}
\end{align}
We have dynamics of the concentration of reactants from the mass balance for the CSTR:
\begin{align}
    V_{scr} \dot{\con{NO_x}}^{out} &= F_{vol} \lr{\con{NO_x}^{in}-\con{NO_x}^{out}} - V_{scr} \lr{k_{scr} \con{NO_x}^{out} g_{sat}\lr{\con{NH_3}^{ads}}} \label{eqn::nox_mass_bal}\\
    % ====
    V_{scr} \dot{\con{NH_3}}^{out} &= F_{vol} \lr{\con{NH_3}^{in}-\con{NH_3}^{out}} \notag \\
    & \quad  + V_{scr}\lr{ - k_{ads} \con{NH_3}^{out} \lr{\Gamma - g_{sat}\lr{\con{NH_3}^{ads}}} + k_{des} \con{NH_3}^{ads}} \label{eqn::nh3_mass_bal}
    \\
    % ====
    \dot{\con{NH_3}}^{ads} &= k_{ads} \con{NH_3}^{out} \lr{\Gamma - g_{sat}\lr{\con{NH_3}^{ads}}}
                            - k_{des} \con{NH_3}^{ads}
                            - k_{scr} \con{NO_x}^{out} \con{NH_3}^{ads} \label{eqn::nh3_ads_mass_bal}
\end{align}

The actual input to the system is urea (AdBlue ($32.5\%$ aqueous urea solution) (\cite{nova2014urea}))
injection that is converted to ammonia. This can be modelled by the following equation~(\ref{eqn::urea_inj}).

\begin{equation}{\label{eqn::urea_inj}}
    \dot{\con{NH_3}}^{in} = \frac{1}{\tau_{urea}} \lrf{ - \con{NH_3}^{in} +   \frac{b_{urea}}{F_{vol}}  u_{inj}}
\end{equation}

The four equations presented constitute the nonlinear state-space model for the SCR-ASC system under the single CSTR
assumption. However, this model lacks parsimony and identifiability given the current measurement set. To obtain an
identifiable model, the nonlinear system can be linearized, and lumped parameters for input-output equations can be
derived based on the available measurements \cite{charla2024reduced}.

Furthermore, the hybrid dynamics modeled in the above equations capture the high-frequency reaction kinetics coupled
with mass flow. The slowest process, or dominant mode, can be approximated as a first-order system with residence time
as the time constant. For effective sampling of such a system, the sampling frequency should be n-times (atleast twice) the inverse of the residence time. Let, $\tau$ be the residence time fo the reacting flow through the SCR-ASC system, then the sampling frequency $f_s = 1/t_s$ should be:
\begin{align}
    f_s &\geq \frac{n}{\tau}
    \implies t_s \leq \frac{\tau}{n} \qquad n \geq 2
\end{align}
Thus, the CSTR model is valid only for high sampling frequencies. If we plot the residence time on the y-axis and the
sampling time on the x-axis, the valid region for the CSTR model lies in the shaded area of
Figure~\ref{fig::cstr_validity}.
\begin{figure}[!ht]
    \centering
    \includegraphics[width=0.6\textwidth]{./figs/1-intro/cstr_valid.png}
    \caption{Validity Region for CSTR Model On $\tau - t_s$ Plane}
    \label{fig::cstr_validity}
\end{figure}
The above analysis highlights the limitations of the CSTR model in practical applications, particularly regarding
sampling frequency requirements. This insight further motivates the development of alternative modeling approaches that
can accommodate lower sampling frequencies while still accurately capturing the dynamics of the SCR-ASC system.
