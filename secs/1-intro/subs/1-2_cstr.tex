\section{SCR-ASC Modelling: State of the Art}

A deductive approach to aging diagnostics models the dynamics of the signal-generating system, specifically the SCR-ASC
system, and incorporates aging effects by tracking changes in selected model parameters. Numerous studies have addressed
the modeling and control of SCR-ASC systems \cite{yuan2015diesel}. A common modeling strategy involves approximating the
partial differential equation (PDE) model for the plug flow reaction as a set of ordinary differential equations (ODEs)
using a series of Continuously Stirred Tank Reactors (CSTRs) (\cite{hsieh2011development}; \cite{nova2014urea}). The
number of CSTRs implemented often serves as an indicator of model fidelity. For instance, \cite{chen2016estimation},
\cite{ma2017observer}, \cite{devarakonda2009model}, \cite{hsieh2011development} employ four-cell CSTR models,
\cite{hu2018failure},  \cite{hu2017improving}, \cite{stadlbauer2015adaptive}, \cite{schar2004control} utilize three-cell
models, and \cite{jiang2019hydrothermal}, \cite{upadhyay2002modeling}, \cite{devarakonda2008adequacy} adopt two-cell
models. The single CSTR approach was initially justified in \cite{devarakonda2008adequacy}, where a nonlinear model was
developed based on these assumptions and subsequently linearized for feedback control design
(\cite{devarakonda2009model}). Observers have been developed to estimate the states associated with catalyst storage
(\cite{ma2017observer}; \cite{jain2020term}). A method for detecting catalyst aging by observing changes in the
catalyst's maximum storage capacity, modeled as an exponential function of temperature, was also introduced in
\cite{ma2017observer} and \cite{jiang2019hydrothermal}. Additionally, an on-board diagnostics (OBD) approach for
catalyst aging, utilizing tailpipe $NO_x$ and ammonia measurements, was proposed in \cite{matsumoto2016model}. This
method is based on the hypothesis of reduced adsorption sites and has been demonstrated on real-world trucks.

A persistent challenge in these studies is that system identification leads to a nonconvex, nonlinear parameter
estimation problem. Moreover, many approaches assume the availability of all gaseous states at the tailpipe for state
estimation and to mitigate cross-sensitivity effects in $NO_x$ sensors. This assumption is not valid in practical
applications, as tailpipe ammonia measurements are generally unavailable in commercial vehicles. Another notable
limitation of the CSTR model, regarding sensor sampling, is that its discretization does not consider reactant residence
time, which results in impractically high sampling frequencies. Furthermore, accurate discretization requires at least
two CSTRs to adequately capture system dynamics and causality \cite{charla2024reduced}, thereby increasing the model
order. The remainder of the section derives the CSTR model based on the three principal reactions and examines its
limitations.
