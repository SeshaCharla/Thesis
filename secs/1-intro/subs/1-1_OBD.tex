\section{Aging Mechanisms and On-Board Diagnostics Review}

Catalyst aging occurs through several distinct mechanisms, each reducing the concentration of adsorption sites required for catalytic reduction. The literature outlines multiple methodologies for investigating these processes:
\begin{enumerate}
        \item \itbf{Hydrothermal Aging}: Exposure to heat and water vapor from the exhaust leads to irreversible structural and chemical changes on the catalyst surface, reducing ammonia storage capacity \cite{kim2007modeling}.

        \item \itbf{Chemical Deposits}: Deposits of sulphur, phosphorus, and urea-related compounds obstruct active sites for ammonia adsorption, thereby decreasing storage capacity \cite{kim2007modeling}.

        \item \itbf{Platinum Deposits}: Trace amounts of volatilized platinum from the diesel oxidation catalyst (DOC) can accumulate on the front section of the SCR catalyst \cite{toops2010deactivation},\cite{jen2009detection}.

        \item \itbf{Catalyst Formulation-Specific Aging} \cite{daya2018development}\cite{daya2020kinetic}: Aging mechanisms depend on the catalyst's chemical composition. Examples include dealumination in Fe-zeolite catalysts \cite{toops2010deactivation} and the sintering of anatase TiO2 (higher surface area per mass) to rutile TiO2 (lower surface area per mass) in titanium dioxide-based catalysts \cite{gieshoff2000improved}.
\end{enumerate}

Catalyst aging is spatially heterogeneous \cite{cheng2007laboratory}. On-engine dynamometer-accelerated aging results in
greater degradation at the front of the catalyst compared to the aft \cite{kim2007modeling}. This variation is
attributed not only to elevated temperatures \cite{toops2010deactivation} but also to the accumulation of metal and
chemical deposits \cite{kim2007modeling}. Cummins' research on high-fidelity aging models
\cite{daya2018development},\cite{daya2020kinetic} identifies several types of active sites for ammonia storage,
including Bronsted acid sites, copper sites, and physisorbed ammonia sites. Two primary sites, S1 and S2, are involved
in the aging process, and their behavior evolves as aging progresses. Mild aging increases S2 while maintaining a
constant sum of S1 and S2, whereas severe aging reduces both S1 and S2. The reaction rates for adsorption and desorption
remain unchanged during mild aging. Standard and fast SCR reactions are largely unaffected by aging
\cite{daya2018development}. The standard reaction predominantly occurs on S1 sites in a fresh catalyst and shifts to S2
sites in an aged catalyst. Ammonia oxidation increases with aging on S2 sites. Therefore, it can be concluded that aging
primarily reduces the concentration of adsorption sites while preserving similar reaction kinetics.

On-board aging diagnostics for SCR-ASC systems are categorized as intrusive or non-intrusive. Intrusive diagnostics
require intervention in the catalyst's normal operation, using controlled excitation \cite{herman2010diagnostic} of
specific system dynamics and comparing the resulting responses to a known reference
\cite{pezzini2009methodology},\cite{hu2017improving}. Non-intrusive diagnostics, by contrast, assess the aging state
without actively modifying system inputs.

Given the significant impact of On-Board Diagnostics (OBD) performance on public health and the environment, the Environmental Protection Agency (EPA) and California Air Resources Board (CARB) have established specific requirements for SCR systems \cite{jain2023model}. Only the regulations relevant to SCR are paraphrased here, as this is the focus of the present review:
\begin{enumerate}
        \item \itbf{Malfunction Criterion}: For 2016 and subsequent models, OBD must detect catalyst degradation during Supplemental Emission Test (SET) or Federal Test Procedure (FTP) for a degraded catalyst with NOx emission exceeding by 0.2 g/bhp-hr.

        \item \itbf{Enable Conditions}: The enable monitoring conditions must be designed such that:
        \begin{enumerate}
                \item OBD must be activated at least once in FTP
                \item The conditions are expected to occur during normal operations.
                \item The In-use Performance Ratio (IUMPR) must not be less than 0.3 for 2024 and subsequent models.
        \end{enumerate}

        \item Intrusive Diagnostics: Intrusive diagnostics are allowed if they have minimal impact on tailpipe emissions or if another diagnostics algorithm has already indicated a fault.
\end{enumerate}

The In-Use Performance Ratio (IUMPR) is formally defined as the ratio of the number of diagnostic algorithm activations
to the total number of drive cycles evaluated. The California Code of Regulations specifies a drive cycle as a trip that
satisfies any of the following four criteria: (1) initiation with engine start and termination with engine shut-off; (2)
initiation with engine start and conclusion after four hours of continuous engine operation; (3) initiation at the
conclusion of a preceding four-hour continuous engine operation period; or (4) initiation at the end of a previous
four-hour continuous engine-on period and termination with engine shut-off. This regulatory framework ensures consistent
assessment of diagnostic algorithm performance across varying operational scenarios.
